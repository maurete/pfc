\documentclass[a4paper,11pt,oneside]{article}
%
% importar el archivo conf/packages.tex
\include{conf/packages}
%
% ===
% === Propiedades del documento: título, autor, etc
% ===
%
\newcommand{\titulo}{{\large FICH --- UNL}\\Métodos Numéricos y
  Simulación -- 2012\\{Guía de Trabajos Prácticos 3}}
\newcommand{\autor}{Torrez, Mauro}
\newcommand{\fecha}{\today}
\newcommand{\tituloPDF}{MNS GTP3}
\newcommand{\autorPDF}{Mauro Torrez}
\newcommand{\asuntoPDF}{MNS GTP3}
\newcommand{\clavesPDF}{MNS GTP3}
%
% importar los archivos conf/config.tex y conf/comandos.tex
\include{conf/config}
\include{conf/comandos}
\include{conf/fuentes}
%
% ===
% === Inicio del documento
% === 
%
\begin{document}
% crear la página de título
\maketitle
%
\section*{Ejercicio 4}
%
Resuelvo el problema
\begin{align*}
  \begin{cases}
    \dx{\phi}{2}-\phi=0, & 0\leq x\leq 1,\\
    \phi(0)=0,\\
    \phi(1)=1
  \end{cases}
\end{align*}
mediante el método de elementos finitos, con una partición de M elementos en $x$.

Planteo mediante residuos ponderados
\begin{align*}
  \int_0^1 W_l \left( \dx{\hfi}{2}-\hfi \right)dx=0,\qquad l=1,2,\ldots,M+1,\\
%% \end{align*}
%% donde 
%% \begin{align*}
  \hfi = \sum_{m=1}^{M+1}\phi_m N_m.
\end{align*}
%
%\begin{nota}
  Para este problema particular, ignoro explícitamente el residuo en el contorno, ya
  que el valor de los nodos en los extremos es conocido, y las funciones de forma
  que utilizaré son de soporte compacto.
%\end{nota}
%

La forma débil de esta integral será entonces
\begin{align*}
  -\int_0^1 \left(\dx{W_l}{}\dx{\hfi}{}+W_l\hfi\right)dx + \left[W_l\dx{\hfi}{}\right]_0^1=0
\end{align*}
Otra vez, ignoro la evaluación en el contorno:
\begin{align*}
  -\int_0^1 \left(\dx{W_l}{}{}\sum\phi_m\dx{N_m}{}+W_l\sum\phi_mN_m\right)dx=0.
\end{align*}
%
Expresando esto como $\B{K\phi=f}$, tengo las componentes de $\B{K}$ y $\B{f}$
\begin{align*}
  \B{K}_{lm}&=\int_0^1 \left(\dx{W_l}{}{}\dx{N_m}{}+W_lN_m\right)dx, &
  \B{f}_{l}&=\left[
    \begin{array}{c}
      \left.-\dx{\hfi}{}\right|_{x=0}\\0\\\vdots\\0\\\left.\dx{\hfi}{}\right|_{x=1}
    \end{array}
    \right],&1\leq l,m\leq M+1.
\end{align*}
Uso Galerkin, entonces queda
\begin{align*}
  \B{K}_{lm}&=\int_0^1 \left(\dx{N_l}{}{}\dx{N_m}{}+N_lN_m\right)dx, &1\leq l,m\leq M+1.
\end{align*}
%
%
\subsection*{Assemble}
Pasamos ahora a ver el problema desde el punto de vista elemental. Para cada elemento
$e$, la matriz elemental correspondiente se define como
\begin{align*}
  \B{K}^e_{l'm'}&=\int_0^{h^e} \left(\dx{N^e_{l'}}{}{}\dx{N^e_{m'}}{}+N^e_{l'}N^e_{m'}\right)dx, &l',m'=1,2.
\end{align*}
La contribución de estas matrices elementales a la matriz global viene dada por
\begin{align*}
  \B{K}_{lm} = \bigwedge_{l'm'}\B{K}^e_{l'm'}
\end{align*}
donde $\bigwedge$ es el operador de ensamblaje.
%
\subsubsection*{Contribución elemental}
Para el cálculo de la contribución elemental, defino las funciones de forma
$N^e_1$ y $N^e_2$ en el dominio elemental normalizado
\begin{align*}
  \hat\Omega^e=\left\{\xi|\xi\in[-1,1]\right\}
\end{align*}
con
\begin{align*}
  \xi=\frac{2(x-x_c^e)}{h^e}
\end{align*}
donde $x_c^e$ es la coordenada central del elemento, $h^e$ la longitud del elemento, y entonces
el elemento se define en el rango $-1\leq\xi\leq 1$.

Ahora defino las $N^e_{l}$ según
\begin{align*}
  N_1^e &= \frac{1-\xi}{2},& N_2^e &= \frac{\xi+1}{2}.
\end{align*}
Teniendo en cuenta que
\begin{align*}
  \dx{N_{l'}^e}{}=\dvar{N_{l'}^e}{\xi}{}\dx{\xi}{}=\frac{2}{h^e}\dvar{N_{l'}^e}{\xi}{},\qquad dx=\frac{h^e}{2}d\xi
\end{align*}
calculo la matriz
\begin{align*}
  \B{K}^e_{l'm'}&=\int_0^{h^e} \left(\dx{N^e_{l'}}{}{}\dx{N^e_{m'}}{}+N^e_{l'}N^e_{m'}\right)dx\\
  &=\int_{-1}^{1} \left(\frac{2}{h^e}\dvar{N^e_{l'}}{\xi}{}\frac{2}{h^e}\dvar{N^e_{m'}}{\xi}{}+N^e_{l'}N^e_{m'}\right)\frac{h^e}{2}d\xi\\
  &=\int_{-1}^{1} \left(\frac{2}{h^e}\dvar{N^e_{l'}}{\xi}{}\dvar{N^e_{m'}}{\xi}{}+\frac{h^e}{2}N^e_{l'}N^e_{m'}\right)d\xi
\end{align*}
entonces
\begin{align*}
\B{K}^e_{11}&=\int_{-1}^{1} \left(\frac{2}{h^e}\frac{-1}{2}\frac{-1}{2}+\frac{h^e}{2}\frac{1-\xi}{2}\frac{1-\xi}{2}\right)d\xi=\int_{-1}^{1} \left(\frac{1}{2h^e}+\frac{h^e(1-2\xi+\xi^2)}{8}\right)d\xi\\
&=\left(\frac{1}{2h^e}+\frac{h^e}{8}\right)\int_{-1}^{1}d\xi
- \frac{h^e}{4}\int_{-1}^{1}\xi d\xi + \frac{h^e}{8} \int_{-1}^{1}\xi^2 d\xi\\
&=\left(\frac{1}{2h^e}+\frac{h^e}{8}\right) (2)
- \frac{h^e}{4}(0) + \frac{h^e}{8}\left( \frac{2}{3}\right)=\frac{1}{h^e}+\frac{h^e}{4}
+ \frac{h^e}{12}\\%
&=\frac{1}{h^e}
+ \frac{h^e}{3}\\
\B{K}^e_{22}&=\int_{-1}^{1} \left(\frac{2}{h^e}\frac{1}{2}\frac{1}{2}+\frac{h^e}{2}\frac{\xi+1}{2}\frac{\xi+1}{2}\right)d\xi=\int_{-1}^{1} \left(\frac{1}{2h^e}+\frac{h^e(1+2\xi+\xi^2)}{8}\right)d\xi\\
&\;\;\vdots\\
&=\frac{1}{h^e}+ \frac{h^e}{3}\\
\B{K}^e_{12}=\B{K}^e_{21}&=\int_{-1}^{1} \left(\frac{2}{h^e}\frac{1}{2}\frac{-1}{2}+\frac{h^e}{2}\frac{1-\xi}{2}\frac{\xi+1}{2}\right)d\xi
=\int_{-1}^{1} \left(\frac{-1}{2h^e}+\frac{h^e(1-\xi^2)}{8}\right)d\xi\\
&=\left(\frac{-1}{2h^e}+\frac{h^e}{8}\right)\int_{-1}^{1}d\xi-\frac{h^e}{8}\int_{-1}^{1}\xi^2d\xi\\
&=\left(\frac{-1}{2h^e}+\frac{h^e}{8}\right)(2)-\frac{h^e}{8}\left(\frac{2}{3}\right)=-\frac{1}{h^e}+\frac{h^e}{4}-\frac{h^e}{12}\\
&=-\frac{1}{h^e}+\frac{h^e}{6}.
\end{align*}
%
%
\subsubsection*{Gather}
La matriz global $\B{K}$ se forma sumando en la posición $(l,l), 1\leq l\leq M$ la
contribución de la matriz elemental $\B{K}^l$, esto es, para cada $l=1\ldots M$:
\begin{align*}
  \B{K}_{l,l}&=\B{K}^l_{1,1},\\
\B{K}_{l,l+1}&=\B{K}^l_{1,2},\\
\B{K}_{l+1,l}&=\B{K}^l_{2,1},\\
\B{K}_{l+1,l+1}&=\B{K}^l_{2,2}.
\end{align*}
\subsection*{Solve}
La solución al sistema se obtiene resolviendo el sistema de ecuaciones armado
en las secciones anteriores. Por ejemplo, para un M=4, tendremos
\begin{align*}
  \left[
    \begin{array}{ccccc}
      \frac{1}{h^e}+\frac{h^e}{3} & -\frac{1}{h^e}+\frac{h^e}{6} & 0 & 0 & 0 \\
       -\frac{1}{h^e}+\frac{h^e}{6} & 2\left(\frac{1}{h^e}+\frac{h^e}{3}\right) &
       -\frac{1}{h^e}+\frac{h^e}{6} & 0 & 0 \\
       0 & -\frac{1}{h^e}+\frac{h^e}{6} & 2\left(\frac{1}{h^e}+\frac{h^e}{3}\right) &
       -\frac{1}{h^e}+\frac{h^e}{6} & 0\\
       0 & 0 & -\frac{1}{h^e}+\frac{h^e}{6} & 2\left(\frac{1}{h^e}+\frac{h^e}{3}\right) &
       -\frac{1}{h^e}+\frac{h^e}{6}\\
       0 & 0 & 0 & -\frac{1}{h^e}+\frac{h^e}{6} & \frac{1}{h^e}+\frac{h^e}{3}
    \end{array}
\right]
  \left[
    \begin{array}{c}
      \phi_1 \\ \phi_2 \\ \phi_3 \\ \phi_4\\ \phi_5
    \end{array}
\right]
=  \left[
    \begin{array}{c}
      \left.-\dx{\hfi}{}\right|_{x=0}\\0 \\ 0\\ 0\\\left.\dx{\hfi}{}\right|_{x=1}
    \end{array}
\right]
\end{align*}
En este problema en particular, los valores de $\phi_1$ y $\phi_{5}$ son conocidos,
así que eliminando las primeras y últimas filas y columnas del sistema y estableciendo
a mano el valor de los respectivos nodos, queda resolver el sistema reducido
\begin{align*}
  \left[
    \begin{array}{ccc}
        2\left(\frac{1}{h^e}+\frac{h^e}{3}\right) &
       -\frac{1}{h^e}+\frac{h^e}{6} & 0  \\
        -\frac{1}{h^e}+\frac{h^e}{6} & 2\left(\frac{1}{h^e}+\frac{h^e}{3}\right) &
       -\frac{1}{h^e}+\frac{h^e}{6} \\
        0 & -\frac{1}{h^e}+\frac{h^e}{6} & 2\left(\frac{1}{h^e}+\frac{h^e}{3}\right) 
    \end{array}
\right]
  \left[
    \begin{array}{c}
       \phi_2 \\ \phi_3 \\ \phi_4
    \end{array}
\right]
=  \left[
    \begin{array}{c}
      \phi_1\left(\frac{1}{h^e}+\frac{h^e}{6}\right)\\ 0\\\phi_5\left(\frac{1}{h^e}+\frac{h^e}{6}\right)
    \end{array}
\right]
\end{align*}
\newpage
\section*{Ejercicio 5}
\subsection*{a.}
\consigna{%
Analizar la barra de sección constante con una fuerza distribuida
$b(x)$ a lo largo de su longitud la cual tiene el valor $L$ y
además se encuentra sometida a una fuerza puntual $P$. La sección
transversal es constante y de valor $A$. Calcular los desplazamientos,
tensiones y deformaciones.
}

El Principio de los Trabajos Virtuales establece que
\begin{align*}
  \iiint_V \delta\epsilon\,\sigma\,d\!V = \int_0^L \delta u\,b\,d\!x + \sum_i \delta u_i\, X_i
\end{align*}
%
dode $\delta u$ y $\delta \epsilon$ son el desplazamiento y deformación virtual genéricos
de un punto de la barra, y $\delta u_i$ es el desplazamiento virtual del punto de actuación
de la fuerza $X_i$.

Integrando sobre la sección transversal, y teniendo en cuenta que $\sigma=E\epsilon=E\dx{u}{}$ con $E$ el módulo de Young, tengo
\begin{align*}
  \int_0^L \delta\epsilon\,A E\dx{u} \,d\!x   &= \int_0^L \delta u\,b\,d\!x + \sum_i \delta u_i\, X_i .
\end{align*}
Ésta es la ecuación de equilibrio que debo resolver para $u$.
%
\subsubsection*{Contribución elemental}
Como en el ejecicio anterior, considero dentro del elemento el dominio normalizado
\begin{align*}
  \hat\Omega^e=\left\{\xi|\xi\in[-1,1]\right\}, && \T{con } && \xi=\frac{2(x-x_c^e)}{h^e}.
\end{align*}
Otra vez, $x_c^e$ es la coordenada central del elemento, $h^e$ la longitud del elemento, y entonces
el dominio elemental será el rango $-1\leq\xi\leq 1$.
Las funciones de forma $N^e_{l}$ serán entonces
\begin{align*}
  N_1^e &= \frac{1-\xi}{2},& N_2^e &= \frac{\xi+1}{2}.
\end{align*}
La solución $u$ la aproximamos según
\begin{align*}
  u = N_1^e u_1 + N_2^e u_2
\end{align*}
con $u_1$, $u_2$ el valor de u en los nodos 1, 2. El strain viene dado por
\begin{align*}
  \epsilon &= \dx{u}{} = \dvar{u}{\xi}{} \dvar{\xi}{x}{} = \frac{2}{h^e} \dvar{u}{\xi}{}
                      = \frac{2}{h^e} \left(\dvar{N_1^e}{\xi}{} u_1 + \dvar{N_2^e}{\xi}{} u_2\right)
                      = \frac{2}{h^e} \left(\frac{-1}{2} u_1 + \frac{1}{2} u_2\right) \\
  \epsilon &= \frac{1}{h^e}(u_2-u_1).
\end{align*}

Las fuerzas entre elementos se transmiten únicamente a través de los nodos [Oñate,p2.6].
Dichas fuerzas, que llamaremos ``de equilibrio'', pueden calcularse para cada elemento
haciendo uso del Principio de los Trabajos Virtuales según
\begin{align*}
  \int_{-1}^{1} \delta\epsilon\,A E\epsilon \,d\!\xi   &= \int_{-1}^{1} \delta u\,b\,d\!\xi + \delta u_1\, X_1 + \delta u_2\, X_2 ,
\end{align*}
donde $\delta u_1$, $\delta u_2$, $X_1$, $X_2$ son los desplazamientos virtuales y las fuerzas d e equilibrio en los nodos 1 y 2. Interpolamos el desplazamiento virtual dentro del elemento
\begin{align*}
  \delta u = N_1^e \delta u_1 + N_2^e \delta u_2
\end{align*}
y el strain virtual será
\begin{align*}
  \delta \epsilon = \dx{}{}(\delta u) = \frac{2}{h^e}\left( \dvar{N_1^e}{\xi}{} \delta u_1 + \dvar{N_2^e}{\xi}{} \delta u_2 \right).
\end{align*}
Volviendo al PTV para el elemento, tengo
\begin{align*}
 \begin{split}
    \left(\frac{2}{h^e}\right)^2 \int_{-1}^{1} \left(\dvar{N_1^e}{\xi}{}\delta u_1+\dvar{N_2^e}{\xi}{}\delta u_2\right)
         \,A E \left(\dvar{N_1^e}{\xi}{}u_1+\dvar{N_2^e}{\xi}{}u_2\right) \,d\!\xi  \qquad \qquad\\ \qquad \qquad
    = \int_{-1}^{1}(N_1^e \delta u_1 + N_2^e \delta u_2) \,b\,d\!\xi + \delta u_1\, X_1 + \delta u_2\, X_2
  \end{split}
\end{align*}
Reagrupando los términos,
\begin{align*}
 \begin{split}
   \delta u_1 \left[ \left(\frac{2}{h^e}\right)^2 \int_{-1}^{1} \dvar{N_1^e}{\xi}{}
         \,A E \left(\dvar{N_1^e}{\xi}{}u_1+\dvar{N_2^e}{\xi}{}u_2\right) \,d\!\xi
         - \int_{-1}^{1}N_1^e \,b\,d\!\xi - \, X_1 \right] \qquad \qquad\\ \qquad \qquad
    + \delta u_2 \left[ \left(\frac{2}{h^e}\right)^2 \int_{-1}^{1} \dvar{N_2^e}{\xi}{}
         \,A E \left(\dvar{N_1^e}{\xi}{}u_1+\dvar{N_2^e}{\xi}{}u_2\right) \,d\!\xi
         - \int_{-1}^{1}N_2^e \,b\,d\!\xi -  X_2 \right] =0
 \end{split}
\end{align*}
Como los desplazamientos virtuales son arbitrarios, las expresiones entre corchetes deberán
ser nulas. Esto se traduce en el sistema de ecuaciones
\begin{align*}
  \left(\frac{2}{h^e}\right)^2 \int_{-1}^{1} \left( \dvar{N_1^e}{\xi}{} A E \dvar{N_1^e}{\xi}{}u_1
      + \dvar{N_1^e}{\xi}{} A E \dvar{N_2^e}{\xi}{}u_2\right) \,d\!\xi
      - \int_{-1}^{1}N_1^e \,b\,d\!\xi - X_1 &=0 \\ 
  \left(\frac{2}{h^e}\right)^2 \int_{-1}^{1} \left( \dvar{N_2^e}{\xi}{} A E \dvar{N_1^e}{\xi}{}u_1
      + \dvar{N_2^e}{\xi}{} A E \dvar{N_2^e}{\xi}{}u_2\right) \,d\!\xi
      - \int_{-1}^{1}N_2^e \,b\,d\!\xi - X_2  &=0,
\end{align*}
en forma matricial,
\begin{align*}
  \left(\frac{2}{h^e}\right)^2
  \left[
    \begin{array}{cc}
      \int_{-1}^{1} \dvar{N_1^e}{\xi}{} A E \dvar{N_1^e}{\xi}{} \,d\!\xi &
      \int_{-1}^{1} \dvar{N_1^e}{\xi}{} A E \dvar{N_2^e}{\xi}{} \,d\!\xi \\
      \int_{-1}^{1} \dvar{N_2^e}{\xi}{} A E \dvar{N_1^e}{\xi}{} \,d\!\xi &
      \int_{-1}^{1} \dvar{N_2^e}{\xi}{} A E \dvar{N_2^e}{\xi}{} \,d\!\xi
   \end{array}
    \right]
\left[
  \begin{array}{c}
    u_1 \\ u_2
  \end{array}
\right] = 
\left[
  \begin{array}{c}
    \int_{-1}^{1}N_1^e \,b\,d\!\xi + X_1 \\
    \int_{-1}^{1}N_2^e \,b\,d\!\xi + X_2
  \end{array}
\right].
\end{align*}
Considerando $A$, $E$, $b$ como constantes e integrando
\begin{align*}
  AE \left(\frac{2}{h^e}\right)^2
  \left[
    \begin{array}{cc}
      \int_{-1}^{1} \frac{-1}{2}\frac{-1}{2} \,d\!\xi &
      \int_{-1}^{1} \frac{-1}{2}\frac{1}{2}  \,d\!\xi \\
      \int_{-1}^{1} \frac{1}{2} \frac{-1}{2} \,d\!\xi &
      \int_{-1}^{1} \frac{1}{2} \frac{1}{2}  \,d\!\xi
   \end{array}
    \right]
\left[
  \begin{array}{c}
    u_1 \\ u_2
  \end{array}
\right] =
\left[
  \begin{array}{c}
   b \int_{-1}^{1}N_1^e \,d\!\xi + X_1 \\
   b \int_{-1}^{1}N_2^e \,d\!\xi + X_2
  \end{array}
\right]\\
  AE \left(\frac{2}{h^e}\right)^2
  \left[
    \begin{array}{cc}
      \frac{1}{2}  &
      \frac{-1}{2} \\
      \frac{-1}{2} &
      \frac{1}{2}
   \end{array}
    \right]
\left[
  \begin{array}{c}
    u_1 \\ u_2
  \end{array}
\right] =
\left[
  \begin{array}{c}
   b + X_1 \\
   b + X_2
  \end{array}
\right]
\end{align*}





\end{document}
