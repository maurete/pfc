\documentclass[a4paper,11pt,oneside]{article}
%
% importar el archivo conf/packages.tex
%
% ===
% === Trick para detectar si el documento está siendo compilado con pdflatex
% ===
%
% Esto me setea la variable pdf dependiendo del valor de \pdfoutput, que es >0
% sólo cuando estoy usando pdflatex para compilar el documento. Con esto puedo
% hacer  \ifpdf {...} \fi, que se ejecuta colo cuando compilo con pdflatex.
%% \newif\ifpdf
%% \ifnum\pdfoutput<0
%% \pdffalse\fi
%% \ifnum\pdfoutput=0
%% \pdffalse\fi
%% \ifnum\pdfoutput>0
%% \pdftrue\fi
%
% ===
% === I18n / L10n
% ===
%
% babel me da separación de sílabas para palabras en el idioma que le paso como
%       argumento opcional.
\usepackage[spanish,es-tabla,english]{babel}
%
% inputenc define la codificación de caracteres del código fuente, acá utf8.
\usepackage[utf8]{inputenc}
%
% ===
% === Gráficos
% ===
% 
% pst-pdf me permite usar PSTricks con pdflatex. Necesito cargarlo sólo si está
%         definida la variable pdf, por eso está entre \ifpdf ... \fi
%\ifpdf\usepackage{pst-pdf}\fi
%
% color me permite usar colores en el documento.
\usepackage{color}
%
% graphicx me da el comando \includegraphics para insertar imágenes (?)
\usepackage{graphicx}
%
% pstricks es un conjunto de macros basadas en PostScript para TeX, en
%          castellano: me da un entorno pstricks y comandos que uso dentro de
%          éste, que me sirven para dibujar figuras/diagramas/etc de manera
%          relativamente simple.
%\usepackage{pstricks}
%
% pst-circ me da macros para pstricks que me dibujan elementos de circuitos
%\usepackage{pst-circ}
%
% pst-plot me provee de funciones de ploteo para pstricks
%\usepackage{pst-plot}
%
% pst-2dplot me sirve para plotear en pstricks, entorno pstaxes
%\usepackage{pst-2dplot}
%
% ===
% === Verbatims
% ===
%
% verbatim es una reimplementación de los entornos verbatim[*]
%          provee el comando \verbatiminput{archivo} y el entorno comment, que
%          hace que LaTeX ignore directamente todo lo que está adentro
%\usepackage{verbatim}
%
% moreverb implementa el entorno verbatimtab indentando los tabs que encuentre,
%          y también el entorno listing, que pone números de línea al verbatim.
%          Para cambiar el ancho de la tabulacion, uso
%          \renewcommand\verbatimtabsize{<ancho del tab>\relax}
%          También define el entorno boxedverbatim.
%\usepackage{moreverb}
%
% listings me da el entorno lstlisting con resaltado de sintaxis.
%          Para setear el lenguaje del código, hago \lstset{language=<lang>}
%\usepackage{listings}
%
% url es un verbatim para escribir URL's que permite linebreaks dentro de ésta.
%     para usarlo, \url{<URL>}
\usepackage{url}
%
% ===
% === Más packages
% ===
%
%% \usepackage{mdwlist}		%Para listas mas compactas
%% \usepackage{textcomp}		%Para algunos símbolos
%% \usepackage{colortbl}		%Para celdas de colores en tablas
%% \usepackage{fancyhdr}		%Para encabezados/pie
\usepackage{bbold}		%Fuente bb para modo math: \mathbb{R} = reales
\usepackage{dsfont}		%Fuente ds para modo math: \mathds{R} = reales
\usepackage{multirow}		%Para "combinar" celdas en tablas
\usepackage{float}		%Para mejorar cuadros, figuras, etc
%% \usepackage{fancybox}		%Para recuardos de texto con bordes "fancy"
%% \usepackage{dingbat}		%Para dingbats
%\usepackage{marginal}		%Para  notas al margen que no puedo hacer andar
\usepackage{amsmath}		%Para enornos matemáticos mas flexibles
%\usepackage{varwidth}		%varwidth es un minipage que se ajusta al ancho mínimo


\usepackage[backend=biber,sorting=none,style=ieee,eprint=false,url=false]{biblatex} %% style=ieee
%% requiere texlive-bibtex-extra en debian


\usepackage{enumitem}
\setlist{noitemsep}
%% \setlist[description]{noitemsep}
%% \setlist[enumerate]{noitemsep}
%% \setlist[itemize]{noitemsep}

\usepackage{tikz}
\usepackage{pgfkeys}
\usepackage{pgfgantt}

% typearea: uso con koma-script para ajustar márgenes de página.
% vars globales a setear en la clase koma-script: DIV=12, BCOR=margen de ``binding'' para double side
\usepackage{typearea}

% para poder usar footnotes p.ej, adentro de un tabular
\usepackage{footnote}
\makesavenoteenv{tabular}

% para tabulars mas lindos/legibles
\usepackage{booktabs}

%\usepackage{glossaries}

\usepackage[spanish]{algorithm2e}

% para highlight (comando \hl{})
\usepackage{soulutf8}

% para teoremans etc
\usepackage{amsthm}

% para tunear citations
%\usepackage[square,comma,numbers,sort&compress]{natbib}


%
% ===
% === Propiedades del documento: título, autor, etc
% ===
%
\newcommand{\titulo}{{\large FICH --- UNL}\\Métodos Numéricos y
  Simulación -- 2012\\{Guía de Trabajos Prácticos 3}}
\newcommand{\autor}{Torrez, Mauro}
\newcommand{\fecha}{\today}
\newcommand{\tituloPDF}{MNS GTP3}
\newcommand{\autorPDF}{Mauro Torrez}
\newcommand{\asuntoPDF}{MNS GTP3}
\newcommand{\clavesPDF}{MNS GTP3}
%
% importar los archivos conf/config.tex y conf/comandos.tex
% config.tex: configuraciones del documento
%\selectlanguage{spanish}		%Elijo idioma español

%Permitir que los entornos equation, align, etc permitan saltos de página
%\allowdisplaybreaks[1]

%Tweaks
%% \setlength{\parindent}{0mm}		%Sangría de 1a. línea
%% \setlength{\hoffset}{2.6mm}		%
%% \setlength{\voffset}{-5.4mm}		%
%% \setlength{\topmargin}{0mm}		%
%% \setlength{\oddsidemargin}{5mm}	%
%% \setlength{\evensidemargin}{5mm}	%
%% \setlength{\marginparsep}{5mm}	%
%% \setlength{\headheight}{12.5mm}	%
%% \setlength{\headsep}{2.5mm}		%
%% \setlength{\footskip}{10mm}		%
%% \setlength{\textwidth}{14.1cm}		%
%% \setlength{\textheight}{232mm}	%
%% \setlength{\fboxrule}{.1pt}
%% \setlength{\parskip}{.5\baselineskip}

%Colores
\definecolor{negro}	{cmyk}{0,0,0,1}
\definecolor{marron}	{cmyk}{0,.5,1,.41}
\definecolor{rojo}	{cmyk}{0,1,1,0}
\definecolor{naranja}	{cmyk}{0,.35,1,0}
\definecolor{amarillo}	{cmyk}{0,0,1,0}
\definecolor{verde}	{cmyk}{1,0,1,0}
\definecolor{azul}	{cmyk}{1,1,0,0}
\definecolor{violeta}	{cmyk}{.45,1,0,0}
\definecolor{gris}	{cmyk}{0,0,0,.5}
\definecolor{blanco}	{cmyk}{0,0,0,0}
\definecolor{dorado}	{cmyk}{0,.16,1,0}
\definecolor{plateado}	{cmyk}{0,0,0,.25}

%% \title{\titulo}
%% \author{\autor}
%% \date{\fecha}

% si uso pdflatex, me setea las propiedades del pdf de salida
%% \ifpdf\pdfinfo{/Title    (\tituloPDF)
%%                /Author   (\autorPDF)
%%                /Subject  (\asuntoPDF)
%%                /Keywords (\clavesPDF)}\fi

% comandos.tex
% en este archivo defino todos los comandos/environment que quiera usar en mi documento.
%
% ===
% === Comandos
% ===
% 
% T: para escribir texto común cuando en modo math
%    uso: \T{texto que aparecerá en letra normal}
\newcommand{\T}{\textrm}
%
% aclaracion: dibuja un recuadrito aclaratorio, como <quote> en HTML.
%             uso: \aclaracion{Texto...}
\newcommand{\aclaracion}[1]{%
\smallpencil\-\begin{minipage}{0.9\textwidth}
%\vspace*{6pt}
{#1}\smallskip\end{minipage}}
%
% consigna: parecido a aclaración, pero con texto _slanted_
%           uso: \consigna{Consigna...}
\newcommand{\consigna}[1]{%
\leftpointright\ \parbox[t]{0.9\textwidth}{\textsl{#1}\vspace{8pt}}}
%
% pinterno: para representar el producto interno entre los dos argumentos
%           uso: \pinterno{X}{Y}
\newcommand{\pinterno}[2]{%
\left\langle #1 , #2 \right\rangle}
%
% === Estilos de texto
%
% resalt: resaltado con fondo verde
%         uso: \resalt{texto resaltado}
\newcommand{\resalt}{\colorbox{yellow}}
%
% sfbf: texto en negrita + slanted
%       uso:
\newcommand{\sfbf}[1]{\textsf{\bfseries #1}}
%
% small bold sans-serif
\newcommand{\sbs}[1]{\textsf{\small\bfseries #1}}
%
% eng: itálica (para palabras en inglés)
%      uso: \eng{some English text}
\newcommand{\eng}{\textit}
%
% mean: significado de una sigla - slanted
%       uso: (...) SNCF: \mean{Société Nationale des Chemins de Fer Francais} ...
\newcommand{\mean}{\textsl}
\newcommand{\desc}{\textsl}
%
% defin: pone en negrita el texto, útil para definiciones
%        uso: \defin{asshole}: vulgar slang for anus
\newcommand{\defin}{\textbf}
%
% R, N: cambia la tipografía en modo math, probar para ver cómo quedan
%       uso: \R{R} , \N{N}
\newcommand{\R}{\mathds}
\newcommand{\N}{\mathbf}
\newcommand{\C}{\mathcal}
\newcommand{\B}{\boldsymbol}
%
% dx: para escribir d2y/dx2, etc
\newcommand{\dx}[2]{\frac{d^{#2}\!#1}{d\!x^{#2}}}
%
% dp: para escribir derivadas parciales d2y/dx2, etc
\newcommand{\dpar}[3]{\frac{\partial^{#3}#1}{\partial{#2}^{#3}}}
%
% dvar: para escribir derivadas totales d2y/d(VAR)2, etc
\newcommand{\dvar}[3]{\frac{d^{#3}#1}{d{#2}^{#3}}}
%
% evalen: para escribir (loquesea)|_{evaluado_en}
\newcommand{\evalen}[2]{\left.{#2}\right|_{#1}}
%
% lil: para escribir texto pequeño. más cómodo que { \footnotesize texto pequeño... }
%      uso: \lil{texto pequeño... }
\newcommand{\lil}[1]{\footnotesize #1}  %Para texto pequeñooo
%
% mono: escribe el texto que le paso como parámetro con letra de ancho fijo
%       uso: \mono{texto monoespaciado}
\newcommand{\mono}[1]{{\texttt{#1}}}
%
% === Símbolos
%
\newcommand{\y}{\wedge}			%Y (Lógica)
\newcommand{\ve}{\vee}			%O (Lógica)
\newcommand{\ent}{\supset}		%Entonces (Lógica)
\newcommand{\dimp}{\leftrightarrow}	%Doble implicativo, equivalencia (Lógica)
\newcommand{\sii}{\leftrightarrow}	%Si y sólo si (Lógica)
\newcommand{\equi}{\equiv}		%Equivalencia (Lógica)
\newcommand{\portanto}{\vdash}		%Por lo tanto (Lógica)
\newcommand{\por}{\cdot}		%Producto punto
\newcommand{\RR}[1][1]{\mathds{R}}	%R de reales
\newcommand{\hfi}{\hat{\phi}}           %fi con gorrito arriba
\newcommand{\bfi}{\bar{\phi}}           %fi con raya arriba
\newcommand{\hpsi}{\hat{\psi}}          %Letra griega psi con gorrito arriba
\newcommand{\II}{\B{I}}                 %w negrita
\newcommand{\KK}{\B{K}}                 %w negrita
\newcommand{\QQ}{\B{Q}}                 %w negrita
\newcommand{\YY}{\B{Y}}                 %w negrita
\newcommand{\Bg}{\B{g}}                 %w negrita
\newcommand{\nn}{\B{n}}                 %w negrita
\newcommand{\uu}{\B{u}}                 %w negrita
\newcommand{\vv}{\B{v}}                 %w negrita
\newcommand{\ww}{\B{w}}                 %w negrita
\newcommand{\xx}{\B{x}}                 %x negrita
\newcommand{\yy}{\B{y}}                 %y negrita
\newcommand{\zz}{\B{z}}                 %z negrita
\newcommand{\BPhi}{\B{\Phi}}            %\Phi negrita
\newcommand{\Balpha}{\B{\alpha}}        %\alpha negrita
\newcommand{\Bbeta}{\B{\beta}}          %\beta negrita
\newcommand{\Btheta}{\B{\theta}}        %\theta negrita
\newcommand{\Bxi}{\B{\xi}}              %\xi negrita
%
%
% ===
% === Environments
% ===
% 
% enunciado: un environment que básicamente tiene el mismo efecto que el
%            comando consigna.
%            uso: \begin{enunciado} ... contenido ... \end{enunciado}
\newenvironment{enunciado}
{\leftpointright\ \begin{varwidth}[t]{0.9\textwidth}\textsl}
{\end{varwidth}\vspace{8pt}}
%
% pvi: para tipear la definición de un problema de valor inicial/funciones
%      definidas de a trozos/etc directamente en el texto (sin necesidad de
%      cambiar a un modo matemático.
%      uso: \begin{pvi} linea1 \\ linea 2 \\ ... \end{pvi}
\newenvironment{pvi}{\begin{equation}\begin{cases}}
{\end{cases}\end{equation}}
%
% pvi*: comp pvi, pero sin número de ecuación
\newenvironment{pvi*}{\begin{equation*}\begin{cases}}
{\end{cases}\end{equation*}}
%
% verbatimsmall: un verbatim con letra más chica. usualmente queda bastante
%                mejor que el verbatim pelado.
%                uso: \begin{verbatimsmall} ........ \end{verbatimsmall}
\newenvironment{verbatimsmall}{\small\begin{verbatim*}}
{\end{verbatim*}}
%
% nota: escribe una aclaracion dentro del texto
\newenvironment{nota}{$$\left[\;\begin{minipage}{0.95\textwidth}\slshape}
{\end{minipage}\;\right]$$}
%
%
% ===
% === Comandos ``históricos''
% ===
%
%% %\begin{pspicture}
%% \def\tierra(#1){%Para dibujar el símbolo de tierra en el entorno PSTricks
%% 	\rput(#1){
%% 		\psdot(0,0)
%% 		\psline(0,0)(0,-0.45)
%% 		\psline(-0.5,-0.45)(0.5,-0.45)
%% 		\psline(-0.35,-0.6)(0.35,-0.6)
%% 		\psline(-0.2,-0.75)(0.2,-0.75)
%% 	}%
%% }
%% %\end{pspicture}

\newcommand{\codigo}[2]{%Para generar un recuadro con código
	%\setlength{\hrulewidth}{0.1pt}
	\begin{flushleft}
	\underline{#1}
	\begin{tabular}{@{\quad}|l}
		\begin{minipage}{.85\textwidth}\smallskip{#2}
	\end{minipage}\end{tabular}\end{flushleft}%
}

\newcommand{\filecodigo}[1]{%Insertar código verbatim desde un archivo
\codigo{#1}{\verbatiminput{#1}}}%Requiere el paquete verbatim
\newcommand{\filecodigobis}[1]{{\verbatiminput{#1}}}%Requiere el paquete verbatim

%% \newcommand{\grafico}[3][1]{%Para generar un plot de un archivo con coords.
%% %\def\deequis=#1
%% \begin{minipage}{0.5\textwidth}\begin{center}
%% \begin{pspicture}(6,5)
%% 	\psgrid[subgriddiv=1,gridlabels=0pt,gridwidth=.1pt](1,3)(1,1)(6,5)
%% 	\psset{xunit=5cm,yunit=2cm}
%% 	\fileplot[linewidth=1pt,linecolor=blue,origin={0.2,1.5}]{#2}
%% 	\psset{xunit=1cm,yunit=1cm}
%% 	\psaxes[Dx=#1,dx=5,Oy=-1,Dy=1,dy=2]{-}(0.9,1)(6,5)
%% 	\rput(4,0.4){\textsl{#3}}
%% \end{pspicture}\end{center}\end{minipage}}

%% \newcommand{\eqncode}[2]{%
%% \begin{center}
%% \begin{tabular}{l@{\hspace{0.5cm}}r}
%% \begin{minipage}{.4\textwidth}
%% \begin{equation*}
%% #1
%% \end{equation*}
%% \end{minipage}
%% &
%% \fbox{\begin{minipage}{.4\textwidth}
%% %\setlength{\parskip}{4mm}
%% \filecodigobis{#2}
%% \end{minipage}}
%% \end{tabular}
%% \end{center}
%% }

%% \newcommand{\eqncodeb}[2]{%
%% \begin{center}\begin{tabular}{l@{\hspace{0.5cm}}r}
%% \begin{minipage}{.4\textwidth}#1\end{minipage} &
%% \fbox{\begin{minipage}{.4\textwidth}\filecodigobis{#2}\end{minipage}}
%% \end{tabular}\end{center}}

%% \newenvironment{matemcode}[1]{\newline
%% \begin{tabular}{l@{\hspace{0.5cm}}r}
%% \begin{minipage}{.4\textwidth}
%% \parbox[t]{.4\textwidth}{\begin{equation*}#1\end{equation*}}\end{minipage}
%% &\begin{Sbox}\begin{minipage}{.4\textwidth}}
%% {\end{minipage}\end{Sbox}\fbox{\TheSbox}\end{tabular}\newline}

%% \newenvironment{encuadrar}[1]{\begin{Sbox}\begin{varwidth}{#1\textwidth}}
%% {\end{varwidth}\end{Sbox}\fbox{\TheSbox}}

%% \newenvironment{parboxenv}{\begin{Sbox}}
%% {\end{Sbox}\parbox[t]{.9\textwidth}{\TheSbox}}

% multicolumn y multirow
\newcommand{\mcol}[3]{\multicolumn{#1}{#2}{#3}}
\newcommand{\mrow}[3]{\multirow{#1}{#2}{#3}}

%% Serif .......................................................................
%%
%% New Century Schoolbook
%% \usepackage[T1]{fontenc}
%% \usepackage{fouriernc}
%%
%%
%% TeX Gyre Schola (New Century extendida)
%% \usepackage[T1]{fontenc}
%% \usepackage{tgschola}
%%
%%
%% Utopia
%% \usepackage[T1]{fontenc}
%% \usepackage{fourier}
%%
%%
%% Utopia (con MathDesign)
%% \usepackage[T1]{fontenc}
%% \usepackage[adobe-utopia]{mathdesign}
%%
%%
%% Computer Concrete
%% \usepackage[T1]{fontenc}
%% \usepackage{concmath}
%%
%%
%% Charter BT
 \usepackage[T1]{fontenc}
 \usepackage[bitstream-charter]{mathdesign}
%%
%%
%% Nimbus Roman (clon de Times)
%% \usepackage[T1]{fontenc}
%% \usepackage{nimbus}
%%
%%
%% TeX Gyre Termes (version mejorada de Nimbus Roman)
%% \usepackage[T1]{fontenc}
%% \usepackage{tgtermes}
%%
%%
%% GFS Bodoni
%% \usepackage[T1]{fontenc}
%% \usepackage[default]{gfsbodoni}
%%
%%
%% Baskervald ADF
%% \usepackage[T1]{fontenc}
%% \usepackage{baskervald}
%%
%%
%% Efont Serif -- descargar de http://openlab.jp/efont/serif/
%% \usepackage[T1]{fontenc}
%% \usepackage{efont,mathesf}
%% \renewcommand*\oldstylenums[1]{{\fontfamily{esfod}\selectfont#1}}
%%
%%
%%
%%
%%
%% Sans-Serif ..................................................................
%%
%%
%% Optima (clon de, URW Classico)
%% \usepackage[T1]{fontenc}
%% \renewcommand*\sfdefault{uop}
%%
%%
%% Avantgarde (clon de, URW Gothic)
%% \usepackage[T1]{fontenc}
%% \usepackage{avant}
%%
%%
%% TeX Gyre Adventor (version mejorada de Avantgarde)
%% \usepackage[T1]{fontenc}
%% \usepackage{tgadventor}
%%
%%
%% Nimbus Sans (clon de Helvetica)
%% \usepackage[T1]{fontenc}
%% \usepackage{nimbus}
%%
%%
%% Helvetica (clon de, Nimbus Sans)
%% \usepackage[T1]{fontenc}
%% \usepackage[scaled]{helvet}
%%
%%
%% TeX Gyre Heros (version mejorada de Nimbus Sans)
%% \usepackage[T1]{fontenc}
%% \usepackage{tgheros}
%%
%%
%% Boilinum
%% \usepackage[T1]{fontenc}
%% \usepackage{libertine}
%%
%%
%% Computer Modern Bright
%% \usepackage[T1]{fontenc}
%% \usepackage{cmbright}
%%
%%
%% Latin Modern Sans
%% \usepackage[T1]{fontenc}
%% \usepackage{lmodern}
%%
%%
%% Epigrafica
%% \usepackage[OT1]{fontenc}
%% \usepackage{epigrafica}
%%
%%
%%
%% Si quiero el documento en sans en vez de Roman:
%% \renewcommand*\familydefault{\sfdefault}
%% ...............................................
%% 
%%
%%
%% Monospaced ..................................................................
%%
%%
%% Pandora Typewriter
%% \usepackage[T1]{fontenc}
%% \usepackage{pandora}
%%
%%
%% Letter Gothic
%% \usepackage[T1]{fontenc}
%% \usepackage{ulgothic}
%%
%%
%% Inconsolata
%% \usepackage[T1]{fontenc}
%% \usepackage{inconsolata}
%%

%
% ===
% === Inicio del documento
% === 
%
\begin{document}
% crear la página de título
\maketitle
%
\section*{Ejercicio 4}
%
Resuelvo el problema
\begin{align*}
  \begin{cases}
    \dx{\phi}{2}-\phi=0, & 0\leq x\leq 1,\\
    \phi(0)=0,\\
    \phi(1)=1
  \end{cases}
\end{align*}
mediante el método de elementos finitos, con una partición de M elementos en $x$.

Planteo mediante residuos ponderados
\begin{align*}
  \int_0^1 W_l \left( \dx{\hfi}{2}-\hfi \right)dx=0,\qquad l=1,2,\ldots,M+1,\\
%% \end{align*}
%% donde 
%% \begin{align*}
  \hfi = \sum_{m=1}^{M+1}\phi_m N_m.
\end{align*}
%
%\begin{nota}
  Para este problema particular, ignoro explícitamente el residuo en el contorno, ya
  que el valor de los nodos en los extremos es conocido, y las funciones de forma
  que utilizaré son de soporte compacto.
%\end{nota}
%

La forma débil de esta integral será entonces
\begin{align*}
  -\int_0^1 \left(\dx{W_l}{}\dx{\hfi}{}+W_l\hfi\right)dx + \left[W_l\dx{\hfi}{}\right]_0^1=0
\end{align*}
Otra vez, ignoro la evaluación en el contorno:
\begin{align*}
  -\int_0^1 \left(\dx{W_l}{}{}\sum\phi_m\dx{N_m}{}+W_l\sum\phi_mN_m\right)dx=0.
\end{align*}
%
Expresando esto como $\B{K\phi=f}$, tengo las componentes de $\B{K}$ y $\B{f}$
\begin{align*}
  \B{K}_{lm}&=\int_0^1 \left(\dx{W_l}{}{}\dx{N_m}{}+W_lN_m\right)dx, &
  \B{f}_{l}&=\left[
    \begin{array}{c}
      \left.-\dx{\hfi}{}\right|_{x=0}\\0\\\vdots\\0\\\left.\dx{\hfi}{}\right|_{x=1}
    \end{array}
    \right],&1\leq l,m\leq M+1.
\end{align*}
Uso Galerkin, entonces queda
\begin{align*}
  \B{K}_{lm}&=\int_0^1 \left(\dx{N_l}{}{}\dx{N_m}{}+N_lN_m\right)dx, &1\leq l,m\leq M+1.
\end{align*}
%
%
\subsection*{Assemble}
Pasamos ahora a ver el problema desde el punto de vista elemental. Para cada elemento
$e$, la matriz elemental correspondiente se define como
\begin{align*}
  \B{K}^e_{l'm'}&=\int_0^{h^e} \left(\dx{N^e_{l'}}{}{}\dx{N^e_{m'}}{}+N^e_{l'}N^e_{m'}\right)dx, &l',m'=1,2.
\end{align*}
La contribución de estas matrices elementales a la matriz global viene dada por
\begin{align*}
  \B{K}_{lm} = \bigwedge_{l'm'}\B{K}^e_{l'm'}
\end{align*}
donde $\bigwedge$ es el operador de ensamblaje.
%
\subsubsection*{Contribución elemental}
Para el cálculo de la contribución elemental, defino las funciones de forma
$N^e_1$ y $N^e_2$ en el dominio elemental normalizado
\begin{align*}
  \hat\Omega^e=\left\{\xi|\xi\in[-1,1]\right\}
\end{align*}
con
\begin{align*}
  \xi=\frac{2(x-x_c^e)}{h^e}
\end{align*}
donde $x_c^e$ es la coordenada central del elemento, $h^e$ la longitud del elemento, y entonces
el elemento se define en el rango $-1\leq\xi\leq 1$.

Ahora defino las $N^e_{l}$ según
\begin{align*}
  N_1^e &= \frac{1-\xi}{2},& N_2^e &= \frac{\xi+1}{2}.
\end{align*}
Teniendo en cuenta que
\begin{align*}
  \dx{N_{l'}^e}{}=\dvar{N_{l'}^e}{\xi}{}\dx{\xi}{}=\frac{2}{h^e}\dvar{N_{l'}^e}{\xi}{},\qquad dx=\frac{h^e}{2}d\xi
\end{align*}
calculo la matriz
\begin{align*}
  \B{K}^e_{l'm'}&=\int_0^{h^e} \left(\dx{N^e_{l'}}{}{}\dx{N^e_{m'}}{}+N^e_{l'}N^e_{m'}\right)dx\\
  &=\int_{-1}^{1} \left(\frac{2}{h^e}\dvar{N^e_{l'}}{\xi}{}\frac{2}{h^e}\dvar{N^e_{m'}}{\xi}{}+N^e_{l'}N^e_{m'}\right)\frac{h^e}{2}d\xi\\
  &=\int_{-1}^{1} \left(\frac{2}{h^e}\dvar{N^e_{l'}}{\xi}{}\dvar{N^e_{m'}}{\xi}{}+\frac{h^e}{2}N^e_{l'}N^e_{m'}\right)d\xi
\end{align*}
entonces
\begin{align*}
\B{K}^e_{11}&=\int_{-1}^{1} \left(\frac{2}{h^e}\frac{-1}{2}\frac{-1}{2}+\frac{h^e}{2}\frac{1-\xi}{2}\frac{1-\xi}{2}\right)d\xi=\int_{-1}^{1} \left(\frac{1}{2h^e}+\frac{h^e(1-2\xi+\xi^2)}{8}\right)d\xi\\
&=\left(\frac{1}{2h^e}+\frac{h^e}{8}\right)\int_{-1}^{1}d\xi
- \frac{h^e}{4}\int_{-1}^{1}\xi d\xi + \frac{h^e}{8} \int_{-1}^{1}\xi^2 d\xi\\
&=\left(\frac{1}{2h^e}+\frac{h^e}{8}\right) (2)
- \frac{h^e}{4}(0) + \frac{h^e}{8}\left( \frac{2}{3}\right)=\frac{1}{h^e}+\frac{h^e}{4}
+ \frac{h^e}{12}\\%
&=\frac{1}{h^e}
+ \frac{h^e}{3}\\
\B{K}^e_{22}&=\int_{-1}^{1} \left(\frac{2}{h^e}\frac{1}{2}\frac{1}{2}+\frac{h^e}{2}\frac{\xi+1}{2}\frac{\xi+1}{2}\right)d\xi=\int_{-1}^{1} \left(\frac{1}{2h^e}+\frac{h^e(1+2\xi+\xi^2)}{8}\right)d\xi\\
&\;\;\vdots\\
&=\frac{1}{h^e}+ \frac{h^e}{3}\\
\B{K}^e_{12}=\B{K}^e_{21}&=\int_{-1}^{1} \left(\frac{2}{h^e}\frac{1}{2}\frac{-1}{2}+\frac{h^e}{2}\frac{1-\xi}{2}\frac{\xi+1}{2}\right)d\xi
=\int_{-1}^{1} \left(\frac{-1}{2h^e}+\frac{h^e(1-\xi^2)}{8}\right)d\xi\\
&=\left(\frac{-1}{2h^e}+\frac{h^e}{8}\right)\int_{-1}^{1}d\xi-\frac{h^e}{8}\int_{-1}^{1}\xi^2d\xi\\
&=\left(\frac{-1}{2h^e}+\frac{h^e}{8}\right)(2)-\frac{h^e}{8}\left(\frac{2}{3}\right)=-\frac{1}{h^e}+\frac{h^e}{4}-\frac{h^e}{12}\\
&=-\frac{1}{h^e}+\frac{h^e}{6}.
\end{align*}
%
%
\subsubsection*{Gather}
La matriz global $\B{K}$ se forma sumando en la posición $(l,l), 1\leq l\leq M$ la
contribución de la matriz elemental $\B{K}^l$, esto es, para cada $l=1\ldots M$:
\begin{align*}
  \B{K}_{l,l}&=\B{K}^l_{1,1},\\
\B{K}_{l,l+1}&=\B{K}^l_{1,2},\\
\B{K}_{l+1,l}&=\B{K}^l_{2,1},\\
\B{K}_{l+1,l+1}&=\B{K}^l_{2,2}.
\end{align*}
\subsection*{Solve}
La solución al sistema se obtiene resolviendo el sistema de ecuaciones armado
en las secciones anteriores. Por ejemplo, para un M=4, tendremos
\begin{align*}
  \left[
    \begin{array}{ccccc}
      \frac{1}{h^e}+\frac{h^e}{3} & -\frac{1}{h^e}+\frac{h^e}{6} & 0 & 0 & 0 \\
       -\frac{1}{h^e}+\frac{h^e}{6} & 2\left(\frac{1}{h^e}+\frac{h^e}{3}\right) &
       -\frac{1}{h^e}+\frac{h^e}{6} & 0 & 0 \\
       0 & -\frac{1}{h^e}+\frac{h^e}{6} & 2\left(\frac{1}{h^e}+\frac{h^e}{3}\right) &
       -\frac{1}{h^e}+\frac{h^e}{6} & 0\\
       0 & 0 & -\frac{1}{h^e}+\frac{h^e}{6} & 2\left(\frac{1}{h^e}+\frac{h^e}{3}\right) &
       -\frac{1}{h^e}+\frac{h^e}{6}\\
       0 & 0 & 0 & -\frac{1}{h^e}+\frac{h^e}{6} & \frac{1}{h^e}+\frac{h^e}{3}
    \end{array}
\right]
  \left[
    \begin{array}{c}
      \phi_1 \\ \phi_2 \\ \phi_3 \\ \phi_4\\ \phi_5
    \end{array}
\right]
=  \left[
    \begin{array}{c}
      \left.-\dx{\hfi}{}\right|_{x=0}\\0 \\ 0\\ 0\\\left.\dx{\hfi}{}\right|_{x=1}
    \end{array}
\right]
\end{align*}
En este problema en particular, los valores de $\phi_1$ y $\phi_{5}$ son conocidos,
así que eliminando las primeras y últimas filas y columnas del sistema y estableciendo
a mano el valor de los respectivos nodos, queda resolver el sistema reducido
\begin{align*}
  \left[
    \begin{array}{ccc}
        2\left(\frac{1}{h^e}+\frac{h^e}{3}\right) &
       -\frac{1}{h^e}+\frac{h^e}{6} & 0  \\
        -\frac{1}{h^e}+\frac{h^e}{6} & 2\left(\frac{1}{h^e}+\frac{h^e}{3}\right) &
       -\frac{1}{h^e}+\frac{h^e}{6} \\
        0 & -\frac{1}{h^e}+\frac{h^e}{6} & 2\left(\frac{1}{h^e}+\frac{h^e}{3}\right) 
    \end{array}
\right]
  \left[
    \begin{array}{c}
       \phi_2 \\ \phi_3 \\ \phi_4
    \end{array}
\right]
=  \left[
    \begin{array}{c}
      \phi_1\left(\frac{1}{h^e}+\frac{h^e}{6}\right)\\ 0\\\phi_5\left(\frac{1}{h^e}+\frac{h^e}{6}\right)
    \end{array}
\right]
\end{align*}
\newpage
\section*{Ejercicio 5}
\subsection*{a.}
\consigna{%
Analizar la barra de sección constante con una fuerza distribuida
$b(x)$ a lo largo de su longitud la cual tiene el valor $L$ y
además se encuentra sometida a una fuerza puntual $P$. La sección
transversal es constante y de valor $A$. Calcular los desplazamientos,
tensiones y deformaciones.
}

El Principio de los Trabajos Virtuales establece que
\begin{align*}
  \iiint_V \delta\epsilon\,\sigma\,d\!V = \int_0^L \delta u\,b\,d\!x + \sum_i \delta u_i\, X_i
\end{align*}
%
dode $\delta u$ y $\delta \epsilon$ son el desplazamiento y deformación virtual genéricos
de un punto de la barra, y $\delta u_i$ es el desplazamiento virtual del punto de actuación
de la fuerza $X_i$.

Integrando sobre la sección transversal, y teniendo en cuenta que $\sigma=E\epsilon=E\dx{u}{}$ con $E$ el módulo de Young, tengo
\begin{align*}
  \int_0^L \delta\epsilon\,A E\dx{u} \,d\!x   &= \int_0^L \delta u\,b\,d\!x + \sum_i \delta u_i\, X_i .
\end{align*}
Ésta es la ecuación de equilibrio que debo resolver para $u$.
%
\subsubsection*{Contribución elemental}
Como en el ejecicio anterior, considero dentro del elemento el dominio normalizado
\begin{align*}
  \hat\Omega^e=\left\{\xi|\xi\in[-1,1]\right\}, && \T{con } && \xi=\frac{2(x-x_c^e)}{h^e}.
\end{align*}
Otra vez, $x_c^e$ es la coordenada central del elemento, $h^e$ la longitud del elemento, y entonces
el dominio elemental será el rango $-1\leq\xi\leq 1$.
Las funciones de forma $N^e_{l}$ serán entonces
\begin{align*}
  N_1^e &= \frac{1-\xi}{2},& N_2^e &= \frac{\xi+1}{2}.
\end{align*}
La solución $u$ la aproximamos según
\begin{align*}
  u = N_1^e u_1 + N_2^e u_2
\end{align*}
con $u_1$, $u_2$ el valor de u en los nodos 1, 2. El strain viene dado por
\begin{align*}
  \epsilon &= \dx{u}{} = \dvar{u}{\xi}{} \dvar{\xi}{x}{} = \frac{2}{h^e} \dvar{u}{\xi}{}
                      = \frac{2}{h^e} \left(\dvar{N_1^e}{\xi}{} u_1 + \dvar{N_2^e}{\xi}{} u_2\right)
                      = \frac{2}{h^e} \left(\frac{-1}{2} u_1 + \frac{1}{2} u_2\right) \\
  \epsilon &= \frac{1}{h^e}(u_2-u_1).
\end{align*}

Las fuerzas entre elementos se transmiten únicamente a través de los nodos [Oñate,p2.6].
Dichas fuerzas, que llamaremos ``de equilibrio'', pueden calcularse para cada elemento
haciendo uso del Principio de los Trabajos Virtuales según
\begin{align*}
  \int_{-1}^{1} \delta\epsilon\,A E\epsilon \,d\!\xi   &= \int_{-1}^{1} \delta u\,b\,d\!\xi + \delta u_1\, X_1 + \delta u_2\, X_2 ,
\end{align*}
donde $\delta u_1$, $\delta u_2$, $X_1$, $X_2$ son los desplazamientos virtuales y las fuerzas d e equilibrio en los nodos 1 y 2. Interpolamos el desplazamiento virtual dentro del elemento
\begin{align*}
  \delta u = N_1^e \delta u_1 + N_2^e \delta u_2
\end{align*}
y el strain virtual será
\begin{align*}
  \delta \epsilon = \dx{}{}(\delta u) = \frac{2}{h^e}\left( \dvar{N_1^e}{\xi}{} \delta u_1 + \dvar{N_2^e}{\xi}{} \delta u_2 \right).
\end{align*}
Volviendo al PTV para el elemento, tengo
\begin{align*}
 \begin{split}
    \left(\frac{2}{h^e}\right)^2 \int_{-1}^{1} \left(\dvar{N_1^e}{\xi}{}\delta u_1+\dvar{N_2^e}{\xi}{}\delta u_2\right)
         \,A E \left(\dvar{N_1^e}{\xi}{}u_1+\dvar{N_2^e}{\xi}{}u_2\right) \,d\!\xi  \qquad \qquad\\ \qquad \qquad
    = \int_{-1}^{1}(N_1^e \delta u_1 + N_2^e \delta u_2) \,b\,d\!\xi + \delta u_1\, X_1 + \delta u_2\, X_2
  \end{split}
\end{align*}
Reagrupando los términos,
\begin{align*}
 \begin{split}
   \delta u_1 \left[ \left(\frac{2}{h^e}\right)^2 \int_{-1}^{1} \dvar{N_1^e}{\xi}{}
         \,A E \left(\dvar{N_1^e}{\xi}{}u_1+\dvar{N_2^e}{\xi}{}u_2\right) \,d\!\xi
         - \int_{-1}^{1}N_1^e \,b\,d\!\xi - \, X_1 \right] \qquad \qquad\\ \qquad \qquad
    + \delta u_2 \left[ \left(\frac{2}{h^e}\right)^2 \int_{-1}^{1} \dvar{N_2^e}{\xi}{}
         \,A E \left(\dvar{N_1^e}{\xi}{}u_1+\dvar{N_2^e}{\xi}{}u_2\right) \,d\!\xi
         - \int_{-1}^{1}N_2^e \,b\,d\!\xi -  X_2 \right] =0
 \end{split}
\end{align*}
Como los desplazamientos virtuales son arbitrarios, las expresiones entre corchetes deberán
ser nulas. Esto se traduce en el sistema de ecuaciones
\begin{align*}
  \left(\frac{2}{h^e}\right)^2 \int_{-1}^{1} \left( \dvar{N_1^e}{\xi}{} A E \dvar{N_1^e}{\xi}{}u_1
      + \dvar{N_1^e}{\xi}{} A E \dvar{N_2^e}{\xi}{}u_2\right) \,d\!\xi
      - \int_{-1}^{1}N_1^e \,b\,d\!\xi - X_1 &=0 \\ 
  \left(\frac{2}{h^e}\right)^2 \int_{-1}^{1} \left( \dvar{N_2^e}{\xi}{} A E \dvar{N_1^e}{\xi}{}u_1
      + \dvar{N_2^e}{\xi}{} A E \dvar{N_2^e}{\xi}{}u_2\right) \,d\!\xi
      - \int_{-1}^{1}N_2^e \,b\,d\!\xi - X_2  &=0,
\end{align*}
en forma matricial,
\begin{align*}
  \left(\frac{2}{h^e}\right)^2
  \left[
    \begin{array}{cc}
      \int_{-1}^{1} \dvar{N_1^e}{\xi}{} A E \dvar{N_1^e}{\xi}{} \,d\!\xi &
      \int_{-1}^{1} \dvar{N_1^e}{\xi}{} A E \dvar{N_2^e}{\xi}{} \,d\!\xi \\
      \int_{-1}^{1} \dvar{N_2^e}{\xi}{} A E \dvar{N_1^e}{\xi}{} \,d\!\xi &
      \int_{-1}^{1} \dvar{N_2^e}{\xi}{} A E \dvar{N_2^e}{\xi}{} \,d\!\xi
   \end{array}
    \right]
\left[
  \begin{array}{c}
    u_1 \\ u_2
  \end{array}
\right] = 
\left[
  \begin{array}{c}
    \int_{-1}^{1}N_1^e \,b\,d\!\xi + X_1 \\
    \int_{-1}^{1}N_2^e \,b\,d\!\xi + X_2
  \end{array}
\right].
\end{align*}
Considerando $A$, $E$, $b$ como constantes e integrando
\begin{align*}
  AE \left(\frac{2}{h^e}\right)^2
  \left[
    \begin{array}{cc}
      \int_{-1}^{1} \frac{-1}{2}\frac{-1}{2} \,d\!\xi &
      \int_{-1}^{1} \frac{-1}{2}\frac{1}{2}  \,d\!\xi \\
      \int_{-1}^{1} \frac{1}{2} \frac{-1}{2} \,d\!\xi &
      \int_{-1}^{1} \frac{1}{2} \frac{1}{2}  \,d\!\xi
   \end{array}
    \right]
\left[
  \begin{array}{c}
    u_1 \\ u_2
  \end{array}
\right] =
\left[
  \begin{array}{c}
   b \int_{-1}^{1}N_1^e \,d\!\xi + X_1 \\
   b \int_{-1}^{1}N_2^e \,d\!\xi + X_2
  \end{array}
\right]\\
  AE \left(\frac{2}{h^e}\right)^2
  \left[
    \begin{array}{cc}
      \frac{1}{2}  &
      \frac{-1}{2} \\
      \frac{-1}{2} &
      \frac{1}{2}
   \end{array}
    \right]
\left[
  \begin{array}{c}
    u_1 \\ u_2
  \end{array}
\right] =
\left[
  \begin{array}{c}
   b + X_1 \\
   b + X_2
  \end{array}
\right]
\end{align*}





\end{document}
