\documentclass[12pt,bibliography=openstyle,DIV=12,parskip=half-]{scrartcl}
% esto me setea la variable pdf dependiendo del valor de \pdfoutput, que es >0
% sólo cuando estoy usando pdflatex para compilar el documento
%% \newif\ifpdf
%% \ifnum\pdfoutput<0
%% \pdffalse\fi
%% \ifnum\pdfoutput=0
%% \pdffalse\fi
%% \ifnum\pdfoutput>0
%% \pdftrue\fi


%
% ===
% === Trick para detectar si el documento está siendo compilado con pdflatex
% ===
%
% Esto me setea la variable pdf dependiendo del valor de \pdfoutput, que es >0
% sólo cuando estoy usando pdflatex para compilar el documento. Con esto puedo
% hacer  \ifpdf {...} \fi, que se ejecuta colo cuando compilo con pdflatex.
%% \newif\ifpdf
%% \ifnum\pdfoutput<0
%% \pdffalse\fi
%% \ifnum\pdfoutput=0
%% \pdffalse\fi
%% \ifnum\pdfoutput>0
%% \pdftrue\fi
%
% ===
% === I18n / L10n
% ===
%
% babel me da separación de sílabas para palabras en el idioma que le paso como
%       argumento opcional.
\usepackage[spanish,es-tabla,english]{babel}
%
% inputenc define la codificación de caracteres del código fuente, acá utf8.
\usepackage[utf8]{inputenc}
%
% ===
% === Gráficos
% ===
% 
% pst-pdf me permite usar PSTricks con pdflatex. Necesito cargarlo sólo si está
%         definida la variable pdf, por eso está entre \ifpdf ... \fi
%\ifpdf\usepackage{pst-pdf}\fi
%
% color me permite usar colores en el documento.
\usepackage{color}
%
% graphicx me da el comando \includegraphics para insertar imágenes (?)
\usepackage{graphicx}
%
% pstricks es un conjunto de macros basadas en PostScript para TeX, en
%          castellano: me da un entorno pstricks y comandos que uso dentro de
%          éste, que me sirven para dibujar figuras/diagramas/etc de manera
%          relativamente simple.
%\usepackage{pstricks}
%
% pst-circ me da macros para pstricks que me dibujan elementos de circuitos
%\usepackage{pst-circ}
%
% pst-plot me provee de funciones de ploteo para pstricks
%\usepackage{pst-plot}
%
% pst-2dplot me sirve para plotear en pstricks, entorno pstaxes
%\usepackage{pst-2dplot}
%
% ===
% === Verbatims
% ===
%
% verbatim es una reimplementación de los entornos verbatim[*]
%          provee el comando \verbatiminput{archivo} y el entorno comment, que
%          hace que LaTeX ignore directamente todo lo que está adentro
%\usepackage{verbatim}
%
% moreverb implementa el entorno verbatimtab indentando los tabs que encuentre,
%          y también el entorno listing, que pone números de línea al verbatim.
%          Para cambiar el ancho de la tabulacion, uso
%          \renewcommand\verbatimtabsize{<ancho del tab>\relax}
%          También define el entorno boxedverbatim.
%\usepackage{moreverb}
%
% listings me da el entorno lstlisting con resaltado de sintaxis.
%          Para setear el lenguaje del código, hago \lstset{language=<lang>}
%\usepackage{listings}
%
% url es un verbatim para escribir URL's que permite linebreaks dentro de ésta.
%     para usarlo, \url{<URL>}
\usepackage{url}
%
% ===
% === Más packages
% ===
%
%% \usepackage{mdwlist}		%Para listas mas compactas
%% \usepackage{textcomp}		%Para algunos símbolos
%% \usepackage{colortbl}		%Para celdas de colores en tablas
%% \usepackage{fancyhdr}		%Para encabezados/pie
\usepackage{bbold}		%Fuente bb para modo math: \mathbb{R} = reales
\usepackage{dsfont}		%Fuente ds para modo math: \mathds{R} = reales
\usepackage{multirow}		%Para "combinar" celdas en tablas
\usepackage{float}		%Para mejorar cuadros, figuras, etc
%% \usepackage{fancybox}		%Para recuardos de texto con bordes "fancy"
%% \usepackage{dingbat}		%Para dingbats
%\usepackage{marginal}		%Para  notas al margen que no puedo hacer andar
\usepackage{amsmath}		%Para enornos matemáticos mas flexibles
%\usepackage{varwidth}		%varwidth es un minipage que se ajusta al ancho mínimo


\usepackage[backend=biber,sorting=none,style=ieee,eprint=false,url=false]{biblatex} %% style=ieee
%% requiere texlive-bibtex-extra en debian


\usepackage{enumitem}
\setlist{noitemsep}
%% \setlist[description]{noitemsep}
%% \setlist[enumerate]{noitemsep}
%% \setlist[itemize]{noitemsep}

\usepackage{tikz}
\usepackage{pgfkeys}
\usepackage{pgfgantt}

% typearea: uso con koma-script para ajustar márgenes de página.
% vars globales a setear en la clase koma-script: DIV=12, BCOR=margen de ``binding'' para double side
\usepackage{typearea}

% para poder usar footnotes p.ej, adentro de un tabular
\usepackage{footnote}
\makesavenoteenv{tabular}

% para tabulars mas lindos/legibles
\usepackage{booktabs}

%\usepackage{glossaries}

\usepackage[spanish]{algorithm2e}

% para highlight (comando \hl{})
\usepackage{soulutf8}

% para teoremans etc
\usepackage{amsthm}

% para tunear citations
%\usepackage[square,comma,numbers,sort&compress]{natbib}


% config.tex: configuraciones del documento
%\selectlanguage{spanish}		%Elijo idioma español

%Permitir que los entornos equation, align, etc permitan saltos de página
%\allowdisplaybreaks[1]

%Tweaks
%% \setlength{\parindent}{0mm}		%Sangría de 1a. línea
%% \setlength{\hoffset}{2.6mm}		%
%% \setlength{\voffset}{-5.4mm}		%
%% \setlength{\topmargin}{0mm}		%
%% \setlength{\oddsidemargin}{5mm}	%
%% \setlength{\evensidemargin}{5mm}	%
%% \setlength{\marginparsep}{5mm}	%
%% \setlength{\headheight}{12.5mm}	%
%% \setlength{\headsep}{2.5mm}		%
%% \setlength{\footskip}{10mm}		%
%% \setlength{\textwidth}{14.1cm}		%
%% \setlength{\textheight}{232mm}	%
%% \setlength{\fboxrule}{.1pt}
%% \setlength{\parskip}{.5\baselineskip}

%Colores
\definecolor{negro}	{cmyk}{0,0,0,1}
\definecolor{marron}	{cmyk}{0,.5,1,.41}
\definecolor{rojo}	{cmyk}{0,1,1,0}
\definecolor{naranja}	{cmyk}{0,.35,1,0}
\definecolor{amarillo}	{cmyk}{0,0,1,0}
\definecolor{verde}	{cmyk}{1,0,1,0}
\definecolor{azul}	{cmyk}{1,1,0,0}
\definecolor{violeta}	{cmyk}{.45,1,0,0}
\definecolor{gris}	{cmyk}{0,0,0,.5}
\definecolor{blanco}	{cmyk}{0,0,0,0}
\definecolor{dorado}	{cmyk}{0,.16,1,0}
\definecolor{plateado}	{cmyk}{0,0,0,.25}

%% \title{\titulo}
%% \author{\autor}
%% \date{\fecha}

% si uso pdflatex, me setea las propiedades del pdf de salida
%% \ifpdf\pdfinfo{/Title    (\tituloPDF)
%%                /Author   (\autorPDF)
%%                /Subject  (\asuntoPDF)
%%                /Keywords (\clavesPDF)}\fi

% comandos.tex
% en este archivo defino todos los comandos/environment que quiera usar en mi documento.
%
% ===
% === Comandos
% ===
% 
% T: para escribir texto común cuando en modo math
%    uso: \T{texto que aparecerá en letra normal}
\newcommand{\T}{\textrm}
%
% aclaracion: dibuja un recuadrito aclaratorio, como <quote> en HTML.
%             uso: \aclaracion{Texto...}
\newcommand{\aclaracion}[1]{%
\smallpencil\-\begin{minipage}{0.9\textwidth}
%\vspace*{6pt}
{#1}\smallskip\end{minipage}}
%
% consigna: parecido a aclaración, pero con texto _slanted_
%           uso: \consigna{Consigna...}
\newcommand{\consigna}[1]{%
\leftpointright\ \parbox[t]{0.9\textwidth}{\textsl{#1}\vspace{8pt}}}
%
% pinterno: para representar el producto interno entre los dos argumentos
%           uso: \pinterno{X}{Y}
\newcommand{\pinterno}[2]{%
\left\langle #1 , #2 \right\rangle}
%
% === Estilos de texto
%
% resalt: resaltado con fondo verde
%         uso: \resalt{texto resaltado}
\newcommand{\resalt}{\colorbox{yellow}}
%
% sfbf: texto en negrita + slanted
%       uso:
\newcommand{\sfbf}[1]{\textsf{\bfseries #1}}
%
% small bold sans-serif
\newcommand{\sbs}[1]{\textsf{\small\bfseries #1}}
%
% eng: itálica (para palabras en inglés)
%      uso: \eng{some English text}
\newcommand{\eng}{\textit}
%
% mean: significado de una sigla - slanted
%       uso: (...) SNCF: \mean{Société Nationale des Chemins de Fer Francais} ...
\newcommand{\mean}{\textsl}
\newcommand{\desc}{\textsl}
%
% defin: pone en negrita el texto, útil para definiciones
%        uso: \defin{asshole}: vulgar slang for anus
\newcommand{\defin}{\textbf}
%
% R, N: cambia la tipografía en modo math, probar para ver cómo quedan
%       uso: \R{R} , \N{N}
\newcommand{\R}{\mathds}
\newcommand{\N}{\mathbf}
\newcommand{\C}{\mathcal}
\newcommand{\B}{\boldsymbol}
%
% dx: para escribir d2y/dx2, etc
\newcommand{\dx}[2]{\frac{d^{#2}\!#1}{d\!x^{#2}}}
%
% dp: para escribir derivadas parciales d2y/dx2, etc
\newcommand{\dpar}[3]{\frac{\partial^{#3}#1}{\partial{#2}^{#3}}}
%
% dvar: para escribir derivadas totales d2y/d(VAR)2, etc
\newcommand{\dvar}[3]{\frac{d^{#3}#1}{d{#2}^{#3}}}
%
% evalen: para escribir (loquesea)|_{evaluado_en}
\newcommand{\evalen}[2]{\left.{#2}\right|_{#1}}
%
% lil: para escribir texto pequeño. más cómodo que { \footnotesize texto pequeño... }
%      uso: \lil{texto pequeño... }
\newcommand{\lil}[1]{\footnotesize #1}  %Para texto pequeñooo
%
% mono: escribe el texto que le paso como parámetro con letra de ancho fijo
%       uso: \mono{texto monoespaciado}
\newcommand{\mono}[1]{{\texttt{#1}}}
%
% === Símbolos
%
\newcommand{\y}{\wedge}			%Y (Lógica)
\newcommand{\ve}{\vee}			%O (Lógica)
\newcommand{\ent}{\supset}		%Entonces (Lógica)
\newcommand{\dimp}{\leftrightarrow}	%Doble implicativo, equivalencia (Lógica)
\newcommand{\sii}{\leftrightarrow}	%Si y sólo si (Lógica)
\newcommand{\equi}{\equiv}		%Equivalencia (Lógica)
\newcommand{\portanto}{\vdash}		%Por lo tanto (Lógica)
\newcommand{\por}{\cdot}		%Producto punto
\newcommand{\RR}[1][1]{\mathds{R}}	%R de reales
\newcommand{\hfi}{\hat{\phi}}           %fi con gorrito arriba
\newcommand{\bfi}{\bar{\phi}}           %fi con raya arriba
\newcommand{\hpsi}{\hat{\psi}}          %Letra griega psi con gorrito arriba
\newcommand{\II}{\B{I}}                 %w negrita
\newcommand{\KK}{\B{K}}                 %w negrita
\newcommand{\QQ}{\B{Q}}                 %w negrita
\newcommand{\YY}{\B{Y}}                 %w negrita
\newcommand{\Bg}{\B{g}}                 %w negrita
\newcommand{\nn}{\B{n}}                 %w negrita
\newcommand{\uu}{\B{u}}                 %w negrita
\newcommand{\vv}{\B{v}}                 %w negrita
\newcommand{\ww}{\B{w}}                 %w negrita
\newcommand{\xx}{\B{x}}                 %x negrita
\newcommand{\yy}{\B{y}}                 %y negrita
\newcommand{\zz}{\B{z}}                 %z negrita
\newcommand{\BPhi}{\B{\Phi}}            %\Phi negrita
\newcommand{\Balpha}{\B{\alpha}}        %\alpha negrita
\newcommand{\Bbeta}{\B{\beta}}          %\beta negrita
\newcommand{\Btheta}{\B{\theta}}        %\theta negrita
\newcommand{\Bxi}{\B{\xi}}              %\xi negrita
%
%
% ===
% === Environments
% ===
% 
% enunciado: un environment que básicamente tiene el mismo efecto que el
%            comando consigna.
%            uso: \begin{enunciado} ... contenido ... \end{enunciado}
\newenvironment{enunciado}
{\leftpointright\ \begin{varwidth}[t]{0.9\textwidth}\textsl}
{\end{varwidth}\vspace{8pt}}
%
% pvi: para tipear la definición de un problema de valor inicial/funciones
%      definidas de a trozos/etc directamente en el texto (sin necesidad de
%      cambiar a un modo matemático.
%      uso: \begin{pvi} linea1 \\ linea 2 \\ ... \end{pvi}
\newenvironment{pvi}{\begin{equation}\begin{cases}}
{\end{cases}\end{equation}}
%
% pvi*: comp pvi, pero sin número de ecuación
\newenvironment{pvi*}{\begin{equation*}\begin{cases}}
{\end{cases}\end{equation*}}
%
% verbatimsmall: un verbatim con letra más chica. usualmente queda bastante
%                mejor que el verbatim pelado.
%                uso: \begin{verbatimsmall} ........ \end{verbatimsmall}
\newenvironment{verbatimsmall}{\small\begin{verbatim*}}
{\end{verbatim*}}
%
% nota: escribe una aclaracion dentro del texto
\newenvironment{nota}{$$\left[\;\begin{minipage}{0.95\textwidth}\slshape}
{\end{minipage}\;\right]$$}
%
%
% ===
% === Comandos ``históricos''
% ===
%
%% %\begin{pspicture}
%% \def\tierra(#1){%Para dibujar el símbolo de tierra en el entorno PSTricks
%% 	\rput(#1){
%% 		\psdot(0,0)
%% 		\psline(0,0)(0,-0.45)
%% 		\psline(-0.5,-0.45)(0.5,-0.45)
%% 		\psline(-0.35,-0.6)(0.35,-0.6)
%% 		\psline(-0.2,-0.75)(0.2,-0.75)
%% 	}%
%% }
%% %\end{pspicture}

\newcommand{\codigo}[2]{%Para generar un recuadro con código
	%\setlength{\hrulewidth}{0.1pt}
	\begin{flushleft}
	\underline{#1}
	\begin{tabular}{@{\quad}|l}
		\begin{minipage}{.85\textwidth}\smallskip{#2}
	\end{minipage}\end{tabular}\end{flushleft}%
}

\newcommand{\filecodigo}[1]{%Insertar código verbatim desde un archivo
\codigo{#1}{\verbatiminput{#1}}}%Requiere el paquete verbatim
\newcommand{\filecodigobis}[1]{{\verbatiminput{#1}}}%Requiere el paquete verbatim

%% \newcommand{\grafico}[3][1]{%Para generar un plot de un archivo con coords.
%% %\def\deequis=#1
%% \begin{minipage}{0.5\textwidth}\begin{center}
%% \begin{pspicture}(6,5)
%% 	\psgrid[subgriddiv=1,gridlabels=0pt,gridwidth=.1pt](1,3)(1,1)(6,5)
%% 	\psset{xunit=5cm,yunit=2cm}
%% 	\fileplot[linewidth=1pt,linecolor=blue,origin={0.2,1.5}]{#2}
%% 	\psset{xunit=1cm,yunit=1cm}
%% 	\psaxes[Dx=#1,dx=5,Oy=-1,Dy=1,dy=2]{-}(0.9,1)(6,5)
%% 	\rput(4,0.4){\textsl{#3}}
%% \end{pspicture}\end{center}\end{minipage}}

%% \newcommand{\eqncode}[2]{%
%% \begin{center}
%% \begin{tabular}{l@{\hspace{0.5cm}}r}
%% \begin{minipage}{.4\textwidth}
%% \begin{equation*}
%% #1
%% \end{equation*}
%% \end{minipage}
%% &
%% \fbox{\begin{minipage}{.4\textwidth}
%% %\setlength{\parskip}{4mm}
%% \filecodigobis{#2}
%% \end{minipage}}
%% \end{tabular}
%% \end{center}
%% }

%% \newcommand{\eqncodeb}[2]{%
%% \begin{center}\begin{tabular}{l@{\hspace{0.5cm}}r}
%% \begin{minipage}{.4\textwidth}#1\end{minipage} &
%% \fbox{\begin{minipage}{.4\textwidth}\filecodigobis{#2}\end{minipage}}
%% \end{tabular}\end{center}}

%% \newenvironment{matemcode}[1]{\newline
%% \begin{tabular}{l@{\hspace{0.5cm}}r}
%% \begin{minipage}{.4\textwidth}
%% \parbox[t]{.4\textwidth}{\begin{equation*}#1\end{equation*}}\end{minipage}
%% &\begin{Sbox}\begin{minipage}{.4\textwidth}}
%% {\end{minipage}\end{Sbox}\fbox{\TheSbox}\end{tabular}\newline}

%% \newenvironment{encuadrar}[1]{\begin{Sbox}\begin{varwidth}{#1\textwidth}}
%% {\end{varwidth}\end{Sbox}\fbox{\TheSbox}}

%% \newenvironment{parboxenv}{\begin{Sbox}}
%% {\end{Sbox}\parbox[t]{.9\textwidth}{\TheSbox}}

% multicolumn y multirow
\newcommand{\mcol}[3]{\multicolumn{#1}{#2}{#3}}
\newcommand{\mrow}[3]{\multirow{#1}{#2}{#3}}

%% Serif .......................................................................
%%
%% New Century Schoolbook
%% \usepackage[T1]{fontenc}
%% \usepackage{fouriernc}
%%
%%
%% TeX Gyre Schola (New Century extendida)
%% \usepackage[T1]{fontenc}
%% \usepackage{tgschola}
%%
%%
%% Utopia
%% \usepackage[T1]{fontenc}
%% \usepackage{fourier}
%%
%%
%% Utopia (con MathDesign)
%% \usepackage[T1]{fontenc}
%% \usepackage[adobe-utopia]{mathdesign}
%%
%%
%% Computer Concrete
%% \usepackage[T1]{fontenc}
%% \usepackage{concmath}
%%
%%
%% Charter BT
 \usepackage[T1]{fontenc}
 \usepackage[bitstream-charter]{mathdesign}
%%
%%
%% Nimbus Roman (clon de Times)
%% \usepackage[T1]{fontenc}
%% \usepackage{nimbus}
%%
%%
%% TeX Gyre Termes (version mejorada de Nimbus Roman)
%% \usepackage[T1]{fontenc}
%% \usepackage{tgtermes}
%%
%%
%% GFS Bodoni
%% \usepackage[T1]{fontenc}
%% \usepackage[default]{gfsbodoni}
%%
%%
%% Baskervald ADF
%% \usepackage[T1]{fontenc}
%% \usepackage{baskervald}
%%
%%
%% Efont Serif -- descargar de http://openlab.jp/efont/serif/
%% \usepackage[T1]{fontenc}
%% \usepackage{efont,mathesf}
%% \renewcommand*\oldstylenums[1]{{\fontfamily{esfod}\selectfont#1}}
%%
%%
%%
%%
%%
%% Sans-Serif ..................................................................
%%
%%
%% Optima (clon de, URW Classico)
%% \usepackage[T1]{fontenc}
%% \renewcommand*\sfdefault{uop}
%%
%%
%% Avantgarde (clon de, URW Gothic)
%% \usepackage[T1]{fontenc}
%% \usepackage{avant}
%%
%%
%% TeX Gyre Adventor (version mejorada de Avantgarde)
%% \usepackage[T1]{fontenc}
%% \usepackage{tgadventor}
%%
%%
%% Nimbus Sans (clon de Helvetica)
%% \usepackage[T1]{fontenc}
%% \usepackage{nimbus}
%%
%%
%% Helvetica (clon de, Nimbus Sans)
%% \usepackage[T1]{fontenc}
%% \usepackage[scaled]{helvet}
%%
%%
%% TeX Gyre Heros (version mejorada de Nimbus Sans)
%% \usepackage[T1]{fontenc}
%% \usepackage{tgheros}
%%
%%
%% Boilinum
%% \usepackage[T1]{fontenc}
%% \usepackage{libertine}
%%
%%
%% Computer Modern Bright
%% \usepackage[T1]{fontenc}
%% \usepackage{cmbright}
%%
%%
%% Latin Modern Sans
%% \usepackage[T1]{fontenc}
%% \usepackage{lmodern}
%%
%%
%% Epigrafica
%% \usepackage[OT1]{fontenc}
%% \usepackage{epigrafica}
%%
%%
%%
%% Si quiero el documento en sans en vez de Roman:
%% \renewcommand*\familydefault{\sfdefault}
%% ...............................................
%% 
%%
%%
%% Monospaced ..................................................................
%%
%%
%% Pandora Typewriter
%% \usepackage[T1]{fontenc}
%% \usepackage{pandora}
%%
%%
%% Letter Gothic
%% \usepackage[T1]{fontenc}
%% \usepackage{ulgothic}
%%
%%
%% Inconsolata
%% \usepackage[T1]{fontenc}
%% \usepackage{inconsolata}
%%

%
\addbibresource{biblio.bib}
%
\selectlanguage{spanish}
\hyphenation{micro-RNA}
\hyphenation{micro-RNAs}
\hyphenation{mi-RNA}
\hyphenation{mi-RNAs}
%
\begin{document}
\selectlanguage{spanish}
%
% pagina de titulo
\begin{titlepage}
%
\titlehead{\center Universidad Nacional del Litoral\\
  Facultad de Ingeniería y Ciencias Hídricas}
%
\subtitle{Ingeniería en Informática\\
  Propuesta de Proyecto Final de Carrera}
%
\title{Desarrollo de un clasificador de secuencias de pre-microRNA
  mediante técnicas de Inteligencia Computacional}
\subject{Informe entregable 1}
\author{Mauro Javier Torrez}
%
\publishers{\-\\[4em]{Director\\Dr. Diego H. Milone}\\[2em]
  {Asesora temática\\Dra. Georgina S. Stegmayer}}
%
\date{\-\\[2em]\today}
%
\renewcommand*{\titlepagestyle}{empty}
%\thispagestyle{empty}
\maketitle
\end{titlepage}
\setcounter{page}{1}
%
%
%
%
\section{Bibliografía consultada}
\section{Bases de datos recopiladas}
\subsection{Xue et al.: Triplet-SVM}
\subsubsection{Objetivo del método presentado}
Identificar secuencias de pre-miRNAs a través de un clasificador SVM.
\subsubsection{Características que utiliza}
\begin{itemize}
\item frecuencia de ocurrencia de 32 elementos ``triplet''
  (en cada posición se toma un elemento de la secuencia + 3 de la estructura secundaria)
\end{itemize}
\subsubsection{Disponibilidad de los datos}
Si bien en la página del autor ya no están disponibles los datos, éstos se encuentran
en una versión anterior, disponible en el archivo de internet: \url{PONER URL}

Para el armado de los conjuntos de datos se consideran los sig. criterios:
\begin{itemize}
\item Mínimo de 18 base pairings en el tallo
\item Máximo -15kcal/mol de free energy
\item Ningún loop múltiple
\end{itemize}

También se encuentran disponibles los vectores de entrada para el
clasificador SVM (frecuencia de los 32 triplets).
%
\subsubsection{Conjuntos de datos}
\paragraph{pre-miRNAs humanos}
Tomados de miRNA registry rel 5.0, sept/2004
\begin{description*}
\item[Tipo:] Conjunto de entrenamiento (163) y prueba(30), datos positivos
\item[Num. entradas:] 193
\item[Especies:] Homo sapiens (hsa)
\item[Características:]
id \quad
secuencia \quad
estructura secundaria \quad
SEQ\_LENGTH \quad
GC\_CONTENT \quad
BASEPAIR \quad
FREE\_ENERGY \quad
LEN\_BP\_RATIO
\end{description*}
\paragraph{CROSS-SPECIES}
Tomados de miRNA registry rel 5.0, sept/2004
\begin{description*}
\item[Tipo:] Conjunto de prueba, datos positivos
\item[Num. entradas:] 581
\item[Especies:]
\quad mmu (36)
\quad rno (25)
\quad gga (13)
\quad dre (6)
\quad cbr (73)
\quad cel (110)
\quad dps (71)
\quad dme (71)
\quad osa (96)
\quad ath (75)
\quad ebv (5)
\item[Características:]
id \quad
secuencia \quad
estructura secundaria \quad
SEQ\_LENGTH \quad
GC\_CONTENT \quad
BASEPAIR \quad
FREE\_ENERGY \quad
LEN\_BP\_RATIO
\end{description*}
\paragraph{pre-miRNAs humanos (actualizado)}
Tomados de miRNA registry rel 5.0, sept/2004
\begin{description*}
\item[Tipo:] Conjunto de prueba, datos positivos
\item[Num. entradas:] 39
\item[Especies:]  Homo sapiens (hsa)
\item[Características:]id \quad
secuencia \quad
estructura secundaria \quad
SEQ\_LENGTH \quad
GC\_CONTENT \quad
BASEPAIR \quad
FREE\_ENERGY \quad
LEN\_BP\_RATIO
\end{description*}
\paragraph{CONSERVED-HAIRPIN}
Tomados de UCSC database, region 56000001--57000000 del cromosoma humano 19
\begin{description*}
\item[Tipo:] Conjunto de prueba, datos desconocidos (algunos positivos)
\item[Num. entradas:] 2444
\item[Especies:]  Homo sapiens (hsa)
\item[Características:]
id \quad
secuencia \quad
estructura secundaria \quad
SEQ\_LENGTH \quad
GC\_CONTENT \quad
BASEPAIR \quad
FREE\_ENERGY \quad
LEN\_BP\_RATIO
\end{description*}
\paragraph{CODING}
Tomados de USC database, secuencias protein-coding (CDSs)
\begin{description*}
\item[Tipo:] Conjunto de entrenamiento, datos negativos
\item[Num. entradas:] 8494
\item[Especies:]  Homo sapiens (hsa)
\item[Características:]
id \quad
secuencia \quad
estructura secundaria \quad
SEQ\_LENGTH \quad
GC\_CONTENT \quad
BASEPAIR \quad
FREE\_ENERGY \quad
LEN\_BP\_RATIO
\end{description*}
%
%
%
%
%
\subsection{Ng \& Mishra: miPred}
\subsubsection{Objetivo del método presentado}
Identificar secuencias de pre-miRNAs a través de un clasificador SVM.
\subsubsection{Características que utiliza}
\begin{itemize}
\item frecuencia de ocurrencia de 32 elementos ``triplet''
  (en cada posición se toma un elemento de la secuencia + 3 de la estructura secundaria)
\end{itemize}
\subsubsection{Disponibilidad de los datos}
Si bien en la página del autor ya no están disponibles los datos, éstos se encuentran
en una versión anterior, disponible en el archivo de internet: \url{PONER URL}

Para el armado de los conjuntos de datos se consideran los sig. criterios:
\begin{itemize}
\item Mínimo de 18 base pairings en el tallo
\item Máximo -15kcal/mol de free energy
\item Ningún loop múltiple
\end{itemize}

También se encuentran disponibles los vectores de entrada para el
clasificador SVM (frecuencia de los 32 triplets).
%
\subsubsection{Conjuntos de datos}
\paragraph{pre-miRNAs humanos}
Tomados de miRNA registry rel 5.0, sept/2004
\begin{description*}
\item[Tipo:] Conjunto de entrenamiento (163) y prueba(30), datos positivos
\item[Num. entradas:] 193
\item[Especies:] Homo sapiens (hsa)
\item[Características:]
id \quad
secuencia \quad
estructura secundaria \quad
SEQ\_LENGTH \quad
GC\_CONTENT \quad
BASEPAIR \quad
FREE\_ENERGY \quad
LEN\_BP\_RATIO
\end{description*}
%
%
%
%
%
\subsection{Batuwita et al.: microPred}
\subsubsection{Objetivo del método presentado}
Identificar secuencias de pre-miRNAs a través de un clasificador SVM.
\subsubsection{Características que utiliza}
\begin{itemize}
\item frecuencia de ocurrencia de 32 elementos ``triplet''
  (en cada posición se toma un elemento de la secuencia + 3 de la estructura secundaria)
\end{itemize}
\subsubsection{Disponibilidad de los datos}
Si bien en la página del autor ya no están disponibles los datos, éstos se encuentran
en una versión anterior, disponible en el archivo de internet: \url{PONER URL}

Para el armado de los conjuntos de datos se consideran los sig. criterios:
\begin{itemize}
\item Mínimo de 18 base pairings en el tallo
\item Máximo -15kcal/mol de free energy
\item Ningún loop múltiple
\end{itemize}

También se encuentran disponibles los vectores de entrada para el
clasificador SVM (frecuencia de los 32 triplets).
%
\subsubsection{Conjuntos de datos}
\paragraph{pre-miRNAs humanos}
Tomados de miRNA registry rel 5.0, sept/2004
\begin{description*}
\item[Tipo:] Conjunto de entrenamiento (163) y prueba(30), datos positivos
\item[Num. entradas:] 193
\item[Especies:] Homo sapiens (hsa)
\item[Características:]
id \quad
secuencia \quad
estructura secundaria \quad
SEQ\_LENGTH \quad
GC\_CONTENT \quad
BASEPAIR \quad
FREE\_ENERGY \quad
LEN\_BP\_RATIO
\end{description*}

%% Batuwita et al.: microPred
%% 1. datos disponibles sin plegar, pero con las features calculadas
%% 2. dataset positivo: 695 pre-miRNAs hsa (mirBase 12)
%% 3. datasets negativos:
%% 1. CODING de Xue (8494)
%% 2. otros ncRNAs hsa: 754 (695 con secstruct multi-branched)
%% 1. features (48):
%% 1. 29 idem miPred
%% 2. 2 MFE-related
%% 3. 4 RNAfold-related
%% 4. 6 Mfold-related
%% 5. 7 calculadas con scripts propios



%% Sewer et al.: Mir-abela
%% 1. extrae candidatos de miRNAs usando sliding windows en las regiones del genoma que se sabe hay miRNAs
%% 2. especies: human, mouse, rat
%% 3. algunas features “calculables”:
%% 1. energía
%% 2. long del stem simple más largo
%% 3. long del loop del hairpin
%% 4. proporción de nt A/C/G/U en el stem
%% 5. proporción de pares A-U/C-G/G-U en el stem
%% 1. no hay datos de entrenamiento





%% Hertel & Stadler: RNAmicro
%% 1. no hay datos
%% 2. features sacables:
%% 1. long stem
%% 2. long loop
%% 3. G+C

%% Helvik et al.: Microprocessor SVM
%% 1. no hay datos
%% 2. valida con 332 miRNAs hsa (miRBase 8.0) + 130 miRNAs (miRBase 8.1)
%% 3. plega con RNAfold default
%% 4. muchas features

%% Yousef et al.: BayesmiRNAfind
%% 1. no hay datos (dice que estan como supplementary data pero en Bioinformatics no están, tampoco en el sitio bioinfo.wistar.upenn.edu)
%% 2. dataset positivo: no queda claro de dónde lo sacó (pasando un sliding window por las regiones candidatas? de mirbase?)
%% 3. dataset negativo: 190739 no-miRNAs sacados aleatoriamente de las 3’UTR de mRNAs humanos.

%% Nam et al: ProMiR
%% 1. no hay datos, solo disponible el dataset positivo, sin plegar, en http://rfam.sanger.ac.uk
%% 2. no usa “features”, sino estados de transición en la secuencia para entrenar un clasificador HMM
%% Jiang et al.: MiPred
%% 1. dataset positivo: mirna registry database, release 8.2
%% 2. dataset negativo: CODING de Xue
%% 3. features(34): Xue + MFE + P-value
%% 4. disponible: 163 hsa (+), 168 random (-), sin plegar, sin características extraídas. Aparentemente idem Xue.



%% Huang et al.: MiRFinder
%% 1. datos no tan disponibles (train sin etiquetar, sólo vectores libsvm, test sólo secuencias, sin features) 
%% 2. features (18):
%% 1. 1: Minimum Free Energy
%% 2. 2: The difference of the MFE of the sequence pair
%% 3. 3: The difference of the structure of the sequence pair
%% 4. 4–7: Base pairing and other properties of the 22 mer hypothesized mature miRNA
%% 5. 8: The mutation frequency of the sequence segment pair
%% 6. 9–18: The frequency of the 10 possible secondary structure elements (combinations of 2 adjacent characters) in the pseudo code of stem region (represented by the new syntax)
%% 1. training (+): vectores libSVM sin etiquetar: mirBase 8.2 human, mouse, pig, cattle, dog, sheep
%% 2. training (-): sequence segments extracted from UCSC genome pair-wise alignments (human, mouse)
%% Lim et al.: mirCheck/mirScan
%% 1. no hay datos
%% 2. especies: C. elegans
%% 3. features:
%% 1. base pairing of the miRNA portion of the fold-back
%% 2. base pairing of the rest of the fold-back
%% 3. stringent sequence conservation in the 5Ј half of the miRNA
%% 4. slightly less stringent sequence conservation in the 3Ј half of the miRNA
%% 5.  sequence biases in the first five bases of the miRNA (especially a U at the first position)
%% 6. a tendency toward having symmetric rather than asymmetric internal loops and bulges in the miRNA region
%% 7. and the presence of two to nine consensus base pairs between the miRNA and the terminal loop region, with a preference for 4–6 bp.
%% Gkirtzou et al: MatureBayes
%% 1. datos no disponibles
%% 2. conjunto train:
%% 1. 533 hsa, 422 rno de mirBase 10.0
%% 1. test:
%% 1. entradas nuevas de hsa, rno en mirBase 11.0-14.0
%% 2. especies dme, zebrafish en mirBase 14
%% 1. no se usan features útiles para un clasificador svm
%% Lai et al.: MiRSeeker
%% 1. no hay datos
%% 2. scope del paper distinto al nuestro
%% Ding et al.: MiRenSVM
%% 1. training:
%% 1. (+) 692 hsa, 52 aga de mirBase 12.0
%% 2. (-) 9225 hsa, 92 aga de UTRdb release 22.0 (long 70-150nt)
%% 3. (-) 754 ncRNAs hsa usados en microPred
%% 4. (-) 256 ncRNAs aga de Rfam 9.1 (<150nt)
%% 1. test:
%% 1. (+) 14 hsa, 14 aga nuevos en mirBase 13.0
%% 2. (+) 5328 pre-miRNAs de otras 27 especies
%% 3. (-) ??? aparentemente parte de los de UTRdb en train (usa sólo 5428 para entrenar)
%% 1. datos no disponibles, aunque se pueden armar los datasets a mano (secuencias disponibles, hace falta plegar, armar conjuntos train/test)
%% 2. features de miPred + features de microPred
%% miPredGA
%% 1. train:
%% 1. (+) idem microPred
%% 2. (-) idem microPred (CODING de Xue + 754 ncRNAs)
%% 1. test:
%% 1. separa del dataset de train
%% 1. features:
%% 1. idem microPred, rankeadas según libSVM
%% 1. datos disponibles de microPred. no aporta nada nuevo en lo que hace los datos. 
%% Sheng et al.: mirCos
%% 1. no sirve
%% Xu et al.: miRank
%% 1. features:
%% 1. 1 MFE normalizada
%% 2. 2 base pairing propensity normalizada (1para c/brazo)
%% 3. 1 long del loop normalizada
%% 4. 32 triplets como Xue
%% 1. train:
%% 1. (+) 533 hsa, 38 aga de mirBase 1/9/2007
%% 2. (-) 1000 hsa, 20000+ aga obtenidos escaneando el genoma con ventanas de 90nt y filtrando por caracts. de plegado
%% 1. test: no especificado
%% 2. datos disponibles: sólo hsa, formato svm, sin info de secuencia/estructura secundaria
\section{Métodos de clasificación codificados y pruebas preliminares}
\section{Métodos de extracción de características codificados}
\section{Armado de la base de datos definitiva}

%
%
\printbibliography
%
\end{document}
