\documentclass[12pt,bibliography=openstyle,DIV=12,parskip=full-]{scrartcl}
\include{conf/preconfig}
\include{conf/packages}
\include{conf/config}
\include{conf/comandos}
\include{conf/fuentes}
%
%
\selectlanguage{spanish}
\hyphenation{micro-RNA}
\hyphenation{micro-RNAs}
\hyphenation{mi-RNA}
\hyphenation{mi-RNAs}
%
\begin{document}
\selectlanguage{spanish}
%
% pagina de titulo
\begin{titlepage}
%
\titlehead{\center Universidad Nacional del Litoral\\
  Facultad de Ingeniería y Ciencias Hídricas}
%
\subject{Ingeniería en Informática\\
  Propuesta de Proyecto Final de Carrera}
%
\title{Desarrollo de un clasificador de secuencias de pre-microRNA
  mediante técnicas de Inteligencia Computacional}
\author{Mauro Javier Torrez}
%
\publishers{\-\\[4em]{Director\\Dr. Diego H. Milone}\\[2em]
  {Asesora temática\\Dra. Georgina S. Stegmayer}}
%
\date{\-\\[2em]\today}
%
\renewcommand*{\titlepagestyle}{empty}
%\thispagestyle{empty}
\maketitle
\end{titlepage}
\setcounter{page}{1}
%
%
\section{Título del Proyecto}
Desarrollo de un clasificador de
secuencias de pre-microRNA mediante
técnicas de Inteligencia Computacional
%
%
\section{Palabras clave}
microRNA, pre-microRNA, Inteligencia Computacional, Máquinas de Vector
de Soporte, Perceptrón Multicapa, identificación, clasificación
%
%
\section{Justificación}
Hace aproximadamente una década se propuso que un nuevo tipo de
pequeñas moléculas de RNA antes ignoradas, aunque abundantes,
jugarían un papel decisivo en la reproducción de las células,
mediando en la diferenciación en distintos tipos de tejidos y/o en la
permanencia de las mismas en un estado particular de diferenciación
\cite{lee-mammal}. Estas moléculas se denominan microRNA o
miRNA. Aunque su mecanismo exacto de acción aún no se conoce
completamente, estudios recientes demuestran que están implicadas, por
ejemplo, en la evolución del cáncer (sea como inhibidores o promotores
de éste) \cite{aurora}, y en procesos de infección viral
\cite{lecellier}.

Los miRNAs son pequeñas cadenas de RNA (ARN, ácido ribonucleico), de
unos 22 nucleótidos de longitud, que ejercen una importante función
reguladora de la expresión génica celular \cite{bartel116}.  Se ha
demostrado su efecto regulador en varios procesos genéticos dentro de
la célula, como la transcripción de mRNAs (\emph{RNA mensajeros}) y la
síntesis de proteínas. Al alterar los niveles de miRNAs en las
células, es posible medir su impacto en estos procesos de forma
cuantitativa \cite{lili}.

El efecto regulador de la expresión génica de los microRNAs en los
procesos celulares puede tener gran implicancia en el desarrollo y
evolución de la enfermedad celular. Expresiones aberrantes de los
miRNAs se han observado en muchos cánceres. Además, se ha demostrado
que juegan un rol importante en la carcinogénesis \cite{lili}. 
Funciones recientemente descubiertas de los miRNAs incluyen el control
de la proliferación y muerte celular, metabolismo de la grasa en
moscas, y control del desarrollo de flores y hojas en
plantas \cite{bartel116}. 

\newpage
Los miRNAs se presentan dentro de una molécula denominada pre-miRNA
(miRNA precursor), de unos ~70nt de longitud, la cual contiene uno o
más miRNAs ``maduros'' en su secuencia.  En la figura \ref{horquilla}
se representa en forma esquemática la secuencia de una cadena de
pre-miRNA y su correspondiente \emph{estructura secundaria}. Esta
estructura secundaria viene dada por la forma en que la cadena de
pre-miRNA se ``plega'' sobre sí misma logrando una mayor estabilidad
molecular. Típicamente, en el reino animal se encuentra que la
estructura secundaria de un pre-miRNA conforma una especie de
\emph{horquilla}, como se puede ver en la figura.
\cite{bartel116}\cite{sewer}

El debate acerca del número e identidad de los miRNAs en diferentes
genomas es una cuestión abierta. A modo de ejemplo, para el caso del
genoma humano se ha estimado que se encontrarían miles de pre-miRNAs
\cite{batuwita}, y hasta ahora han sido identificados unos 1900.  Sin
embargo, este número podría ser mucho mayor, en especial si se tiene
en cuenta que estas predicciones consideran sólo aquellos pre-miRNAs
que se encuentran conservados entre especies relativamente distantes,
como primates y roedores, y no aquellos pre-miRNAs de evolución más
reciente \cite{sewer}.  La base de datos
miRBase\footnote{\url{http://mirbase.org/}}
\cite{mirbase2}\cite{mirbase3} recopila el conjunto de (pre-)miRNAs
conocidos hasta la fecha. En la última versión
disponible\footnote{\url{ftp://mirbase.org/pub/mirbase/20/}}, de junio
de 2013, se listan 24521 pre-miRNAs experimentalmente validados,
conteniendo un total de 30424 miRNAs ``maduros''.
%
\begin{figure}
  \small\slshape\center
  \includegraphics[width=.9\textwidth]{img/hsa-mir-299_ss.pdf}
  \caption{\small\slshape Representación esquemática de la secuencia (arriba) y
    estructura secundaria tipo horquilla (abajo) del pre-miRNA
    \textbf{hsa-mir-299}. Según información disponible en miRBase,
    este pre-miRNA experimentalmente validado contiene 2 miRNAs
    maduros, hsa-mir-299-5p y hsa-mir-299-3p en cada uno de sus
    brazos. Estructura secundaria calculada con RNAfold
    \cite{vienna}.}
  \label{horquilla}
\end{figure}

%\newpage
Inicialmente, la identificación de nuevos miRNAs se realizaba en forma
experimental mediante secuenciado y clonación directa. Ésta es la
primera elección, pero sólo aquellos miRNAs abundantes pueden ser
fácilmente detectados mediante esta técnica. Sin embargo,
no todos los miRNAs están bien expresados en múltiples tipos de
tejidos. Aquellos miRNAs que tienen un bajo nivel de expresión, que se
expresan en tejidos específicos y/o que se presentan en estadíos de
desarrollo celular específicos pueden ser fácilmente ignorados
mediante la técnica experimental. Estudios recientes sugieren que los
miRNAs humanos expresados en bajos niveles evolucionan
rápidamente. \cite{ding}\cite{xu}

En pos de superar estas dificultades propias del método experimental
es que surgen técnicas computacionales para encontrar aquellos miRNAs
que son específicos a determinados tipos de tejidos o estadíos de
desarrollo celular, y aquellos escasamente
expresados. \cite{sheng}\cite{xu}

Los métodos computacionales para el reconocimiento de genes miRNA se
han desarrollado en dos direcciones principales: los métodos
comparativos, basados en la conservación ya sea de la secuencia y/o la
estructura secundaria entre distintas especies, y los no-comparativos,
basados en el aprendizaje de máquina o Inteligencia
Computacional. Estos dos enfoques se complementan mutuamente al
encarar distintas estrategias para la predicción de nuevos
miRNAs. \cite{batuwita}\cite{sheng}

En base a lo que se ha observado en la bibliografía y al problema
planteado, se propone para este trabajo desarrollar un sistema
informático para la identificación de cadenas de pre-microRNA de tipo
no-comparativo, que haga uso de técnicas de Inteligencia Computacional
para la clasificación de patrones.

Una primera parte del sistema consistirá en generar un vector de
características de cada secuencia. Para este propósito se procederá al
diseño y codificación de un algoritmo para la extracción de
características a partir de la propia secuencia en conjunto con la
estructura secundaria y otros indicadores moleculares que se
encuentren disponibles. El conjunto de características a extraer se
determinará en base a criterios del rendimiento obtenido en el
algoritmo de clasificación y la disponibilidad de datos
complementarios a la secuencia y estructura secundaria para todas las
entradas.

%\newpage
La segunda parte del sistema consistirá en clasificar los datos de
entrada identificando aquellas secuencias candidatas a ser
pre-miRNAs. En este caso se trabajará con técnicas de aprendizaje de
máquina como las Máquinas de Vector de Soporte \cite{svm} y Perceptrón
Multicapa \cite{mlp1}\cite{mlp2}. En principio se trabajará con estas
técnicas comparando su rendimiento, y se tomará en consideración la
pertinencia de incluir éstas y otras técnicas en el clasificador
final.
%También se especificará y codificará una interfaz de usuario para la
%utilización del sistema por parte del usuario final.

El impacto social del desarrollo de este proyecto se verá en un
aumento en la productividad de los investigadores del área al
posibilitarle una mayor facilidad de acceso que los sistemas actualmente
disponibles. También se pretende que sirva como puntapié inicial en
desarrollos futuros de sistemas similares con mayor complejidad y
rendimiento. En una visión englobadora, se agilizará la investigación
en el área con las implicancias que esto trae a la sociedad en
general, como el desarrollo de nuevos métodos de prevención y
tratamiento de enfermedades de diversa índole en las personas,
animales y plantas; y una mayor y mejor comprensión de los procesos
involucrados en su aparición y desarrollo.
En lo personal, el desarrollo de este proyecto me permitirá
profundizar mi conocimiento en diferentes áreas de la Inteligencia
Computacional, así como en la introducción al campo de la
bioinformática. Considero que el desarrollo de este proyecto me
permitirá desarrollar habilidades y conocimientos que me serán de gran
valor en el desempeño como profesional, y en caso de seguir una
carrera de investigación servirá de introducción informal a la
metodología de trabajo en este campo.
%
%
\section{Objetivo general del proyecto}
Desarrollar un método computacional inteligente con interfaz de
usuario para la identificación de secuencias de RNA candidatas a ser
pre-miRNAs.
\section{Objetivos específicos}
\begin{itemize}
\item Generar una base de datos de pre-miRNAs conocidos en plantas y
  animales, armonizando los conjuntos de características entre las
  distintas bases de datos disponibles.
\item Codificar diferentes métodos de inteligencia computacional para
  trabajar sobre los datos. Se trabajarán al menos dos, SVM y MLP con
  la posibilidad de incorporar otros conforme se avance en este
  objetivo.
\item Comparar el rendimiento de los distintos métodos codificados y
  parametrizar éste o éstos buscando obtener la mayor performance en
  la identificación.
\item Especificar y codificar una interfaz de línea de comandos y una
  interfaz web para la utilización del método.
\end{itemize}
%
%
\section{Alcances}
\begin{itemize}
\item El trabajo se centrará en la utilización de algoritmos de
  clasificación de tipo Máquina de Vector de Soporte (SVM) y Perceptrón
  Multicapa (MLP), comparando el rendimiento de cada uno.
\item El sistema contará además con una interfaz de usuario
  documentada tal que éste sea accesible a los usuarios destino.
\item Se trabajará durante el desarrollo exclusivamente con datos que
  se encuentren disponibles, ya clasificados y validados
  experimentalmente.
\item En el sistema se trabajará únicamente con la identificación de
  pre-miRNAs, quedando fuera de nuestro alcance la identificación de
  el/los miRNAs “maduros” contenidos en éstos.
\end{itemize}
%
%
\section{Metodología a emplear}
%% El desarrollo del proyecto se realizará en las cuatro etapas que se
%% enumeran a continuación. Cabe notar que estas etapas, si bien proveen
%% una separación lógica entre los diferentes aspectos del trabajo a
%% realizar, no implican necesariamente una separación temporal de
%% tareas. Dada la naturaleza del sistema será necesaria la
%% retroalimentación entre las etapas, lo que implicará un grado mayor o
%% menor de simultaneidad entre las mismas.
El desarrollo del proyecto se llevará a cabo siguiendo un modelo de
ciclo de vida en cascada según las etapas que se enumeran a
continuación.  Cabe notar que, si bien estas etapas proveen una
separación lógica de tareas a tomar como referencia, no implican
necesariamente un orden cronológico de tareas a realizar. Así también,
debido a la naturaleza del proyecto se contempla la posibilidad de una
retroalimentación entre etapas, pudiendo retroceder a una fase
anterior si así es requerido.
%
%% \subsection{Estudio del estado del arte, selección de herramientas informáticas}
%% Se procederá a recopilar y estudiar la bibliografía referente al tema
%% y se determinarán los métodos de clasificación sobre los que se
%% trabajará, las características a extraer, los métodos de extracción de
%% características necesarios. También se evaluarán las diferentes
%% herramientas de desarrollo y lenguajes a utilizar tomando en
%% consideración factores como requerimientos de hardware/software,
%% disponibilidad, rendimiento, portabilidad y facilidad de uso.
\subsection{Estudio del estado del arte y herramientas informáticas
  disponibles}
Se procederá a recopilar y estudiar la bibliografía referente al tema
y se determinarán los métodos de clasificación y de extracción de
características disponibles. También se estudiarán las diferentes
herramientas de desarrollo y lenguajes disponibles tomando en
consideración factores como requerimientos de hardware/software,
disponibilidad, rendimiento, portabilidad y facilidad de uso.
%
%% \subsection{Armado de la base de datos y desarrollo de los algoritmos específicos}
%% Se realizará una búsqueda de datos disponibles en Internet a partir de
%% la bibliografía consultada. Con estos datos se generará una base de
%% datos de pre-miRNAs conocidos y validados experimentalmente así como
%% de secuencias que no sean pre-miRNAs para su utilización en el diseño
%% de los algoritmos de clasificación.

%% Se codificarán métodos de
%% inteligencia computacional para trabajar sobre la base de datos
%% armada. Se evaluará la performance de clasificación de los métodos
%% mediante métricas estadísticas estándares.
\subsection{Especificación de requerimientos y pruebas iniciales}
Se realizará una búsqueda de datos disponibles en Internet a partir de
la bibliografía consultada.  Se codificarán varios métodos de
clasificación y se probarán sobre una de las bases de datos
encontradas para obtener una mejor comprensión de los requerimientos
del sistema.
%
Se evaluarán las herramientas informáticas disponibles para la
implementación y se especificarán aquellas a utilizar según los
resultados obtenidos en las pruebas iniciales.
%
\subsection{Preparación de los datos y codificación de partes del sistema}
Se recopilará el conjunto de datos encontrados en la etapa anterior
y se realizará un análisis exhaustivo de éstos.
Se codificarán algoritmos de extracción de características a partir de
la secuencia para su incorporación a la base de datos.
Se preparará una base
de datos propia mediante la selección, limpieza y adecuaciones de
formato en cuanto sea necesario con el objetivo de garantizar la
calidad de la misma. Esta base de datos contendrá secuencias de
pre-miRNAs conocidos y validados experimentalmente así como de
secuencias que no sean pre-miRNAs (ejemplos negativos) los cuales se
utilizarán como entrada para los algoritmos de extracción de
características y clasificación.

El algoritmo de clasificación se desarrollará siguiendo
aproximadamente un modelo de prototipos. En forma iterativa se
realizarán pruebas sobre los clasificadores disponibles variando sus
parámetros y el conjuntos de características de entrada. Cada
iteración se evaluará mediante validación cruzada obteniendo así
medidas de rendimiento discriminadas. A partir de estas medidas de
rendimiento se especificarán el conjunto de características a
considerar y los algoritmos a utilizar en la parte de clasificación,
con sus parámetros correspondientes. Con esta especificación se procederá
a la codificación del algoritmo de clasificación definitivo.
%
%
\subsection{Integración del sistema e interfaz de usuario}
Se definirá el formato de los datos de entrada del sistema final y se
codificará un sistema integrado que incorpore extracción de
características y clasificación. Se seleccionarán los parámetros de
clasificación disponibles (configurables) al usuario final así como
valores por defecto que aseguren un buen rendimiento.
%
Se especificará y codificará una interfaz de usuario de línea de
comandos y una interfaz web, previa evaluación y selección de las
tecnologías a utilizar para la implementación.  Se generará la
documentación correspondiente para la utilización del sistema por
parte del usuario final.
%
\subsection{Pruebas finales y puesta en servicio}
En esta etapa se procederá a evaluar el sistema, verificando el buen
funcionamiento y el cumplimiento de los requerimientos. Se realizarán
pruebas de clasificación para obtener medidas de rendimiento
globales. Mediante pruebas se intentará encontrar y eliminar
aquellos fallos que pudieren haber sido pasados por alto durante el
desarrollo.
%
Finalmente, se procederá a la configuración de un servidor web de
ejemplo y se realizará la puesta en servicio del sistema sobre el
mismo, evaluando y especificando las tecnologías a utilizar para tal
fin.
%
%
\newpage
\section{Plan de tareas propuesto}
Para el desarrollo del proyecto, el alumno dedicará 20 horas semanales
de trabajo para un total de 580 horas.  Se realizarán entre 2 y 4
reuniones mensuales de una hora con el director de proyecto para
seguimiento de avance y asesoramiento.
%
A continuación se presenta la lista de tareas y sub-tareas a
desarrollar en el transcurso del proyecto.
\begin{enumerate}
\item Búsqueda bibliográfica (44h)
  \begin{enumerate*}
  \item Información específica del problema en cuestión: descripción
    de los microRNAs, mecanismo y función, implicancias en el campo de
    la biología (24h).
  \item Perspectiva informática del tema: estructura de los datos de
    entrada, extracción de características, métodos de clasificación
    (20h).
  \end{enumerate*}
\item Implementación inicial de métodos de clasificación (36h)
  \begin{enumerate*}
  \item Búsqueda y verificación de una base de datos de prueba ya
    clasificada y con las características extraídas (8h).
  \item Codificación inicial de clasificadores para obtener una
    primera impresión de las herramientas y lenguajes disponibles,
    requerimientos de software y hardware, y rendimiento (28h).
  \end{enumerate*}
\item Preparación de la base de datos (60h)
  \begin{enumerate*}
  \item Búsqueda y verificación de diferentes bases de datos (12h).
  \item Armado de una base de datos propia unificando el formato y
    características disponibles (24h).
  \item Codificación de métodos para la extracción de características
    a partir de la secuencia e incorporación de las mismas a la base
    de datos (24h).
  \end{enumerate*}
\item Redacción del informe entregable 1 (12h)
\item Diseño del clasificador definitivo (100h)
  \begin{enumerate*}
  \item Pruebas de los clasificadores preliminares (64h).\\
    Tarea iterativa:
    \begin{itemize*}
    \item Selección del método de clasificación.
    \item Ajuste de parámetros del método.
    \item Selección de características de entrada.
    \item Validación cruzada.
    \end{itemize*}
  \item Codificación definitiva del clasificador (36h).
  \end{enumerate*}
\item Pruebas e integración del sistema (80h)
  \begin{enumerate*}
  \item Validación del método de clasificación (20h).
  \item Integración del sistema: extracción de características y
    clasificación (24h).
  \item Especificación y codificación de la interfaz de usuario de
    línea de comandos (16h).
  \item Pruebas de funcionamiento y validación de requerimientos (20h).
  \end{enumerate*}
\item Redacción del informe entregable 2 (12h)
%\item Redacción del informe entregable 3 (12h)
\item Interfaz de usuario web, documentación y puesta en servicio (104h)
  \begin{enumerate*}
  \item Especificación de los lenguajes/tecnologías a utilizar en la
    interfaz web (12h).
  \item Codificación de la interfaz web (36h).
  \item Documentación de las interfaces de usuario de línea de
    comandos y web (20h).
  \item Configuración y puesta en servicio de un servidor de prueba
    (36h).
  \end{enumerate*}
\item Redacción del informe entregable 3 (12h)
\item Redacción del informe final (120h)
\end{enumerate}
%
%
\section{Puntos de control y entregables}
\subsection{Punto de control 1}
Resultados de la revisión bibliográfica, pruebas preliminares y armado
de la base de datos definitiva.  Abarca las etapas 1 y 2 y la primera
parte de la etapa 3.
\begin{description*}
  \item[Fecha:] 25 de julio de 2013
  \item[Entregable:]
  \item
    \begin{minipage}{\textwidth}
      \medskip
      \begin{itemize*}
      \item Bibliografía consultada
      \item Bases de datos recopiladas
      \item Métodos de clasificación codificados y pruebas preliminares
      \item Métodos de extracción de características codificados
      \item Armado de la base de datos definitiva
      \end{itemize*}
    \end{minipage}
\end{description*}
\newpage
\subsection{Punto de control 2}
Resultados de la implementación del clasificador definitivo e
integración del sistema. Abarca la segunda parte de la etapa 3 y la
etapa 4 excepto la interfaz web y la documentación para el usuario
final.
\begin{description*}
  \item[Fecha:] 1 de octubre de 2013
  \item[Entregable:]
  \item
    \begin{minipage}{\textwidth}
      \medskip
      \begin{itemize*}
      \item Resultados de las pruebas de los distintos clasificadores
      \item Especificación del algoritmo de clasificación definitivo:
        métodos seleccionados, parámetros, conjunto de características
        considerado
      \item Algoritmo de clasificación codificado
      \item Resultados de la validación del algoritmo de clasificación
      \item Codificación del sistema integrado
      \item Interfaz de usuario de línea de comandos codificada
      \item Resultados de la validación de requerimientos
      \end{itemize*}
    \end{minipage}
\end{description*}
\subsection{Punto de control 3}
Interfaz de usuario web, puesta en servicio y documentación para el
usuario final. Abarca parte de la etapa 4 y la etapa 5.
\begin{description*}
  \item[Fecha:] 4 de noviembre de 2013
  \item[Entregable:]
  \item
    \begin{minipage}{\textwidth}
      \medskip
      \begin{itemize*}
      \item Documentación de la interfaz de línea de comandos
      \item Tecnologías web seleccionadas para la interfaz de usuario
      \item Interfaz de usuario web codificada
      \item Documentación de la interfaz de usuario web
      \item Servidor de prueba puesto en servicio
      \end{itemize*}
    \end{minipage}
\end{description*}
%
%
\newpage
\section{Cronograma tentativo}
En el siguiente cronograma se distribuye la carga de trabajo de 580
horas en 29 semanas. Se toma como fecha de inicio del proyecto el día
3 de junio de 2013 y como fecha de finalización estimada el día 20 de
diciembre de 2013.
\begin{center}
\definecolor{barblue}{RGB}{153,204,254}
\definecolor{groupblue}{RGB}{51,102,254}
\definecolor{linkred}{RGB}{165,0,33}
\sffamily
\begin{ganttchart}[
y unit chart=5.5mm,
x unit=0.95mm,
y unit title=0.6cm,
hgrid style/.append style={draw=black!5, line width=.75pt},
vgrid={*4{draw=black!15, line width=.75pt},%
  *1{draw=black!40, line width=.75pt}},
canvas/.append style={fill=none, draw=black!40, line width=.75pt},
include title in canvas=true,
title height=1,
title/.append style={draw=black!40, line width=.75pt,fill=none},
title label font=\bfseries\footnotesize,
bar label font=\mdseries\small\color{black!70},
bar incomplete/.append style={fill=black!63},
bar height=.6,
%bar label node/.append style={left=2cm},
bar/.append style={draw=none, fill=barblue},
progress label text=,
group incomplete/.append style={fill=groupblue},
group left shift=0,
group right shift=0,
group top shift=.3,
group height=.6,
group/.append style={shape=ganttbar},
milestone/.append style={shape=ganttmilestone,fill=linkred,draw=none},
milestone label font=\slshape\bfseries\small\color{black!80},
milestone height=.55,
milestone top shift=.35,
milestone left shift=-0.5,
milestone right shift=1.5,
milestone label node/.append style={below=8pt,left=0pt},
]{1}{145}
\gantttitle{Jun}{20}\gantttitle{Jul}{23}\gantttitle{Ago}{22}\gantttitle{Sep}{21}
\gantttitle{Oct}{23}\gantttitle{Nov}{21}\gantttitle{Dic}{15}\\
\gantttitle[title label node/.append style={below left=-1.26ex and -1.1ex}]%
{Semana:\quad1}{5}\gantttitlelist{2,...,29}{5} \\
% 20 horas semanales: 1 dia = 4h
\ganttgroup{Tarea 1}{1}{11} \\%11 dias
\ganttbar{1.a}{1}{6} \\ %24h
\ganttbar{1.b}{7}{11} \\ %20h
%
\ganttgroup{Tarea 2}{12}{20} \\%9 dias
\ganttbar{2.a}{12}{13} \\ %8h
\ganttbar{2.b}{14}{20} \\ %28h
%
\ganttgroup{Tarea 3}{21}{35} \\%15 dias
\ganttbar{3.a}{21}{23} \\%12h
\ganttbar{3.b}{24}{29} \\%24h
\ganttbar{3.c}{30}{35} \\%24h
% entregable 1
\ganttgroup{Tarea 4}{36}{38} \\%3 dias
\ganttmilestone{Control 1}{38}\\
%
\ganttgroup{Tarea 5}{39}{63} \\%25 dias
\ganttbar{5.a}{39}{54} \\%64h
\ganttbar{5.b}{55}{63} \\%20h
%
\ganttgroup{Tarea 6}{64}{83} \\%20 dias
\ganttbar{6.a}{64}{68} \\%20h
\ganttbar{6.b}{69}{74} \\%24h
\ganttbar{6.c}{75}{78} \\%16h
\ganttbar{6.d}{79}{83} \\%20h
% entregable 2
\ganttgroup{Tarea 7}{84}{86} \\%3 dias
\ganttmilestone{Control 2}{86}\\
\ganttgroup{Tarea 8}{87}{112} \\%26 dias
\ganttbar{8.a}{87}{89} \\%12h
\ganttbar{8.b}{90}{98} \\%36h
\ganttbar{8.c}{99}{103}\\%20h
\ganttbar{8.d}{104}{112} \\%36h
% 10 entregable 3
%\ganttgroup{Tarea 9}{88}{90} \\%3 dias
%\ganttmilestone{Control 3}{90}\\
% entregable 3
\ganttgroup{Tarea 9}{113}{115} \\%3 dias
\ganttmilestone{Control 3}{115}\\
\ganttgroup{Tarea 10}{116}{145} %30 dias
\end{ganttchart}
\end{center}
%
%
\section{Riesgos y estrategias de mitigación}
%
\subsection{Problemas en el armado de la base de datos}
Al trabajar con bases de datos de origen diverso y sobre las
cuales no se tienen garantías de calidad, se presenta el riesgo de
encontrar más inconsistencias de lo previsto en la etapa de armado de
la base de datos.
\begin{description*}
  \item[Probabilidad:] Media
  \item[Impacto:] Retraso en el armado de la base de datos definitiva
    al requerirse tiempo extra de depuración
  \item[Mitigación:] En caso de tratarse de una base de datos
    particularmente problemática, se analizará la conveniencia de no
    considerarla para el armado de la base definitiva.
\end{description*}
%
\subsection{Retraso en los tiempos previstos por razones ajenas al proyecto}
Al encontrarse el estudiante trabajando en un área que es ajena al
desarrollo de este proyecto, se considera el riesgo de una carga
laboral excesiva que pudiera provocar un retraso en el desarrollo del
actual proyecto. En pos de disminuir el presente riesgo, se ha
conversado el tema en el entorno laboral del alumno.
\begin{description*}
  \item[Probabilidad:] Baja
  \item[Impacto:] Retraso en el cumplimiento del cronograma del
    proyecto
  \item[Mitigación:] Replanificar tareas intentando mantener las
    fechas previstas, dedicando más horas de trabajo al desarrollo del
    proyecto.
\end{description*}
%
\subsection{Problemas de portabilidad, compatibilidad y/o
  licencias del software de base para el clasificador en el servidor Web}
Dado que los servidores Web en general poseen recursos limitados y un
entorno de software administrado diferente a aquel de un equipo de
escritorio, se da el riesgo de que el software utilizado como soporte
en el sistema (lenguajes de programación, software específico) no
pueda ser utilizado en el servidor de interfaz Web.
\begin{description*}
  \item[Probabilidad:] Media
  \item[Impacto:] Retraso en la implementación de la interfaz Web,
    necesidad de volver a codificar partes del sistema en otro
    lenguaje compatible
  \item[Mitigación:] Búsqueda de software alternativo a utilizar en el
    servidor y recodificación de aquellas partes del sistema
    incompatibles. Como último recurso, se implementará el servidor en
    la misma máquina de desarrollo, aunque tal elección implique que
    el servicio no estará disponible para su utilización en un entorno
    de producción.
\end{description*}
%
%
\section{Recursos necesarios y disponibles}
Al momento de iniciar el proyecto, todos los recursos necesarios se
encuentran disponibles:
\begin{itemize*}
\item Material bibliográfico
\item Servicios:
  \begin{itemize*}
  \item Conexión a Internet
  \item Servidor Web (Instancia de Amazon EC2 ``micro'': Intel Xeon
    1x2.6GHz, 600MB RAM)
  \end{itemize*}
\item Hardware:
  \begin{itemize*}
  \item PC de escritorio (Intel Core 2, 4GB RAM)
  \item Notebook (Intel Core i5, 4GB RAM)
  \end{itemize*}
\item Software:
  \begin{itemize*}
  \item Sistema operativo Debian GNU/Linux 7.0 ``Wheezy''
  \item Entorno de desarrollo/Editor GNU Emacs 24
  \item Software científico (MATLAB, GNU Octave)
  \item Software de scripting (Bash, Python, Perl)
  \item Software de servidor Web (Debian GNU/Linux, Apache
    httpd/nginx, PHP 5)
  \end{itemize*}
\item Bases de datos para el desarrollo y pruebas
\item Insumos varios:
  \begin{itemize*}
  \item Artículos de librería
  \item Impresora y tóner
  \item Fotocopias
  \item Pendrive
  \item Pasajes en colectivo
  \end{itemize*}
\item Recursos humanos: alumno y director.
\end{itemize*}
%
%
\newpage
\section{Presupuesto}
En la tabla a continuación se detalla el presupuesto para el Proyecto
Final. Para la elaboración del mismo, se tuvieron en cuenta las
siguientes consideraciones:
\begin{enumerate*}
\item El costo de la hora hombre del Alumno se considera al valor de
  mercado de un programador Junior en \$40/h.
\item Se considera que el Director dedicará 15 horas mensuales en el
  seguimiento, asesoramiento y correcciones a la tarea del Alumno. El
  costo de la hora hombre del Director se toma en \$200/h.
\item Se considera el software con precio nulo, ya que se en principio
  se utilizará software libre. La decisión de utilizar software
  comercial incidirá en el costo indirecto con el valor de la
  licencia.
\item No se consideran intereses así como
  tampoco el costo de oportunidad.
\end{enumerate*}
{\sffamily\small
\newcommand\GR[1]{{\bfseries #1}}
\newcommand\SL[1]{{\slshape #1}}
\begin{tabular}{p{4cm}crlrr}
  \GR{Tarea/Concepto}&  \GR{Recurso}  & \mcol{2}{c}{\GR{Cantidad}}
                                                     &\GR{Costo unitario}
                                                               &\GR{Costo total}\\\hline
    \mrow{3}{*}{Tareas 1 a 4}
                     & RRHH alumno    &   152 & hs.  & \$  40  & \$ 6080  \\
                     & RRHH director  &    30 & hs.  & \$ 200  & \$ 6000  \\
                     & Impresión      &   200 & pág. & \$ 0,2  & \$ 40    \\
    \mcol{5}{l}{\quad\SL{Subtotal C.D. tareas 1--4}}           & \SL{\$12120}\\\hline
    \mrow{3}{*}{Tareas 5 a 7}
                     & RRHH alumno    &   164 & hs.  & \$  40  & \$ 6560  \\
                     & RRHH director  &    30 & hs.  & \$ 200  & \$ 6000  \\
                     & Impresión      &   100 & pág. & \$ 0,2  & \$ 20    \\
    \mcol{5}{l}{\quad\SL{Subtotal C.D. tareas 5--7}}           & \SL{\$12580}\\\hline
    \mrow{4}{*}{Tareas 8 a 9}
                     & RRHH alumno    &   116 & hs.  & \$  40  & \$ 4640  \\
                     & RRHH director  &    15 & hs.  & \$ 200  & \$ 3000  \\
                 & Renta servidor web & 3000  & hs.  &US\$ 0.06& \$ 1188%
    \footnote{Precio estimado. Se considera el dólar americano a
      \$5,50 pesos argentinos y aplicando el 20\% de recargo a compras
      en el exterior según R.G. AFIP 3450/2013}                           \\
                     & Impresión      &   100 & pág. & \$ 0,2  & \$ 30    \\
    \mcol{5}{l}{\quad\SL{Subtotal C.D. tareas 8--9}}           & \SL{\$8858} \\\hline
    \mrow{4}{*}{Tarea 10}
                     & RRHH alumno    &   120 & hs.  & \$  40  & \$ 4800  \\
                     & RRHH director  &    20 & hs.  & \$ 200  & \$ 4000  \\
                     & Impresión      &   400 & pág. & \$ 0,2  & \$ 80    \\
                     & Encuadernado   &     3 & unid.& \$ 30   & \$ 120   \\
    \mcol{5}{l}{\quad\SL{Subtotal C.D. tarea 10}}              & \SL{\$9000} \\\hline
    \mrow{5}{*}{Costos Indirectos}
                     & Serv. Internet &     7 & mes  & \$  200 & \$ 1400  \\
                     & PC escritorio  &     1 & unid.& \$ 6000 & \$ 6000  \\
                     & Transporte     &    80 & pas. & \$ 2,9  & \$ 232   \\
                & Elementos de oficina&\mcol{2}{c}{N/A}&       & \$ 150   \\
                     & Software     &\mcol{2}{c}{N/A}& \$ 0    & \$ 0     \\
    \mcol{5}{l}{\quad\SL{Subtotal C.I.}}                       & \SL{\$7782} \\\hline
    \mcol{5}{l}{\GR{Costo total del Proyecto}}                  & \GR{\$50340}\\\hline
  \end{tabular}}
%
%
\printbibliography
%
\end{document}
