\documentclass[12pt,bibliography=oldstyle,DIV=12,parskip=half-,titlepage]{scrartcl}
\include{conf/preconfig}
\include{conf/packages}
\include{conf/config}
\include{conf/comandos}
\include{conf/fuentes}
%
\addbibresource{res/bibliografia.bib}
%
\selectlanguage{spanish}
\hyphenation{micro-RNA}
\hyphenation{micro-RNAs}
\hyphenation{mi-RNA}
\hyphenation{mi-RNAs}
%
%
%
%
\addtokomafont{descriptionlabel}{\small}
\setkomafont{subject}{\LARGE\usekomafont{disposition}}
\setkomafont{title}{\normalfont\slshape}
\setkomafont{subtitle}{\LARGE\usekomafont{disposition}}
%
\begin{document}
\selectlanguage{spanish}
%
% pagina de titulo
%
\titlehead{\center\large
    Universidad Nacional del Litoral\\
    Facultad de Ingeniería y Ciencias Hídricas
}
%
%
\title{\LARGE ``Desarrollo de un clasificador de secuencias de pre-microRNA
  mediante técnicas de Inteligencia Computacional''}
\subject{Proyecto Final de Carrera\\Ingeniería en
  Informática}
\subtitle{~\\[.2ex]Informe entregable 3\\[.2ex]~}
\author{{Alumno: Mauro J. Torrez}\and{Director: Dr. Diego H. Milone}}
%
\date{~\\[2em]\today}
%
\renewcommand*{\titlepagestyle}{empty}
%\thispagestyle{empty}
\maketitle
\setcounter{page}{1}
%
%
%\section{}
En el presente Informe se presenta una revisión de las tareas
efectuadas por el alumno según lo estipulado en la Propuesta de
Proyecto Final de Carrera.
%
%
\section{Interfaz de usuario web}
%
Para la creación de la interfaz de usuario web se optó por utilizar la
herramienta \eng{Web-demo builder}\footnote{Página web:
  \url{https://bitbucket.org/sinc-lab/webdemobuilder/}}, desarrollado
en el laboratorio \eng{sinc(i)} de la Facultad de Ingeniería y
Ciencias Hídricas.

Se codificó un método específico para la utilización del software a
través de una interfaz web, y un archivo en formato {\mono Makefile}
que permite la generación del archivo empaquetado en formato zip para
utilizar con el software Web-demo builder.

Finalmente, se procedió a la puesta en servicio de una interfaz web de
demostración en el servidor de pruebas provisto por los creadores de
Web-demo builder, la cual está disponible en la siguiente URL:

\url{http://ec2-52-5-194-68.compute-1.amazonaws.com/scripts/57769864/webif/}
%
%
\section{Documentación de usuario}
%
Se generó documentación para el usuario en el mismo código fuente para
referencia en línea, así como mediante la creación de una guía de
usuario que se adjunta al presente Informe.

La documentación del código fuente se realizó siguiendo el estándar de
documentación de Matlab, resultando en la adición de aproximadamente
700 líneas de ayuda que el usuario puede consultar en la línea de
comandos de Matlab mediante la función {\mono help}.

Además, se redactó una guía del usuario con instrucciones de
instalación, requerimientos del sistema, utilización de la interfaz de
línea de comandos, y generación y utilización de la interfaz web.  Se
adjunta a este documento una copia de dicha guía del usuario.
%
%
%
\renewcommand{\bibfont}{\normalfont\footnotesize}
\printbibliography
\end{document}
