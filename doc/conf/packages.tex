%
% ===
% === Trick para detectar si el documento está siendo compilado con pdflatex
% ===
%
% Esto me setea la variable pdf dependiendo del valor de \pdfoutput, que es >0
% sólo cuando estoy usando pdflatex para compilar el documento. Con esto puedo
% hacer  \ifpdf {...} \fi, que se ejecuta colo cuando compilo con pdflatex.
%% \newif\ifpdf
%% \ifnum\pdfoutput<0
%% \pdffalse\fi
%% \ifnum\pdfoutput=0
%% \pdffalse\fi
%% \ifnum\pdfoutput>0
%% \pdftrue\fi
%
% ===
% === I18n / L10n
% ===
%
% babel me da separación de sílabas para palabras en el idioma que le paso como
%       argumento opcional.
\usepackage[spanish,english]{babel}
%
% inputenc define la codificación de caracteres del código fuente, acá utf8.
\usepackage[utf8]{inputenc}
%
% ===
% === Gráficos
% ===
% 
% pst-pdf me permite usar PSTricks con pdflatex. Necesito cargarlo sólo si está
%         definida la variable pdf, por eso está entre \ifpdf ... \fi
%\ifpdf\usepackage{pst-pdf}\fi
%
% color me permite usar colores en el documento.
\usepackage{color}
%
% graphicx me da el comando \includegraphics para insertar imágenes (?)
\usepackage{graphicx}
%
% pstricks es un conjunto de macros basadas en PostScript para TeX, en
%          castellano: me da un entorno pstricks y comandos que uso dentro de
%          éste, que me sirven para dibujar figuras/diagramas/etc de manera
%          relativamente simple.
%\usepackage{pstricks}
%
% pst-circ me da macros para pstricks que me dibujan elementos de circuitos
%\usepackage{pst-circ}
%
% pst-plot me provee de funciones de ploteo para pstricks
%\usepackage{pst-plot}
%
% pst-2dplot me sirve para plotear en pstricks, entorno pstaxes
%\usepackage{pst-2dplot}
%
% ===
% === Verbatims
% ===
%
% verbatim es una reimplementación de los entornos verbatim[*]
%          provee el comando \verbatiminput{archivo} y el entorno comment, que
%          hace que LaTeX ignore directamente todo lo que está adentro
%\usepackage{verbatim}
%
% moreverb implementa el entorno verbatimtab indentando los tabs que encuentre,
%          y también el entorno listing, que pone números de línea al verbatim.
%          Para cambiar el ancho de la tabulacion, uso
%          \renewcommand\verbatimtabsize{<ancho del tab>\relax}
%          También define el entorno boxedverbatim.
%\usepackage{moreverb}
%
% listings me da el entorno lstlisting con resaltado de sintaxis.
%          Para setear el lenguaje del código, hago \lstset{language=<lang>}
%\usepackage{listings}
%
% url es un verbatim para escribir URL's que permite linebreaks dentro de ésta.
%     para usarlo, \url{<URL>}
\usepackage{url}
%
% ===
% === Más packages
% ===
%
%%\usepackage{mdwlist}		%Para listas mas compactas
%% \usepackage{textcomp}		%Para algunos símbolos
%% \usepackage{colortbl}		%Para celdas de colores en tablas
%% \usepackage{fancyhdr}		%Para encabezados/pie
%% \usepackage{bbold}		%Fuente bb para modo math: \mathbb{R} = reales
%% \usepackage{dsfont}		%Fuente ds para modo math: \mathds{R} = reales
\usepackage{multirow}		%Para "combinar" celdas en tablas
\usepackage{float}		%Para mejorar cuadros, figuras, etc
%% \usepackage{fancybox}		%Para recuardos de texto con bordes "fancy"
%% \usepackage{dingbat}		%Para dingbats
%\usepackage{marginal}		%Para  notas al margen que no puedo hacer andar
\usepackage{amsmath}		%Para enornos matemáticos mas flexibles
%\usepackage{varwidth}		%varwidth es un minipage que se ajusta al ancho mínimo


\usepackage[backend=biber,sorting=none,style=ieee,eprint=false,url=false]{biblatex} %% style=ieee
%% requiere texlive-bibtex-extra en debian


\usepackage{enumitem}
\setlist{noitemsep}
%% \setlist[description]{noitemsep}
%% \setlist[enumerate]{noitemsep}
%% \setlist[itemize]{noitemsep}

\usepackage{tikz}
\usepackage{pgfkeys}
\usepackage{pgfgantt}

% typearea: uso con koma-script para ajustar márgenes de página.
% vars globales a setear en la clase koma-script: DIV=12, BCOR=margen de ``binding'' para double side
\usepackage{typearea}

% para poder usar footnotes p.ej, adentro de un tabular
\usepackage{footnote}
\makesavenoteenv{tabular}

% para tabulars mas lindos/legibles
\usepackage{booktabs}

%\usepackage{glossaries}

\usepackage[spanish]{algorithm2e}
