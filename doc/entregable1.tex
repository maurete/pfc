\documentclass[12pt,bibliography=openstyle,DIV=12,parskip=half-]{scrartcl}
\include{conf/preconfig}
\include{conf/packages}
\include{conf/config}
\include{conf/comandos}
\include{conf/fuentes}
%
\addbibresource{biblio.bib}
%
\selectlanguage{spanish}
\hyphenation{micro-RNA}
\hyphenation{micro-RNAs}
\hyphenation{mi-RNA}
\hyphenation{mi-RNAs}
%
\begin{document}
\selectlanguage{spanish}
%
% pagina de titulo
\begin{titlepage}
%
\titlehead{\center Universidad Nacional del Litoral\\
  Facultad de Ingeniería y Ciencias Hídricas}
%
\subtitle{Ingeniería en Informática\\
  Propuesta de Proyecto Final de Carrera}
%
\title{Desarrollo de un clasificador de secuencias de pre-microRNA
  mediante técnicas de Inteligencia Computacional}
\subject{Informe entregable 1}
\author{Mauro Javier Torrez}
%
\publishers{\-\\[4em]{Director\\Dr. Diego H. Milone}\\[2em]
  {Asesora temática\\Dra. Georgina S. Stegmayer}}
%
\date{\-\\[2em]\today}
%
\renewcommand*{\titlepagestyle}{empty}
%\thispagestyle{empty}
\maketitle
\end{titlepage}
\setcounter{page}{1}
%
%
%
%
\section{Bibliografía consultada}
\section{Bases de datos recopiladas}
\subsection{Xue et al.: Triplet-SVM}
\subsubsection{Objetivo del método presentado}
Identificar secuencias de pre-miRNAs a través de un clasificador SVM.
\subsubsection{Características que utiliza}
\begin{itemize}
\item frecuencia de ocurrencia de 32 elementos ``triplet'' (en cada
  posición se toma un elemento de la secuencia + 3 de la estructura
  secundaria)
\end{itemize}
\subsubsection{Disponibilidad de los datos}
Si bien en la página del autor ya no están disponibles los datos,
éstos se encuentran en una versión anterior, disponible en el archivo
de internet: \url{http://web.archive.org/web/*/http://bioinfo.au.tsinghua.edu.cn/mirnasvm/}

Para el armado de los conjuntos de datos se consideran los
sig. criterios:
\begin{itemize}
\item Mínimo de 18 base pairings en el tallo
\item Máximo -15kcal/mol de free energy
\item Ningún loop múltiple
\end{itemize}

También se encuentran disponibles los vectores de entrada para el
clasificador SVM (frecuencia de los 32 triplets).
%
\subsubsection{Conjuntos de datos}
\paragraph{pre-miRNAs humanos}
Tomados de miRNA registry rel 5.0, sept/2004
\begin{description*}
\item[Tipo:] Conjunto de entrenamiento (163) y prueba(30), datos
  positivos
\item[Num. entradas:] 193
\item[Especies:] Homo sapiens (hsa)
\item[Características:] id \quad secuencia \quad estructura secundaria
  \quad SEQ\_LENGTH \quad GC\_CONTENT \quad BASEPAIR \quad
  FREE\_ENERGY \quad LEN\_BP\_RATIO
\end{description*}
\paragraph{CROSS-SPECIES}
Tomados de miRNA registry rel 5.0, sept/2004
\begin{description*}
\item[Tipo:] Conjunto de prueba, datos positivos
\item[Num. entradas:] 581
\item[Especies:]
\quad mmu (36)
\quad rno (25)
\quad gga (13)
\quad dre (6)
\quad cbr (73)
\quad cel (110)
\quad dps (71)
\quad dme (71)
\quad osa (96)
\quad ath (75)
\quad ebv (5)
\item[Características:]
id \quad
secuencia \quad
estructura secundaria \quad
SEQ\_LENGTH \quad
GC\_CONTENT \quad
BASEPAIR \quad
FREE\_ENERGY \quad
LEN\_BP\_RATIO
\end{description*}
%
\paragraph{pre-miRNAs humanos (actualizado)}
Tomados de miRNA registry rel 5.0, sept/2004
\begin{description*}
\item[Tipo:] Conjunto de prueba, datos positivos
\item[Num. entradas:] 39
\item[Especies:]  Homo sapiens (hsa)
\item[Características:]id \quad
secuencia \quad
estructura secundaria \quad
SEQ\_LENGTH \quad
GC\_CONTENT \quad
BASEPAIR \quad
FREE\_ENERGY \quad
LEN\_BP\_RATIO
\end{description*}
%
\paragraph{CONSERVED-HAIRPIN}
Tomados de UCSC database, region 56000001--57000000 del cromosoma
humano 19
\begin{description*}
\item[Tipo:] Conjunto de prueba, datos desconocidos (algunos
  positivos)
\item[Num. entradas:] 2444
\item[Especies:]  Homo sapiens (hsa)
\item[Características:]
id \quad
secuencia \quad
estructura secundaria \quad
SEQ\_LENGTH \quad
GC\_CONTENT \quad
BASEPAIR \quad
FREE\_ENERGY \quad
LEN\_BP\_RATIO
\end{description*}
\paragraph{CODING}
Tomados de USC database, secuencias protein-coding (CDSs)
\begin{description*}
\item[Tipo:] Conjunto de entrenamiento, datos negativos
\item[Num. entradas:] 8494
\item[Especies:]  Homo sapiens (hsa)
\item[Características:]
id \quad
secuencia \quad
estructura secundaria \quad
SEQ\_LENGTH \quad
GC\_CONTENT \quad
BASEPAIR \quad
FREE\_ENERGY \quad
LEN\_BP\_RATIO
\end{description*}
%
%
%
%
%
\subsection{Ng \& Mishra: miPred}
\subsubsection{Objetivo del método presentado}
Identificar secuencias de pre-miRNAs a través de un clasificador SVM.

El método además es probado para detectar pre-miRNAs en genes de
virus, recorriendo éste mediante una ventana deslizante de 95nt y
clasificando las secuencias extraídas.

\url{http://web.bii.a-star.edu.sg/archive/stanley/Publications/}
\subsubsection{Características que utiliza}
\begin{itemize}
\item $dP$
\item $dG$
\item $MFEI_1$
\item $dQ$
\item $dD$
\item $dF$
\item $MFEI_2$
\item otras?
\end{itemize}
\subsubsection{Disponibilidad de los datos}
Los datos se encuentran disponibles, según se listan en las entradas
siguientes.
%
\subsubsection{Conjuntos de datos}
\paragraph{pre-miRNAs no redundantes}
Tomados de miRBase 8.2 (jul/2006) para todas las especies disponibles,
se filtran aquellas secuencias redundantes mediante un algoritmo de
agrupación incremental ``ávido''\cite{greedy}.
\begin{description*}
\item[Tipo:] Conjunto de datos positivos
\item[Num. entradas:] 2241
\item[Especies:] 45 especies agrupadas en artropoda, nematoda,
  vertebrata, viridiplantae y virus
\item[Características:] id \quad secuencia \quad estructura secundaria
  \quad Len \quad A \quad C \quad G \quad U \quad G+C \quad A+U \quad
  AA \quad AC \quad AG \quad AU \quad CA \quad CC \quad CG \quad CU
  \quad GA \quad GC \quad GG \quad GU \quad UA \quad UC \quad UG \quad
  UU \quad \%A \quad \%C \quad \%G \quad \%U \quad \%G+C \quad \%A+U
  \quad \%AA \quad \%AC \quad \%AG \quad \%AU \quad \%CA \quad \%CC
  \quad \%CG \quad \%CU \quad \%GA \quad \%GC \quad \%GG \quad \%GU
  \quad \%UA \quad \%UC \quad \%UG \quad \%UU \quad pb \quad Npb \quad
  mfe \quad Nmfe \quad Q \quad NQ \quad D \quad ND \quad
\end{description*}

\paragraph{ncRNAs funcionales}
Tomados de Rfam 7.0, consiste en un conjunto de ncRNAs de los que se
han eliminado 46 tipos de pre-miRNAs.
\begin{description*}
\item[Tipo:] Conjunto de datos negativos
\item[Num. entradas:] 12387
\item[Especies:] prokaryota y eukaryota (?)
\item[Características:] id \quad secuencia \quad estructura secundaria
  \quad Len \quad A \quad C \quad G \quad U \quad G+C \quad A+U \quad
  AA \quad AC \quad AG \quad AU \quad CA \quad CC \quad CG \quad CU
  \quad GA \quad GC \quad GG \quad GU \quad UA \quad UC \quad UG \quad
  UU \quad \%A \quad \%C \quad \%G \quad \%U \quad \%G+C \quad \%A+U
  \quad \%AA \quad \%AC \quad \%AG \quad \%AU \quad \%CA \quad \%CC
  \quad \%CG \quad \%CU \quad \%GA \quad \%GC \quad \%GG \quad \%GU
  \quad \%UA \quad \%UC \quad \%UG \quad \%UU \quad pb \quad Npb \quad
  mfe \quad Nmfe \quad Q \quad NQ \quad D \quad ND \quad
\end{description*}

\paragraph{mRNAs}
Consiste en 31 mRNAs mensajeros que se pliegan en estructuras
complejas con MFEs extremadamente negativas. Tomados de GeneBank DNA
Database.
\begin{description*}
\item[Tipo:] Conjunto de datos negativos
\item[Num. entradas:] 31
\item[Especies:] varias, entre ellas: perro, hsa, mmu, cel, rno, ...
\item[Características:] id \quad secuencia \quad estructura secundaria
  \quad Len \quad A \quad C \quad G \quad U \quad G+C \quad A+U \quad
  AA \quad AC \quad AG \quad AU \quad CA \quad CC \quad CG \quad CU
  \quad GA \quad GC \quad GG \quad GU \quad UA \quad UC \quad UG \quad
  UU \quad \%A \quad \%C \quad \%G \quad \%U \quad \%G+C \quad \%A+U
  \quad \%AA \quad \%AC \quad \%AG \quad \%AU \quad \%CA \quad \%CC
  \quad \%CG \quad \%CU \quad \%GA \quad \%GC \quad \%GG \quad \%GU
  \quad \%UA \quad \%UC \quad \%UG \quad \%UU \quad pb \quad Npb \quad
  mfe \quad Nmfe \quad Q \quad NQ \quad D \quad ND \quad
\end{description*}

\paragraph{pseudo-hairpins}
Ídem al dataset CODING de triplet-svm, en este caso con más
características extraídas
\begin{description*}
\item[Tipo:] Conjunto de datos negativos
\item[Num. entradas:] 8494
\item[Especies:] homo sapiens (hsa)
\item[Características:] id \quad secuencia \quad estructura secundaria
  \quad Len \quad A \quad C \quad G \quad U \quad G+C \quad A+U \quad
  AA \quad AC \quad AG \quad AU \quad CA \quad CC \quad CG \quad CU
  \quad GA \quad GC \quad GG \quad GU \quad UA \quad UC \quad UG \quad
  UU \quad \%A \quad \%C \quad \%G \quad \%U \quad \%G+C \quad \%A+U
  \quad \%AA \quad \%AC \quad \%AG \quad \%AU \quad \%CA \quad \%CC
  \quad \%CG \quad \%CU \quad \%GA \quad \%GC \quad \%GG \quad \%GU
  \quad \%UA \quad \%UC \quad \%UG \quad \%UU \quad pb \quad Npb \quad
  mfe \quad Nmfe \quad Q \quad NQ \quad D \quad ND \quad
\end{description*}
%
%
%
%
%
\subsection{Batuwita et al.: microPred}
\subsubsection{Objetivo del método presentado}
Identificar secuencias de pre-miRNAs a través de un clasificador SVM.
\subsubsection{Características que utiliza}
Se utilizan las mismas características que en miPred además de las
listadas a continuación.

El vecor de características a utilizar se determina aplicando ``F-Score''
al vector completo (ver paper).
\begin{itemize*}
\item MFEI3
\item MFEI4
\item NEFE
\item Freq
\item Diversity
\item ...
\end{itemize*}
\subsubsection{Disponibilidad de los datos}

%
\subsubsection{Conjuntos de datos}
\paragraph{pre-miRNAs humanos}
Tomados de miRBase 12. Contiene 660 pre-miRNAs single-loop y 31 multi-loop.
\begin{description*}
\item[Tipo:] Conjunto de datos positivos
\item[Num. entradas:] 691
\item[Especies:] Homo sapiens (hsa)
\item[Características:]
id \quad secuencia \quad \textbf{estructura secundaria no disponible}
\quad \%G+C \quad \%AA \quad \%AC \quad \%AG \quad \%AU \quad \%CA
\quad \%CC \quad \%CG \quad \%CU \quad \%GA \quad \%GC \quad \%GG
\quad \%GU \quad \%UA \quad \%UC \quad \%UG \quad \%UU \quad MFEI1
\quad MFEI2 \quad dG \quad dP \quad dQ \quad dD \quad dF \quad ZG
\quad ZP \quad ZQ \quad ZD \quad ZF \quad MFEI3 \quad MFEI4 \quad EAFE
\quad Freq \quad Diversity \quad Diff \quad dH \quad dH/L \quad dS
\quad dS/L \quad Tm \quad Tm/L \quad |A-U|/L \quad |G-C|/L \quad
|G-U|/L \quad Avg\_Bp\_Stem \quad \%(A-U)/stems \quad \%(G-C)/stems
\quad \%(G-U)/stems
\end{description*}

\paragraph{pseudo-hairpins}
Ídem dataset CODING de Xue, con más características extraídas.
\begin{description*}
\item[Tipo:] Conjunto de datos negativos
\item[Num. entradas:] 8494
\item[Especies:] Homo sapiens (hsa)
\item[Características:]
id \quad secuencia \quad \textbf{estructura secundaria no disponible}
\quad \%G+C \quad \%AA \quad \%AC \quad \%AG \quad \%AU \quad \%CA
\quad \%CC \quad \%CG \quad \%CU \quad \%GA \quad \%GC \quad \%GG
\quad \%GU \quad \%UA \quad \%UC \quad \%UG \quad \%UU \quad MFEI1
\quad MFEI2 \quad dG \quad dP \quad dQ \quad dD \quad dF \quad ZG
\quad ZP \quad ZQ \quad ZD \quad ZF \quad MFEI3 \quad MFEI4 \quad EAFE
\quad Freq \quad Diversity \quad Diff \quad dH \quad dH/L \quad dS
\quad dS/L \quad Tm \quad Tm/L \quad |A-U|/L \quad |G-C|/L \quad
|G-U|/L \quad Avg\_Bp\_Stem \quad \%(A-U)/stems \quad \%(G-C)/stems
\quad \%(G-U)/stems
\end{description*}



\paragraph{Otros ncRNAs humanos}
Base de datos ensamblada manualmente por los autores. Contiene ncRNAs humanos
que no son miRNAs.
\begin{description*}
\item[Tipo:] Conjunto de datos negativos
\item[Num. entradas:] 754
\item[Especies:] Homo sapiens (hsa)
\item[Características:]
id \quad secuencia \quad \textbf{estructura secundaria no disponible}
\quad \%G+C \quad \%AA \quad \%AC \quad \%AG \quad \%AU \quad \%CA
\quad \%CC \quad \%CG \quad \%CU \quad \%GA \quad \%GC \quad \%GG
\quad \%GU \quad \%UA \quad \%UC \quad \%UG \quad \%UU \quad MFEI1
\quad MFEI2 \quad dG \quad dP \quad dQ \quad dD \quad dF \quad ZG
\quad ZP \quad ZQ \quad ZD \quad ZF \quad MFEI3 \quad MFEI4 \quad EAFE
\quad Freq \quad Diversity \quad Diff \quad dH \quad dH/L \quad dS
\quad dS/L \quad Tm \quad Tm/L \quad |A-U|/L \quad |G-C|/L \quad
|G-U|/L \quad Avg\_Bp\_Stem \quad \%(A-U)/stems \quad \%(G-C)/stems
\quad \%(G-U)/stems
\end{description*}

%% Batuwita et al.: microPred
%% 1. datos disponibles sin plegar, pero con las features calculadas
%% 2. dataset positivo: 695 pre-miRNAs hsa (mirBase 12)
%% 3. datasets negativos:
%% 1. CODING de Xue (8494)
%% 2. otros ncRNAs hsa: 754 (695 con secstruct multi-branched)
%% 1. features (48):
%% 1. 29 idem miPred
%% 2. 2 MFE-related
%% 3. 4 RNAfold-related
%% 4. 6 Mfold-related
%% 5. 7 calculadas con scripts propios



%% Sewer et al.: Mir-abela
%% 1. extrae candidatos de miRNAs usando sliding windows en las regiones del genoma que se sabe hay miRNAs
%% 2. especies: human, mouse, rat
%% 3. algunas features “calculables”:
%% 1. energía
%% 2. long del stem simple más largo
%% 3. long del loop del hairpin
%% 4. proporción de nt A/C/G/U en el stem
%% 5. proporción de pares A-U/C-G/G-U en el stem
%% 1. no hay datos de entrenamiento





%% Hertel & Stadler: RNAmicro
%% 1. no hay datos
%% 2. features sacables:
%% 1. long stem
%% 2. long loop
%% 3. G+C

%% Helvik et al.: Microprocessor SVM
%% 1. no hay datos
%% 2. valida con 332 miRNAs hsa (miRBase 8.0) + 130 miRNAs (miRBase 8.1)
%% 3. plega con RNAfold default
%% 4. muchas features

%% Yousef et al.: BayesmiRNAfind
%% 1. no hay datos (dice que estan como supplementary data pero en Bioinformatics no están, tampoco en el sitio bioinfo.wistar.upenn.edu)
%% 2. dataset positivo: no queda claro de dónde lo sacó (pasando un sliding window por las regiones candidatas? de mirbase?)
%% 3. dataset negativo: 190739 no-miRNAs sacados aleatoriamente de las 3’UTR de mRNAs humanos.

%% Nam et al: ProMiR
%% 1. no hay datos, solo disponible el dataset positivo, sin plegar, en http://rfam.sanger.ac.uk
%% 2. no usa “features”, sino estados de transición en la secuencia para entrenar un clasificador HMM
%% Jiang et al.: MiPred
%% 1. dataset positivo: mirna registry database, release 8.2
%% 2. dataset negativo: CODING de Xue
%% 3. features(34): Xue + MFE + P-value
%% 4. disponible: 163 hsa (+), 168 random (-), sin plegar, sin características extraídas. Aparentemente idem Xue.



%% Huang et al.: MiRFinder
%% 1. datos no tan disponibles (train sin etiquetar, sólo vectores libsvm, test sólo secuencias, sin features) 
%% 2. features (18):
%% 1. 1: Minimum Free Energy
%% 2. 2: The difference of the MFE of the sequence pair
%% 3. 3: The difference of the structure of the sequence pair
%% 4. 4–7: Base pairing and other properties of the 22 mer hypothesized mature miRNA
%% 5. 8: The mutation frequency of the sequence segment pair
%% 6. 9–18: The frequency of the 10 possible secondary structure elements (combinations of 2 adjacent characters) in the pseudo code of stem region (represented by the new syntax)
%% 1. training (+): vectores libSVM sin etiquetar: mirBase 8.2 human, mouse, pig, cattle, dog, sheep
%% 2. training (-): sequence segments extracted from UCSC genome pair-wise alignments (human, mouse)
%% Lim et al.: mirCheck/mirScan
%% 1. no hay datos
%% 2. especies: C. elegans
%% 3. features:
%% 1. base pairing of the miRNA portion of the fold-back
%% 2. base pairing of the rest of the fold-back
%% 3. stringent sequence conservation in the 5Ј half of the miRNA
%% 4. slightly less stringent sequence conservation in the 3Ј half of the miRNA
%% 5.  sequence biases in the first five bases of the miRNA (especially a U at the first position)
%% 6. a tendency toward having symmetric rather than asymmetric internal loops and bulges in the miRNA region
%% 7. and the presence of two to nine consensus base pairs between the miRNA and the terminal loop region, with a preference for 4–6 bp.
%% Gkirtzou et al: MatureBayes
%% 1. datos no disponibles
%% 2. conjunto train:
%% 1. 533 hsa, 422 rno de mirBase 10.0
%% 1. test:
%% 1. entradas nuevas de hsa, rno en mirBase 11.0-14.0
%% 2. especies dme, zebrafish en mirBase 14
%% 1. no se usan features útiles para un clasificador svm
%% Lai et al.: MiRSeeker
%% 1. no hay datos
%% 2. scope del paper distinto al nuestro
%% Ding et al.: MiRenSVM
%% 1. training:
%% 1. (+) 692 hsa, 52 aga de mirBase 12.0
%% 2. (-) 9225 hsa, 92 aga de UTRdb release 22.0 (long 70-150nt)
%% 3. (-) 754 ncRNAs hsa usados en microPred
%% 4. (-) 256 ncRNAs aga de Rfam 9.1 (<150nt)
%% 1. test:
%% 1. (+) 14 hsa, 14 aga nuevos en mirBase 13.0
%% 2. (+) 5328 pre-miRNAs de otras 27 especies
%% 3. (-) ??? aparentemente parte de los de UTRdb en train (usa sólo 5428 para entrenar)
%% 1. datos no disponibles, aunque se pueden armar los datasets a mano (secuencias disponibles, hace falta plegar, armar conjuntos train/test)
%% 2. features de miPred + features de microPred
%% miPredGA
%% 1. train:
%% 1. (+) idem microPred
%% 2. (-) idem microPred (CODING de Xue + 754 ncRNAs)
%% 1. test:
%% 1. separa del dataset de train
%% 1. features:
%% 1. idem microPred, rankeadas según libSVM
%% 1. datos disponibles de microPred. no aporta nada nuevo en lo que hace los datos. 
%% Sheng et al.: mirCos
%% 1. no sirve
%% Xu et al.: miRank
%% 1. features:
%% 1. 1 MFE normalizada
%% 2. 2 base pairing propensity normalizada (1para c/brazo)
%% 3. 1 long del loop normalizada
%% 4. 32 triplets como Xue
%% 1. train:
%% 1. (+) 533 hsa, 38 aga de mirBase 1/9/2007
%% 2. (-) 1000 hsa, 20000+ aga obtenidos escaneando el genoma con ventanas de 90nt y filtrando por caracts. de plegado
%% 1. test: no especificado
%% 2. datos disponibles: sólo hsa, formato svm, sin info de secuencia/estructura secundaria
\section{Métodos de clasificación codificados y pruebas preliminares}
\section{Métodos de extracción de características codificados}
\section{Armado de la base de datos definitiva}
\begin{description}
\item[Identificador] 
\item[Secuencia] 
\item[Estructura secundaria] 
\item[Energía mínima de plegado] 
\item[] 
\item[] 
\item[] 
\item[] 
\item[] 
\item[] 
\item[] 
\item[] 
\item[] 
\item[] 
\item[] 
\item[] 
\item[] 
\item[] 
\item[] 
\item[] 
\item[] 
\item[] 
\item[] 
\item[] 
\item[] 
\item[] 
\item[] 
\item[] 
\item[] 
\item[] 
\item[] 
\item[] 
\item[] 
\item[] 
\item[] 
\item[] 
\item[] 
\item[] 
\item[] 
\item[] 
\item[] 
\item[] 
\item[] 
\item[] 
\item[] 
\item[] 
\item[] 
\item[] 
\item[] 
\item[] 
\item[] 
\item[] 
\item[] 
\item[] 
\item[] 
\item[] 
\item[] 
\item[] 
\item[] 
\item[] 
\item[] 
\item[] 
\end{description}
%
%
\printbibliography
%
\end{document}
