\documentclass[12pt,bibliography=oldstyle,DIV=12,parskip=half-,titlepage]{scrartcl}
% esto me setea la variable pdf dependiendo del valor de \pdfoutput, que es >0
% sólo cuando estoy usando pdflatex para compilar el documento
%% \newif\ifpdf
%% \ifnum\pdfoutput<0
%% \pdffalse\fi
%% \ifnum\pdfoutput=0
%% \pdffalse\fi
%% \ifnum\pdfoutput>0
%% \pdftrue\fi


%
% ===
% === Trick para detectar si el documento está siendo compilado con pdflatex
% ===
%
% Esto me setea la variable pdf dependiendo del valor de \pdfoutput, que es >0
% sólo cuando estoy usando pdflatex para compilar el documento. Con esto puedo
% hacer  \ifpdf {...} \fi, que se ejecuta colo cuando compilo con pdflatex.
%% \newif\ifpdf
%% \ifnum\pdfoutput<0
%% \pdffalse\fi
%% \ifnum\pdfoutput=0
%% \pdffalse\fi
%% \ifnum\pdfoutput>0
%% \pdftrue\fi
%
% ===
% === I18n / L10n
% ===
%
% babel me da separación de sílabas para palabras en el idioma que le paso como
%       argumento opcional.
\usepackage[spanish,es-tabla,english]{babel}
%
% inputenc define la codificación de caracteres del código fuente, acá utf8.
\usepackage[utf8]{inputenc}
%
% ===
% === Gráficos
% ===
% 
% pst-pdf me permite usar PSTricks con pdflatex. Necesito cargarlo sólo si está
%         definida la variable pdf, por eso está entre \ifpdf ... \fi
%\ifpdf\usepackage{pst-pdf}\fi
%
% color me permite usar colores en el documento.
\usepackage{color}
%
% graphicx me da el comando \includegraphics para insertar imágenes (?)
\usepackage{graphicx}
%
% pstricks es un conjunto de macros basadas en PostScript para TeX, en
%          castellano: me da un entorno pstricks y comandos que uso dentro de
%          éste, que me sirven para dibujar figuras/diagramas/etc de manera
%          relativamente simple.
%\usepackage{pstricks}
%
% pst-circ me da macros para pstricks que me dibujan elementos de circuitos
%\usepackage{pst-circ}
%
% pst-plot me provee de funciones de ploteo para pstricks
%\usepackage{pst-plot}
%
% pst-2dplot me sirve para plotear en pstricks, entorno pstaxes
%\usepackage{pst-2dplot}
%
% ===
% === Verbatims
% ===
%
% verbatim es una reimplementación de los entornos verbatim[*]
%          provee el comando \verbatiminput{archivo} y el entorno comment, que
%          hace que LaTeX ignore directamente todo lo que está adentro
%\usepackage{verbatim}
%
% moreverb implementa el entorno verbatimtab indentando los tabs que encuentre,
%          y también el entorno listing, que pone números de línea al verbatim.
%          Para cambiar el ancho de la tabulacion, uso
%          \renewcommand\verbatimtabsize{<ancho del tab>\relax}
%          También define el entorno boxedverbatim.
%\usepackage{moreverb}
%
% listings me da el entorno lstlisting con resaltado de sintaxis.
%          Para setear el lenguaje del código, hago \lstset{language=<lang>}
%\usepackage{listings}
%
% url es un verbatim para escribir URL's que permite linebreaks dentro de ésta.
%     para usarlo, \url{<URL>}
\usepackage{url}
%
% ===
% === Más packages
% ===
%
%% \usepackage{mdwlist}		%Para listas mas compactas
%% \usepackage{textcomp}		%Para algunos símbolos
%% \usepackage{colortbl}		%Para celdas de colores en tablas
%% \usepackage{fancyhdr}		%Para encabezados/pie
\usepackage{bbold}		%Fuente bb para modo math: \mathbb{R} = reales
\usepackage{dsfont}		%Fuente ds para modo math: \mathds{R} = reales
\usepackage{multirow}		%Para "combinar" celdas en tablas
\usepackage{float}		%Para mejorar cuadros, figuras, etc
%% \usepackage{fancybox}		%Para recuardos de texto con bordes "fancy"
%% \usepackage{dingbat}		%Para dingbats
%\usepackage{marginal}		%Para  notas al margen que no puedo hacer andar
\usepackage{amsmath}		%Para enornos matemáticos mas flexibles
%\usepackage{varwidth}		%varwidth es un minipage que se ajusta al ancho mínimo


\usepackage[backend=biber,sorting=none,style=ieee,eprint=false,url=false]{biblatex} %% style=ieee
%% requiere texlive-bibtex-extra en debian


\usepackage{enumitem}
\setlist{noitemsep}
%% \setlist[description]{noitemsep}
%% \setlist[enumerate]{noitemsep}
%% \setlist[itemize]{noitemsep}

\usepackage{tikz}
\usepackage{pgfkeys}
\usepackage{pgfgantt}

% typearea: uso con koma-script para ajustar márgenes de página.
% vars globales a setear en la clase koma-script: DIV=12, BCOR=margen de ``binding'' para double side
\usepackage{typearea}

% para poder usar footnotes p.ej, adentro de un tabular
\usepackage{footnote}
\makesavenoteenv{tabular}

% para tabulars mas lindos/legibles
\usepackage{booktabs}

%\usepackage{glossaries}

\usepackage[spanish]{algorithm2e}

% para highlight (comando \hl{})
\usepackage{soulutf8}

% para teoremans etc
\usepackage{amsthm}

% para tunear citations
%\usepackage[square,comma,numbers,sort&compress]{natbib}


% config.tex: configuraciones del documento
%\selectlanguage{spanish}		%Elijo idioma español

%Permitir que los entornos equation, align, etc permitan saltos de página
%\allowdisplaybreaks[1]

%Tweaks
%% \setlength{\parindent}{0mm}		%Sangría de 1a. línea
%% \setlength{\hoffset}{2.6mm}		%
%% \setlength{\voffset}{-5.4mm}		%
%% \setlength{\topmargin}{0mm}		%
%% \setlength{\oddsidemargin}{5mm}	%
%% \setlength{\evensidemargin}{5mm}	%
%% \setlength{\marginparsep}{5mm}	%
%% \setlength{\headheight}{12.5mm}	%
%% \setlength{\headsep}{2.5mm}		%
%% \setlength{\footskip}{10mm}		%
%% \setlength{\textwidth}{14.1cm}		%
%% \setlength{\textheight}{232mm}	%
%% \setlength{\fboxrule}{.1pt}
%% \setlength{\parskip}{.5\baselineskip}

%Colores
\definecolor{negro}	{cmyk}{0,0,0,1}
\definecolor{marron}	{cmyk}{0,.5,1,.41}
\definecolor{rojo}	{cmyk}{0,1,1,0}
\definecolor{naranja}	{cmyk}{0,.35,1,0}
\definecolor{amarillo}	{cmyk}{0,0,1,0}
\definecolor{verde}	{cmyk}{1,0,1,0}
\definecolor{azul}	{cmyk}{1,1,0,0}
\definecolor{violeta}	{cmyk}{.45,1,0,0}
\definecolor{gris}	{cmyk}{0,0,0,.5}
\definecolor{blanco}	{cmyk}{0,0,0,0}
\definecolor{dorado}	{cmyk}{0,.16,1,0}
\definecolor{plateado}	{cmyk}{0,0,0,.25}

%% \title{\titulo}
%% \author{\autor}
%% \date{\fecha}

% si uso pdflatex, me setea las propiedades del pdf de salida
%% \ifpdf\pdfinfo{/Title    (\tituloPDF)
%%                /Author   (\autorPDF)
%%                /Subject  (\asuntoPDF)
%%                /Keywords (\clavesPDF)}\fi

% comandos.tex
% en este archivo defino todos los comandos/environment que quiera usar en mi documento.
%
% ===
% === Comandos
% ===
% 
% T: para escribir texto común cuando en modo math
%    uso: \T{texto que aparecerá en letra normal}
\newcommand{\T}{\textrm}
%
% aclaracion: dibuja un recuadrito aclaratorio, como <quote> en HTML.
%             uso: \aclaracion{Texto...}
\newcommand{\aclaracion}[1]{%
\smallpencil\-\begin{minipage}{0.9\textwidth}
%\vspace*{6pt}
{#1}\smallskip\end{minipage}}
%
% consigna: parecido a aclaración, pero con texto _slanted_
%           uso: \consigna{Consigna...}
\newcommand{\consigna}[1]{%
\leftpointright\ \parbox[t]{0.9\textwidth}{\textsl{#1}\vspace{8pt}}}
%
% pinterno: para representar el producto interno entre los dos argumentos
%           uso: \pinterno{X}{Y}
\newcommand{\pinterno}[2]{%
\left\langle #1 , #2 \right\rangle}
%
% === Estilos de texto
%
% resalt: resaltado con fondo verde
%         uso: \resalt{texto resaltado}
\newcommand{\resalt}{\colorbox{yellow}}
%
% sfbf: texto en negrita + slanted
%       uso:
\newcommand{\sfbf}[1]{\textsf{\bfseries #1}}
%
% small bold sans-serif
\newcommand{\sbs}[1]{\textsf{\small\bfseries #1}}
%
% eng: itálica (para palabras en inglés)
%      uso: \eng{some English text}
\newcommand{\eng}{\textit}
%
% mean: significado de una sigla - slanted
%       uso: (...) SNCF: \mean{Société Nationale des Chemins de Fer Francais} ...
\newcommand{\mean}{\textsl}
\newcommand{\desc}{\textsl}
%
% defin: pone en negrita el texto, útil para definiciones
%        uso: \defin{asshole}: vulgar slang for anus
\newcommand{\defin}{\textbf}
%
% R, N: cambia la tipografía en modo math, probar para ver cómo quedan
%       uso: \R{R} , \N{N}
\newcommand{\R}{\mathds}
\newcommand{\N}{\mathbf}
\newcommand{\C}{\mathcal}
\newcommand{\B}{\boldsymbol}
%
% dx: para escribir d2y/dx2, etc
\newcommand{\dx}[2]{\frac{d^{#2}\!#1}{d\!x^{#2}}}
%
% dp: para escribir derivadas parciales d2y/dx2, etc
\newcommand{\dpar}[3]{\frac{\partial^{#3}#1}{\partial{#2}^{#3}}}
%
% dvar: para escribir derivadas totales d2y/d(VAR)2, etc
\newcommand{\dvar}[3]{\frac{d^{#3}#1}{d{#2}^{#3}}}
%
% evalen: para escribir (loquesea)|_{evaluado_en}
\newcommand{\evalen}[2]{\left.{#2}\right|_{#1}}
%
% lil: para escribir texto pequeño. más cómodo que { \footnotesize texto pequeño... }
%      uso: \lil{texto pequeño... }
\newcommand{\lil}[1]{\footnotesize #1}  %Para texto pequeñooo
%
% mono: escribe el texto que le paso como parámetro con letra de ancho fijo
%       uso: \mono{texto monoespaciado}
\newcommand{\mono}[1]{{\texttt{#1}}}
%
% === Símbolos
%
\newcommand{\y}{\wedge}			%Y (Lógica)
\newcommand{\ve}{\vee}			%O (Lógica)
\newcommand{\ent}{\supset}		%Entonces (Lógica)
\newcommand{\dimp}{\leftrightarrow}	%Doble implicativo, equivalencia (Lógica)
\newcommand{\sii}{\leftrightarrow}	%Si y sólo si (Lógica)
\newcommand{\equi}{\equiv}		%Equivalencia (Lógica)
\newcommand{\portanto}{\vdash}		%Por lo tanto (Lógica)
\newcommand{\por}{\cdot}		%Producto punto
\newcommand{\RR}[1][1]{\mathds{R}}	%R de reales
\newcommand{\hfi}{\hat{\phi}}           %fi con gorrito arriba
\newcommand{\bfi}{\bar{\phi}}           %fi con raya arriba
\newcommand{\hpsi}{\hat{\psi}}          %Letra griega psi con gorrito arriba
\newcommand{\II}{\B{I}}                 %w negrita
\newcommand{\KK}{\B{K}}                 %w negrita
\newcommand{\QQ}{\B{Q}}                 %w negrita
\newcommand{\YY}{\B{Y}}                 %w negrita
\newcommand{\Bg}{\B{g}}                 %w negrita
\newcommand{\nn}{\B{n}}                 %w negrita
\newcommand{\uu}{\B{u}}                 %w negrita
\newcommand{\vv}{\B{v}}                 %w negrita
\newcommand{\ww}{\B{w}}                 %w negrita
\newcommand{\xx}{\B{x}}                 %x negrita
\newcommand{\yy}{\B{y}}                 %y negrita
\newcommand{\zz}{\B{z}}                 %z negrita
\newcommand{\BPhi}{\B{\Phi}}            %\Phi negrita
\newcommand{\Balpha}{\B{\alpha}}        %\alpha negrita
\newcommand{\Bbeta}{\B{\beta}}          %\beta negrita
\newcommand{\Btheta}{\B{\theta}}        %\theta negrita
\newcommand{\Bxi}{\B{\xi}}              %\xi negrita
%
%
% ===
% === Environments
% ===
% 
% enunciado: un environment que básicamente tiene el mismo efecto que el
%            comando consigna.
%            uso: \begin{enunciado} ... contenido ... \end{enunciado}
\newenvironment{enunciado}
{\leftpointright\ \begin{varwidth}[t]{0.9\textwidth}\textsl}
{\end{varwidth}\vspace{8pt}}
%
% pvi: para tipear la definición de un problema de valor inicial/funciones
%      definidas de a trozos/etc directamente en el texto (sin necesidad de
%      cambiar a un modo matemático.
%      uso: \begin{pvi} linea1 \\ linea 2 \\ ... \end{pvi}
\newenvironment{pvi}{\begin{equation}\begin{cases}}
{\end{cases}\end{equation}}
%
% pvi*: comp pvi, pero sin número de ecuación
\newenvironment{pvi*}{\begin{equation*}\begin{cases}}
{\end{cases}\end{equation*}}
%
% verbatimsmall: un verbatim con letra más chica. usualmente queda bastante
%                mejor que el verbatim pelado.
%                uso: \begin{verbatimsmall} ........ \end{verbatimsmall}
\newenvironment{verbatimsmall}{\small\begin{verbatim*}}
{\end{verbatim*}}
%
% nota: escribe una aclaracion dentro del texto
\newenvironment{nota}{$$\left[\;\begin{minipage}{0.95\textwidth}\slshape}
{\end{minipage}\;\right]$$}
%
%
% ===
% === Comandos ``históricos''
% ===
%
%% %\begin{pspicture}
%% \def\tierra(#1){%Para dibujar el símbolo de tierra en el entorno PSTricks
%% 	\rput(#1){
%% 		\psdot(0,0)
%% 		\psline(0,0)(0,-0.45)
%% 		\psline(-0.5,-0.45)(0.5,-0.45)
%% 		\psline(-0.35,-0.6)(0.35,-0.6)
%% 		\psline(-0.2,-0.75)(0.2,-0.75)
%% 	}%
%% }
%% %\end{pspicture}

\newcommand{\codigo}[2]{%Para generar un recuadro con código
	%\setlength{\hrulewidth}{0.1pt}
	\begin{flushleft}
	\underline{#1}
	\begin{tabular}{@{\quad}|l}
		\begin{minipage}{.85\textwidth}\smallskip{#2}
	\end{minipage}\end{tabular}\end{flushleft}%
}

\newcommand{\filecodigo}[1]{%Insertar código verbatim desde un archivo
\codigo{#1}{\verbatiminput{#1}}}%Requiere el paquete verbatim
\newcommand{\filecodigobis}[1]{{\verbatiminput{#1}}}%Requiere el paquete verbatim

%% \newcommand{\grafico}[3][1]{%Para generar un plot de un archivo con coords.
%% %\def\deequis=#1
%% \begin{minipage}{0.5\textwidth}\begin{center}
%% \begin{pspicture}(6,5)
%% 	\psgrid[subgriddiv=1,gridlabels=0pt,gridwidth=.1pt](1,3)(1,1)(6,5)
%% 	\psset{xunit=5cm,yunit=2cm}
%% 	\fileplot[linewidth=1pt,linecolor=blue,origin={0.2,1.5}]{#2}
%% 	\psset{xunit=1cm,yunit=1cm}
%% 	\psaxes[Dx=#1,dx=5,Oy=-1,Dy=1,dy=2]{-}(0.9,1)(6,5)
%% 	\rput(4,0.4){\textsl{#3}}
%% \end{pspicture}\end{center}\end{minipage}}

%% \newcommand{\eqncode}[2]{%
%% \begin{center}
%% \begin{tabular}{l@{\hspace{0.5cm}}r}
%% \begin{minipage}{.4\textwidth}
%% \begin{equation*}
%% #1
%% \end{equation*}
%% \end{minipage}
%% &
%% \fbox{\begin{minipage}{.4\textwidth}
%% %\setlength{\parskip}{4mm}
%% \filecodigobis{#2}
%% \end{minipage}}
%% \end{tabular}
%% \end{center}
%% }

%% \newcommand{\eqncodeb}[2]{%
%% \begin{center}\begin{tabular}{l@{\hspace{0.5cm}}r}
%% \begin{minipage}{.4\textwidth}#1\end{minipage} &
%% \fbox{\begin{minipage}{.4\textwidth}\filecodigobis{#2}\end{minipage}}
%% \end{tabular}\end{center}}

%% \newenvironment{matemcode}[1]{\newline
%% \begin{tabular}{l@{\hspace{0.5cm}}r}
%% \begin{minipage}{.4\textwidth}
%% \parbox[t]{.4\textwidth}{\begin{equation*}#1\end{equation*}}\end{minipage}
%% &\begin{Sbox}\begin{minipage}{.4\textwidth}}
%% {\end{minipage}\end{Sbox}\fbox{\TheSbox}\end{tabular}\newline}

%% \newenvironment{encuadrar}[1]{\begin{Sbox}\begin{varwidth}{#1\textwidth}}
%% {\end{varwidth}\end{Sbox}\fbox{\TheSbox}}

%% \newenvironment{parboxenv}{\begin{Sbox}}
%% {\end{Sbox}\parbox[t]{.9\textwidth}{\TheSbox}}

% multicolumn y multirow
\newcommand{\mcol}[3]{\multicolumn{#1}{#2}{#3}}
\newcommand{\mrow}[3]{\multirow{#1}{#2}{#3}}

%% Serif .......................................................................
%%
%% New Century Schoolbook
%% \usepackage[T1]{fontenc}
%% \usepackage{fouriernc}
%%
%%
%% TeX Gyre Schola (New Century extendida)
%% \usepackage[T1]{fontenc}
%% \usepackage{tgschola}
%%
%%
%% Utopia
%% \usepackage[T1]{fontenc}
%% \usepackage{fourier}
%%
%%
%% Utopia (con MathDesign)
%% \usepackage[T1]{fontenc}
%% \usepackage[adobe-utopia]{mathdesign}
%%
%%
%% Computer Concrete
%% \usepackage[T1]{fontenc}
%% \usepackage{concmath}
%%
%%
%% Charter BT
 \usepackage[T1]{fontenc}
 \usepackage[bitstream-charter]{mathdesign}
%%
%%
%% Nimbus Roman (clon de Times)
%% \usepackage[T1]{fontenc}
%% \usepackage{nimbus}
%%
%%
%% TeX Gyre Termes (version mejorada de Nimbus Roman)
%% \usepackage[T1]{fontenc}
%% \usepackage{tgtermes}
%%
%%
%% GFS Bodoni
%% \usepackage[T1]{fontenc}
%% \usepackage[default]{gfsbodoni}
%%
%%
%% Baskervald ADF
%% \usepackage[T1]{fontenc}
%% \usepackage{baskervald}
%%
%%
%% Efont Serif -- descargar de http://openlab.jp/efont/serif/
%% \usepackage[T1]{fontenc}
%% \usepackage{efont,mathesf}
%% \renewcommand*\oldstylenums[1]{{\fontfamily{esfod}\selectfont#1}}
%%
%%
%%
%%
%%
%% Sans-Serif ..................................................................
%%
%%
%% Optima (clon de, URW Classico)
%% \usepackage[T1]{fontenc}
%% \renewcommand*\sfdefault{uop}
%%
%%
%% Avantgarde (clon de, URW Gothic)
%% \usepackage[T1]{fontenc}
%% \usepackage{avant}
%%
%%
%% TeX Gyre Adventor (version mejorada de Avantgarde)
%% \usepackage[T1]{fontenc}
%% \usepackage{tgadventor}
%%
%%
%% Nimbus Sans (clon de Helvetica)
%% \usepackage[T1]{fontenc}
%% \usepackage{nimbus}
%%
%%
%% Helvetica (clon de, Nimbus Sans)
%% \usepackage[T1]{fontenc}
%% \usepackage[scaled]{helvet}
%%
%%
%% TeX Gyre Heros (version mejorada de Nimbus Sans)
%% \usepackage[T1]{fontenc}
%% \usepackage{tgheros}
%%
%%
%% Boilinum
%% \usepackage[T1]{fontenc}
%% \usepackage{libertine}
%%
%%
%% Computer Modern Bright
%% \usepackage[T1]{fontenc}
%% \usepackage{cmbright}
%%
%%
%% Latin Modern Sans
%% \usepackage[T1]{fontenc}
%% \usepackage{lmodern}
%%
%%
%% Epigrafica
%% \usepackage[OT1]{fontenc}
%% \usepackage{epigrafica}
%%
%%
%%
%% Si quiero el documento en sans en vez de Roman:
%% \renewcommand*\familydefault{\sfdefault}
%% ...............................................
%% 
%%
%%
%% Monospaced ..................................................................
%%
%%
%% Pandora Typewriter
%% \usepackage[T1]{fontenc}
%% \usepackage{pandora}
%%
%%
%% Letter Gothic
%% \usepackage[T1]{fontenc}
%% \usepackage{ulgothic}
%%
%%
%% Inconsolata
%% \usepackage[T1]{fontenc}
%% \usepackage{inconsolata}
%%

%
\addbibresource{res/bibliografia.bib}
%
\selectlanguage{spanish}
\hyphenation{micro-RNA}
\hyphenation{micro-RNAs}
\hyphenation{mi-RNA}
\hyphenation{mi-RNAs}
%
%
%
%
\addtokomafont{descriptionlabel}{\small}
\setkomafont{subject}{\LARGE\usekomafont{disposition}}
\setkomafont{title}{\normalfont\slshape}
\setkomafont{subtitle}{\LARGE\usekomafont{disposition}}
%
\begin{document}
\selectlanguage{spanish}
%
% pagina de titulo
%
\titlehead{\center\large
    Universidad Nacional del Litoral\\
    Facultad de Ingeniería y Ciencias Hídricas
}
%
%
\title{\LARGE ``Desarrollo de un clasificador de secuencias de pre-microRNA
  mediante técnicas de Inteligencia Computacional''}
\subject{Proyecto Final de Carrera\\Ingeniería en
  Informática}
\subtitle{~\\[.2ex]Informe entregable 1\\[.2ex]~}
\author{{Alumno: Mauro Javier Torrez}\and{Director: Dr. Diego H. Milone}}
%
\date{~\\[2em]\today}
%
\renewcommand*{\titlepagestyle}{empty}
%\thispagestyle{empty}
\maketitle
\setcounter{page}{1}
%
%
%
%
\section{Introducción}
En este informe entregable se presenta una revisión de las tareas
llevadas a cabo por el alumno tal como se ha estipulado en la
Propuesta de Proyecto Final de Carrera.

El documento se divide en las secciones: Bibliografía consultada,
Bases de datos recopiladas, Codificación de clasificadores de prueba
y Armado de la base de datos definitiva.

En pos de evitar la incorporación del código fuente generado en el
documento, se refiere al lector al repositorio disponible en
\url{https://github.com/maurete/pfc/}, donde se puede consultar
el código fuente y demás archivos generados durante el desarrollo.
%
%
%
%
\section{Bibliografía consultada}
\label{bibliografia}
%
A continuación se enumera la bibliografía consultada, referente a la
perspectiva biológica de los pre-miRNA, las técnicas de clasificación
SVM y MLP, y diferentes implementaciones de clasificadores pre-miRNA
utilizando estas técnicas de Inteligencia Computacional. Se presenta
una breve reseña orientativa junto a cada publicación.
%
\subsection{D. P. Bartel \cite{bartel116}}
Este trabajo brinda una revisión de los desarrollos en la
investigación de los microRNAs y las conclusiones que surgen de éstos
acerca de la biogénesis y función de los miRNAs.  Ofrece una
explicación didáctica del proceso de maduración de un pre-miRNA en
miRNA.  Publicado en 2004, es anterior a la aparición de los métodos
computacionales no-comparativos que permitieron un gran aumento en el
número de nuevos miRNAs descubiertos.
%
\subsection{L. Li et al. \cite{lili}}
En este trabajo, publicado en 2010, se ofrece un panorama de diversos
métodos para la identificación de miRNAs junto a una revisión de las
características biológicas de los mismos. Se presentan métodos tanto
de tipo comparativo como aquellos basados en aprendizaje de máquina,
así como métodos que trabajan sobre los datos experimentales de
técnicas de secuenciamiento en gran escala o \eng{deep sequencing}.
También se presenta el estado del arte en la predicción de dianas
(\eng{targets}) y regulación de la expresión de los miRNAs.
%
\subsection{D. Yue et al. \cite{yue}}
Se presenta en este trabajo un resumen de las técnicas disponibles
para la predicción de dianas y de métodos de predicción de las
funciones de los miRNAs, con una exposición clara de los conceptos
relevantes en este campo de estudio.
%
\subsection{L. Bottou y C.-J. Lin \cite{bottou}}
Este trabajo presenta una descripción del problema general de
clasificación mediante SVM \cite{svm}, detallando el problema de
optimización a resolver y los parámetros a considerar en la
utilización e implementación de las máquinas de vectores de soporte.
%
\subsection{C. Xue et al. \cite{xue}}
%
En este trabajo se presenta un clasificador de secuencias de
pre-microRNA mediante SVM \cite{svm}.  En el clasificador se utiliza
un conjunto de características de tipo estructura-secuencia,
utilizando como entrada al clasificador un vector de frecuencia de 32
``triplets'' que combinan el nucleótido de la secuencia con la
estructura secundaria del entorno donde éste se presenta.

El método aquí descripto es notable en que obtiene una buena tasa de
clasificación utilizando una combinación simple de características de
la secuencia y la estructura secundaria.
%
\subsection{S. K. L. Ng y S. K. Mishra \cite{ng}}
%
En este trabajo se presenta un clasificador de secuencias de
pre-miRNA mediante SVM.

El conjunto de características utilizado consiste en 29 medidas
``globales e intrínsecas'' a la secuencia y su estructura secundaria:
(1) 17 medidas de composición de la secuencia: frecuencia de
ocurrencia de dinucleótidos
$\mono{XY}:(\mono{X},\mono{Y})\in\{\mono{A},\mono{G},\mono{C},\mono{U}\}$
y frecuencia agregada de ocurrencia de los nucleótidos \mono{G} y
\mono{C}; (2) 6 medidas de plegado basadas en la mínima energía libre
y la distribución de los pares de bases; (3) un descriptor
topológico; y (4) 5 variantes normalizadas de estas características.

El método además es probado para detectar pre-miRNAs en genes de
virus, recorriendo éste mediante una ventana deslizante de 95nt y
clasificando las secuencias extraídas.
%
\subsection{R. Batuwita y V. Palade \cite{batuwita}}
En esta publicación se describe un método de clasificación de
pre-miRNAs que elabora sobre \cite{ng} incorporando características
basadas en la mínima energía libre, otras características relacionadas
el programa RNAfold, y características basadas en los pares de bases.

Se discute acerca de la relevancia de cada una de las características
y se presenta luego un conjunto reducido de éstas para mejorar la
performance del clasificador, y también se discute acerca del problema
de \emph{desbalance de clases}, siempre presente en la
clasificación de pre-miRNAs.
%
\subsection{S. Sewer et al. \cite{sewer}}
Este trabajo describe un método de predicción de microRNAs en el
genoma homano, de rata y de ratón. El método tiene un enfoque
comparativo para la selección de las regiones candidatas del genoma,
así como un enfoque no-comparativo para la predicción de los microRNAs
en estas regiones. Para la clasificación se utilizan Máquinas de
Vector de Soporte.
%
\subsection{J. Hertel y P. F. Stadler \cite{hertel}}
Este trabajo presenta un método de clasificación de pre-miRNAs basado
en características de la secuencia y de la estructura secundaria, con
un clasificador basado en SVM. Como paso inicial se utiliza una
técnica de ``alineación múltiple'' de la secuencia, ajustando una
ventana para determinar la posición exacta del pre-miRNA candidato en
la región.  El resultado es una elevada especificidad del 99\% (muy
pocos falsos positivos), con una sensibilidad del 80\%.
%
\subsection{Y. Xu et al. \cite{xu}}
En este trabajo se presenta un método no-comparativo que en lugar de
utilizar un método de aprendizaje supervisado como SVM, utiliza un
algoritmo de \eng{ranking} basado en \eng{random walks}.  Este método
se caracteriza por no requerir de ejemplos negativos para el
entrenamiento. Finalmente el método es aplicado para la identificación
de nuevos pre-miRNAs en \eng{Anopheles gambiae}, un mosquito que es el
principal vector de la malaria en África.
%
\subsection{J. Ding et al. \cite{ding}}
Este trabajo presenta un método de clasificación mediante SVM con un
enfoque específicamente diseñado para resolver el problema de
desbalance de ejemplos positivos y negativos, y que a la vez no asume
características estructurales de los pre-miRNAs.

Se describe la utilización conjunta de diferentes clasificadores SVM
para reducir los problemas del desbalance de clases. Por otro lado
incorpora características de pre-miRNAs tipo multi-loop y realiza una
selección del conjunto de características finales a considerar
mediante la técnica \eng{F-score}, que permite seleccionar aquellas
características de entrada más ``relevantes'' para el clasificador.
%
%
%
%
\section{Bases de datos recopiladas}
%
Tomando como base los métodos de clasificación de pre-miRNAs
consultados en la bibliografía se procedió a revisar los respectivos
materiales suplementarios en búsqueda de bases de datos completas
sobre las cuales trabajar.  A continuación se describen las bases de
datos recopiladas a partir de los materiales suplementarios de
\cite{xue}, \cite{ng} y \cite{batuwita}.
%
\subsection{Triplet-SVM \cite{xue}}
%
Si bien en la página del
autor\footnote{\url{http://bioinfo.au.tsinghua.edu.cn/software/mirnasvm/}}
ya no están disponibles los datos, éstos se pueden encontrar en una
versión anterior de la misma, disponible en el archivo de
internet\footnote{\url{http://web.archive.org/web/20120210235054/http://bioinfo.au.tsinghua.edu.cn/software/mirnasvm/}}.
%
\subsubsection{Características disponibles}
En esta base de datos cada una de las entradas presenta las siguientes
características:
\begin{description}
  [style=sameline,leftmargin=3cm,itemsep=6pt]
%
\item[identificador] cadena de caracteres, p. ej. \mono{hsa-mir-100}.
%
\item[secuencia] cadena de caracteres \mono{A}, \mono{G}, \mono{C},
  \mono{U}.
%
\item[estructura secundaria] cadena de caracteres \mono{(}, \mono{.},
  \mono{)}. Calculada con el programa RNAfold \cite{vienna}.
%
\item[SEQ\_LENGTH] longitud de la secuencia considerada para el armado
  del vector de características. Cuenta aquellos nucleótidos que
  pertenecen al tallo del pre-miRNA.
%
\item[GC\_CONTENT] se define según
  $\frac{\T{cant}(\mono{G})+\T{cant}(\mono{C})}{\mono{SEQ\_LENGTH}}$,
  donde $\T{cant}(x)$ cuenta el número de ocurrencias del nucleótido
  $x$ en el tallo del pre-miRNA.
%
\item[BASEPAIR] cantidad de nucleótidos
  que forman un par de bases en el pre-miRNA.
%
\item[FREE\_ENERGY] mínima energía libre de plegado
  obtenida con RNAfold.
%
\item[LEN\_BP\_RATIO] se define según
  $\frac{\mono{SEQ\_LENGTH}}{\mono{BASEPAIR}}$.
\end{description}
%
El vector de 32 triplets utilizados en el clasificador no está
directamente disponible, si bien se pueden encontrar las rutinas para su
cálculo, en lenguaje Perl.
%
\subsubsection{Conjuntos de datos}
\paragraph{pre-miRNAs humanos}
\begin{description}[style=nextline,leftmargin=3cm,align=right]
\item[Tipo:] Conjunto de entrenamiento (163) y prueba (30), datos
  positivos
\item[Núm. entradas:] 193
\item[Especies:] Homo sapiens (hsa)
\item[Descripción:] Tomados de miRNA registry rel 5.0, sept/2004
\end{description}
\paragraph{CROSS-SPECIES}
\begin{description}[style=nextline,leftmargin=3cm,align=right]
\item[Tipo:] Conjunto de prueba, datos positivos
\item[Núm. entradas:] 581
\item[Especies:] mmu (36), rno (25), gga (13), dre (6), cbr (73), cel
  (110), dps (71), dme (71), osa (96), ath (75), ebv (5)
\item[Descripción:] Tomados de miRNA registry rel 5.0, sept/2004
\end{description}
%
\paragraph{pre-miRNAs humanos (actualizado)}
\begin{description}[style=nextline,leftmargin=3cm,align=right]
\item[Tipo:] Conjunto de prueba, datos positivos
\item[Núm. entradas:] 39
\item[Especies:]  Homo sapiens (hsa)
\item[Descripción:] Conjunto de 39 pre-miRNAs obtenido a partir de los
  89 reportados en \cite{bentwich}, eliminando entradas redundantes y
  aquellas similares a las encontradas en el conjunto de
  entrenamiento.
\end{description}
%
\paragraph{CONSERVED-HAIRPIN}
\begin{description}[style=nextline,leftmargin=3cm,align=right]
\item[Tipo:] Conjunto de prueba, datos mayormente negativos (algunos
  positivos)
\item[Núm. entradas:] 2444
\item[Especies:]  Homo sapiens (hsa)
\item[Descripción:] Se obtiene mediante aplicación de una ventana
  deslizante de 100nt con paso 10nt en la región 56000001--57000000
  del cromosoma humano 19 de UCSC database. La gran mayoría de
  entradas en este conjunto son pseudo-pre-miRNAs, aunque contiene
  algunos pre-miRNAs reales.  Por esto se deberá usar únicamente como
  conjunto de prueba.
\end{description}
%
%
%
%% \begin{tabular}{@{}p{2.8cm}p{1.2cm}p{1.6cm}p{8.6cm}@{}}
%% \toprule
%% Tipo & Elems. & Especies & Descripción \\ \midrule
%% \mcol{4}{l}{CODING} \\ \midrule[.2pt]
%% Entrenamiento, datos negativos &
%% 8494 &
%% Homo sapiens (hsa) &
%%  Entradas obtenidas de las regiones codificantes
%%   (CDS) de los genes humanos RefSeq disponibles en en la base de datos
%%   USC.  Para el armado de este conjunto se consideran los siguientes
%%   criterios:
%%   \begin{itemize}
%%   \item Mínimo de 18 pares de bases en el tallo
%%   \item Máximo -15kcal/mol de energía libre mínima
%%   \item Ningún bucle múltiple (no más de un ``tallo'' en la estructura
%%     secundaria)
%%   \end{itemize}
%% \\ \bottomrule
%% \end{tabular}
%
%
%
%
%
\newpage
\subsection{miPred \cite{ng}}
Los conjuntos de datos de este trabajo están disponibles en la página
del autor\footnote{\url{http://web.bii.a-star.edu.sg/archive/stanley/%
Publications/Supp_materials/06-002-supp.html}}.
%
\subsubsection{Características disponibles}
\begin{description}%
  [style=sameline,leftmargin=3cm,itemsep=4pt]
%
\item[identificador] cadena de caracteres, p. ej. \mono{hsa-let-7a}.
%
\item[secuencia] cadena de caracteres \mono{A}, \mono{G}, \mono{C},
  \mono{U}.
%
\item[estructura secundaria] cadena de caracteres \mono{(}, \mono{.},
  \mono{)}.
%
\item[longitud] de la secuencia completa.
%
\item[A, G, C, U] (x4) número de ocurrencias del nucleótido en la
  secuencia.
%
\item[G+C, A+U] (x2) suma de las ocurrencias de \mono{G} y \mono{C}, y
  \mono{A} y \mono{U} respectivamente.
%
\item[AA, AG, AC, \textellipsis, UU] (x16) número de ocurrencias del
  dinucleótido correspondiente en la secuencia.
%
\item[\%A, \%G, \%C, \%U] (x4) frecuencia de ocurrencia del nucleótido
  correspondiente.
%
\item[\%(G+C), \%(A+U)] (x2) frecuencia de ocurrencia de
  \mono{G}+\mono{C} y \mono{A}+\mono{U}.
%
\item[\%AA, \%AG, \%AC, \textellipsis] (x16) frecuencia de ocurrencia
  de los dinucleótidos correspondientes.
%
\item[pb] cantidad de pares de bases.
%
\item[mfe] mínima energía libre según RNAfold.
%
\item[Q] \emph{entropía de Shannon}\footnote{La entropía de Shannon
  es una medida de aleatoriedad utilizada en Teoría de la
  Información. En forma simple, podemos decir que muestra en cuánto se
  diferencia una secuencia determinada de otra completamente
  aleatoria, y de esta manera brinda una idea de la cantidad de
  información contenida en la misma.} de la secuencia.
%
\item[D] distancia entre pares de bases.
%
\item[Npb, Nmfe, NQ, ND] (x4) valores normalizados de \mono{pb},
  \mono{mfe}, \mono{Q}, \mono{D} por la longitud de la secuencia.
%
\end{description}
%
%
\subsubsection{Conjuntos de datos}
\paragraph{pre-miRNAs no redundantes}
\begin{description}[style=nextline,leftmargin=3cm,align=right]
\item[Tipo:] Conjunto de datos positivos
\item[Núm. entradas:] 2241
\item[Especies:] 45 especies agrupadas en artropoda, nematoda,
  vertebrata, viridiplantae y virus
\item[Descripción:] Tomados de miRBase 8.2 (jul/2006) para todas las
  especies disponibles, se filtran aquellas secuencias redundantes
  mediante un algoritmo de agrupación incremental
  ``ávido''\cite{greedy}.
\end{description}

\paragraph{ncRNAs funcionales}
\begin{description}[style=nextline,leftmargin=3cm,align=right]
\item[Tipo:] Conjunto de datos negativos
\item[Núm. entradas:] 12387
\item[Especies:] muchas (prokaryota y eukaryota)
\item[Descripción:] Tomados de Rfam 7.0, consiste en un conjunto de
  ncRNAs de los que se han eliminado 46 tipos de pre-miRNAs.
\end{description}

\paragraph{mRNAs}
\begin{description}[style=nextline,leftmargin=3cm,align=right]
\item[Tipo:] Conjunto de datos negativos
\item[Núm. entradas:] 31
\item[Especies:] varias, entre ellas: perro, hsa, mmu, cel, rno,
  \textellipsis.
\item[Descripción:] Consiste en 31 mRNAs mensajeros que se pliegan en
  estructuras complejas con MFEs extremadamente negativas. Tomados de
  GeneBank DNA Database.
\end{description}

\paragraph{pseudo-hairpins}
\begin{description}[style=nextline,leftmargin=3cm,align=right]
\item[Tipo:] Conjunto de datos negativos
\item[Núm. entradas:] 8494
\item[Especies:] homo sapiens (hsa)
\item[Descripción:] Ídem al dataset CODING de triplet-svm, en este
  caso con más características extraídas.
\end{description}
%
%
%
%
%
\subsection{microPred \cite{batuwita}}
En esta base de datos están presentes las características presentadas
en \cite{ng} (salvo estructura secundaria) y se introducen otras 19 
características adicionales. De estas 19 se detallan sólo aquellas que están
basadas en la secuencia y/o estructura secundaria, el resto de características
se basa en descriptores moleculares, y dada su complejidad de cálculo
se consideran fuera del alcance del actual proyecto.
%
Los datos se encuentran disponibles en el sitio web de los autores%
\footnote{\url{http://www.cs.ox.ac.uk/people/manohara.rukshan.batuwita/microPred.htm}}.
%
\newpage
\subsubsection{Características disponibles}
\begin{description}
  [style=sameline,leftmargin=3cm,itemsep=4pt]
%
\item[identificador] cadena de caracteres, p. ej. \mono{hsa-let-7a}.
%
\item[secuencia] cadena de caracteres \mono{A}, \mono{G}, \mono{C},
  \mono{U}.
%
\item[MFEI$_3$] \mono{Nmfe} dividida entre el número de loops (bucles)
  de la estructura secundaria.
%
\item[MFEI$_4$] \mono{mfe} dividida entre el número de pares de bases.
%
\item[|A-U|/L, |G-C|/L, |G-U|/L] (x3) número de pares de bases
  \mono{A-U}, \mono{G-C} y \mono{G-U} respectivamente, normalizados
  por la longitud de la secuencia.
%
\item[Avg\_BP\_Stem] número de pares de bases dividido entre el
  número total de tallos en la estructura secundaria.
%
\item[\%(A-U)/n\_stems, \%(G-C)/n\_stems, \%(G-U)/n\_stems] (x3) con
  \mono{\%(X-Y)} la frecuencia del par de bases \mono{X-Y}, dividido
  por el número de tallos en la estructura secundaria.
\end{description}
%
%
\subsubsection{Conjuntos de datos}
\paragraph{pre-miRNAs humanos}
\begin{description}[style=nextline,leftmargin=3cm,align=right]
\item[Tipo:] Conjunto de datos positivos
\item[Núm. entradas:] 691
\item[Especies:] Homo sapiens (hsa)
\item[Descripción:]Tomados de miRBase 12. Contiene 660 pre-miRNAs
  single-loop y 31 multi-loop.
\end{description}
%
\paragraph{pseudo-hairpins}
\begin{description}[style=nextline,leftmargin=3cm,align=right]
\item[Tipo:] Conjunto de datos negativos
\item[Núm. entradas:] 8494
\item[Especies:] Homo sapiens (hsa)
\item[Descripción:] Ídem dataset CODING de Xue, con más
  características extraídas.
\end{description}
%
\paragraph{Otros ncRNAs humanos}
\begin{description}[style=nextline,leftmargin=3cm,align=right]
\item[Tipo:] Conjunto de datos negativos
\item[Núm. entradas:] 754
\item[Especies:] Homo sapiens (hsa)
\item[Descripción:] Conjunto de datos compilado manualmente por los
  autores. Contiene ncRNAs humanos que no son pre-miRNAs.
\end{description}
%
%
%
%
\section{Codificación de clasificadores de prueba}
\label{pruebas}
%
Una vez obtenidas las bases de datos se procedió a codificar
algoritmos de clasificación preliminares.
Se optó por trabajar con el conjunto de datos de \cite{xue},
intentando replicar los resultados obtenidos en ese trabajo mediante
la utilización de herramientas propias. Según se ha observado en la
bibliografía, este trabajo es tomado como modelo de referencia debido a
la simplicidad del método y los buenos resultados que obtiene.

Al analizar el formato de los datos disponibles surge la necesidad
de codificar una herramienta para el manejo de archivos en formato
FASTA\footnote{FASTA es un formato de texto para la representación
  de secuencias de nucleótidos, como es el RNA.
  La especificación se encuentra disponible en
  \url{http://blast.ncbi.nlm.nih.gov/blastcgihelp.shtml}.}.
El formato de salida del programa \mono{RNAFold} es además
muy similar a FASTA, e incorpora la información de la estructura
secuendaria o \emph{plegado} para cada entrada.
%
Se ha codificado la herramienta \mono{fautil.py} en lenguaje Python,
que permite manipular archivos con formato FASTA/RNAfold. También para
Matlab se ha codificado la utilidad \mono{fastaread.m} para lectura
de estos archivos.
%
\subsection{Extracción de características}
%
Tal como en \cite{xue}, se definen 32 elementos ``triplete'', que
relacionan en cada posición la secuencia con la estructura secundaria
en el entorno. Para cada entrada de la base de datos, se calcula la
frecuencia de ocurrencia de cada triplete, generando así un 32-vector
que se utiliza como entrada al clasificador.

%% Tal como en \cite{xue}, se define un ``triplete'' las características
%% se calculan para cada ``triplete'' (\emph{triplet}) en la secuencia,
%% relacionando secuencia y estructura secundaria. Para cada entrada, el
%% vector de características representa la frecuencia de ocurrencia de
%% las 32 configuraciones posibles al combinar el nucleótido central del
%% triplete con la estructura secundaria del mismo.

El proceso detallado se puede describir como sigue:
\begin{enumerate}
\item A partir de la estructura secundaria, se determina una región a
  considerar para los cálculos, de forma que excluya los extremos
  ``sueltos'' así como el bucle central de la cadena, donde las bases
  no forman pares.
\item Se recorre en esta región tanto la secuencia como la
  estructura secundaria con una ventana de 3nt y paso 1nt. En cada
  paso:
  \begin{enumerate}
  \item Se calcula una cadena de ``elemento triplet'' añadiendo al
    caracter central en la secuencia los tres caracteres
    correspondientes a la estructura secundaria (se ignora la
    orientación en el caso de bases que forman un par). Se obtiene
    así, por ejemplo, el elemento ``\mono{A.((}''.
  \item Se incrementa el contador de ocurrencias para el elemento en
    la posición correspondiente del vector.
  \item En los extremos se ignora un caracter del triplet, que se
    considera en la estructura secundaria como un ``\mono{.}''.
  \end{enumerate}
\item Se normaliza el vector de frecuencias dividiéndolo por el número
  total de triplets contados.
\end{enumerate}

En el archivo \mono{fautil.py}, rutina \mono{triplet}, se encuentra
una implementación en Python de este proceso. También se ha codificado
el mismo algoritmo en Matlab, \mono{triplet.m}.
%
\subsection{Clasificadores}
Se generaron 3 clasificadores, según se detallan a continuación.
\begin{enumerate}
\item Clasificador mediante SVM utilizando \emph{libsvm}\cite{libsvm} con los
  parámetros por defecto.
\item Clasificador mediante SVM utilizando el \emph{Bioinformatics
  Toolbox} de Matlab.
\item Clasificador mediante MLP utilizando el \emph{Neural Network
  Toolbox} de Matlab.
\end{enumerate}
%
La librería \emph{libsvm}\cite{libsvm} provee una utilidad \mono{svm-easy} que
busca en forma automática aquellos parámetros que minimizan el
error. En los otros casos, esta búsqueda se realizó en forma manual,
ajustando los parámetros \mono{boxconstraint} y y \mono{sigma} para
SVM y para MLP ajustando el número de capas ocultas y el número de
nodos por capa oculta.
%
\subsection{Pruebas}
Los clasificadores se entrenaron con el mismo conjunto de datos de
\cite{xue}, y se probaron con los conjuntos de prueba \emph{Real} (30
elementos), \emph{Pseudo} (1000 elementos) y \emph{Updated} (39
elementos), también de la misma fuente.

En el caso de SVM-Matlab y MLP, se probaron diversos parámetros hasta
encontrar aquellos que obtienen el mejor rendimiento del clasificador.
En la tabla \ref{testresults} se presentan figuras de rendimiento para
los distintos clasificadores. %En el caso de SVM-Matlab se muestran
%tres resultados representativos correspondientes a diferentes
%parámetros. 
En el caso MLP, se encontró que los resultados no varían
significativamente para distintas configuraciones, con diferencias
menores a 3\% en cada caso.  Se muestran entonces para MLP tasas
representativas.
%
\begin{table}
  \caption{Resultados de las pruebas iniciales de clasificación}
  \center%\sffamily
  \begin{tabular}{lrrr}\toprule
    Clasificador  & \% Real (/30) &
                     \% Pseudo (/1000) & \% Updated (/39) \\\midrule
    Triplet-SVM   &  $93.3$    & $88.1$     & $92.3$     \\
    SVM (libSVM)  & $100.0$    & $87.4$     & $92.3$     \\
    SVM (Matlab)  & $100.0$    & $91.2$     & $89.7$     \\
    %% svm (matlab,2) & $96.7$     & $96.6$     & $48.7$     \\
    %% svm (matlab,3) & $100.0$    & $86.4$     & $97.4$     \\
    MLP           &  $90.0$    & $85.0$     & $91.0$  \\\bottomrule
  \end{tabular}
  \label{testresults}
\end{table}
%

Como se puede observar en la tabla \ref{testresults}, las tasas de
clasificación para los distintos conjuntos resultan satisfactorias, e
incluso sensiblemente mejores a aquellas del trabajo original.  Sin
embargo, estos números deberán tomarse con cuidado, ya que han sido
obtenidos entrenando y validando con particiones estáticas, y podrían
ser resultado de un sobreentrenamiento para estos datos en
particular. Se observa también que el perceptrón multicapa presenta
una buena tasa de clasificación incluso cuando se trata de una única
neurona (sin capas ocultas), siempre para este mismo conjunto de
datos.

Se ha implementado el script \mono{triplet\_libsvm.sh} para las
pruebas con \emph{libsvm}, y los scripts \mono{triplet\_svm.m} y
\mono{triplet\_mlp.m} para las pruebas en Matlab de los clasificadores
SVM y MLP respectivamente. Con el software apropiado, estos scripts
se pueden ejecutar directamente para reproducir los resultados de las
pruebas.
%
%
%
%
%
\section{Armado de la base de datos definitiva}
Para el armado de la base de datos definitiva se tomaron como fuente
los conjuntos de datos y características utilizados en \cite{xue},
\cite{ng} y \cite{batuwita}. Se incorporó al conjunto de datos
la última versión disponible de miRBase \cite{mirbase2}.

Como primer paso se procedió a validar el plegado de las secuencias en
las bases de datos originales aplicando el programa RNAfold sobre las
mismas y comprobando que el plegado obtenido fuera el mismo que el
presente en la base de datos original. Se utilizó la versión 1.8.5 de
RNAfold, ya que con la versión actual (2.1) se obtienen plegados
diferentes en la mayoría de los casos.

Una vez validadas las estructuras secundarias se procedió a filtrar
aquellas entradas con bucles múltiples en la estructura secundaria, ya
que las características de triplete no están definidas en estos casos.
De esta manera se eliminó la base de datos ``mRNA'' de \cite{ng} por
completo, ya que todas las entradas en este caso poseen bucles
múltiples.

Luego se procedió a la ectracción y validación de características
extraídas contra las bases de datos respectivas: las características
de triplete y triplete-extra contra la base de datos de \cite{xue},
las características de la secuencia contra la base de \cite{ng}, y las
de estructura secundaria con la base de datos en \cite{batuwita}.

Además de las características calculadas, se incorporan a la base de
datos los datos de la secuencia y estructura secundaria en formato
compatible RNAfold, junto con otro archivo donde se indica la clase
(pre-miRNA real, pseudo-pre-miRNA o indefinido) de cada entrada
correspondiente.  En el archivo \mono{README.md} de la base de datos
se detalla la estructura de directorios generada junto con el formato
de los archivos para cada caso.

Se codificó la herramienta \mono{feats.py} a partir de la utilidad
\mono{fautil.py}, para la extracción de características de archivos
con formato RNAfold. Se ignoraron aquellas características que no se
pueden calcular directamente con la información de la secuencia y de
la estructura secundaria.

Se codificó la herramienta \mono{tests.py} para la validación, así
como los scripts en Bash \mono{validate.sh} y \mono{generate\_db.sh},
los que pueden ser ejecutados directamente, siempre con el software
requerido, para validar y regenerar la misma base de datos a partir de
las fuentes.
%
\subsection{Características extraídas}
%
A continuación se enumeran las características disponibles en la base
de datos definitiva.
%
\subsubsection{Características de tripletes}
Vector de 32 elementos de tripletes, calculado según \cite{xue}.
Considerando la región del tallo, contiene el número de ocurrencias
del elemento triplete correspondiente, normalizado entre el número
total de ocurrencias.  El orden es el siguiente: \mono{A..., A..(,
  A.(., A.((, A(.., A(.(, A((., A(((, G..., G..(, G.(., G.((, G(..,
  G(.(, G((., G(((, C..., C..(, C.(., C.((, C(.., C(.(, C((., C(((,
  U..., U..(, U.(., U.((, U(.., U(.(, U((., U(((}.
%
\subsubsection{Características de tripletes ``extra''}
Características auxiliares que se obtienen
al calcular el vector de tripletes.
\begin{description}[style=sameline,leftmargin=3cm]
\item[length3] longitud de la secuencia considerada al
  extraer los elementos de triplete (número de bases que
  conforman el tallo de la estructura de horquilla).
\item[basepairs] número de pares de bases en
  el pre-miRNA.
\item[{length3/basepairs}] grado de complementariedad
  entre los dos brazos de la estructura de horquilla. Para una
  complementariedad perfecta se da el valor mínimo de 2, aumentando cuanto más
  bases ``sueltas'' contenga el tallo.
\item[gc\_content] $=(\T{cant(\mono{G})}+\T{cant(\mono{C})})/\mono{length3}$. Proporción de nucleótidos \mono{G} y
  \mono{C} en el tallo.
\end{description}
%
\subsubsection{Medidas de la secuencia}
%Este grupo contiene las siguientes medidas de la secuencia del pre-miRNA:
\begin{description}[style=sameline,leftmargin=3cm]
\item[length] longitud del pre-miRNA, incluyendo extremos sueltos y el
  bucle central.
\item[A, C, G, U] (x4) número de nucleótidos \mono{A}, \mono{C},
  \mono{G} y \mono{U}, respectivamente.
\item[G+C, A+U] (x2) número de nucleótidos \mono{G} y \mono{C}, y
  \mono{A} y \mono{U} respectivamente.
\item[XY] (x16) número de dinucleótidos (dos nucleótidos contiguos)
  \mono{XY}, con $\mono{X,Y}\in\{\mono{A,C,G,U}\}$. El orden es el
  siguiente: \mono{AC, AG, AU, CA, CC, CG, CU, GA, GC, GG, GU, UA, UC,
    UG, UU}.
\end{description}
%
\subsubsection{Medidas de la estructura secundaria}
%En este grupo se presentan las siguientes características relativas al
%plegado del pre-miRNA en su estructura de horquilla:
\begin{description}[style=sameline,leftmargin=3cm]
\item[MFE] Mínima Energía Libre obtenida al plegar la secuencia con
  \mono{RNAfold}.
\item[MFEI1] $={\mono{MFE}}/{(\mono{G+C})\cdot 100}$, con \mono{G+C}
  de las medidas de secuencia.
\item[MFEI4] $=\mono{MFE}/\mono{basepairs}$ con \mono{basepairs} de
  las características de triplete extra.
\item[dP] \mono{basepairs} normalizada con la longitud total
  \mono{length}.
\item[|A-U|/length] número de pares \mono{A-U} normalizado.
\item[|G-C|/length] número de pares \mono{G-C} normalizado.
\item[|G-U|/length] número de pares \mono{G-U} normalizado.
\end{description}
%
\subsection{Conjuntos de datos}
A continuación se listan los conjuntos de datos obtenidos en el armado
de la base de datos.
\begin{description}
\item[mirbase50] de \cite{xue}, contiene 1210 pre-miRNAs reales para
  diferentes especies, entre ellas 193 humanos, 112 de gallina, 207 de
  ratón, 172 de rata y 96 de arroz.
\item[updated] de \cite{xue}, contiene 39 pre-miRNAs humanos.
\item[coding] de \cite{xue}, contiene 8494 pseudo pre-miRNAs humanos.
\item[conserved-hairpin] de \cite{xue}, contiene 2444 pseudo
  pre-miRNAs humanos, aunque se conoce que algunos de ellos son
  pre-miRNAs reales. Se establece la clase para estas entradas como
  indeterminada, representada por el valor ambiguo 0.
\item[mirbase82-nr] de \cite{ng}, contiene 1985 pre-miRNAs de miRBase
  8.2 filtrados a 90\% de similaridad, para 40 especies incluyendo
  vertebrados, plantas y virus.
\item[functional-ncrna] de \cite{ng}, contiene 2657 ncRNAs
  funcionales, excepto pre-miRNAs, para varias
  especies.\item[mirbase12] de \cite{batuwita}, contiene 660
  pre-miRNAs humanos no redundantes originalmente de miRBase 12.0.
\item[other-ncrna] de \cite{batuwita}, contiene 129 ncRNAs humanos que
  no son pre-miRNAs.
\item[mirbase20] de \cite{mirbase2}, incorpora 21433 pre-miRNAs reales
  que no tienen bucles múltiples de miRBase 20, para 204 especies,
  entre ellos 1801 humanos, 1121 de ratón, 423 de rata y 392 de arroz.
\end{description}
%
%
\renewcommand{\bibfont}{\normalfont\footnotesize}
\printbibliography
%
\end{document}



%% Batuwita et al.: microPred
%% 1. datos disponibles sin plegar, pero con las features calculadas
%% 2. dataset positivo: 695 pre-miRNAs hsa (mirBase 12)
%% 3. datasets negativos:
%% 1. CODING de Xue (8494)
%% 2. otros ncRNAs hsa: 754 (695 con secstruct multi-branched)
%% 1. features (48):
%% 1. 29 idem miPred
%% 2. 2 MFE-related
%% 3. 4 RNAfold-related
%% 4. 6 Mfold-related
%% 5. 7 calculadas con scripts propios



%% Sewer et al.: Mir-abela
%% 1. extrae candidatos de miRNAs usando sliding windows en las regiones del genoma que se sabe hay miRNAs
%% 2. especies: human, mouse, rat
%% 3. algunas features “calculables”:
%% 1. energía
%% 2. long del stem simple más largo
%% 3. long del loop del hairpin
%% 4. proporción de nt A/C/G/U en el stem
%% 5. proporción de pares A-U/C-G/G-U en el stem
%% 1. no hay datos de entrenamiento





%% Hertel & Stadler: RNAmicro
%% 1. no hay datos
%% 2. features sacables:
%% 1. long stem
%% 2. long loop
%% 3. G+C

%% Helvik et al.: Microprocessor SVM
%% 1. no hay datos
%% 2. valida con 332 miRNAs hsa (miRBase 8.0) + 130 miRNAs (miRBase 8.1)
%% 3. plega con RNAfold default
%% 4. muchas features

%% Yousef et al.: BayesmiRNAfind
%% 1. no hay datos (dice que estan como supplementary data pero en Bioinformatics no están, tampoco en el sitio bioinfo.wistar.upenn.edu)
%% 2. dataset positivo: no queda claro de dónde lo sacó (pasando un sliding window por las regiones candidatas? de mirbase?)
%% 3. dataset negativo: 190739 no-miRNAs sacados aleatoriamente de las 3’UTR de mRNAs humanos.

%% Nam et al: ProMiR
%% 1. no hay datos, solo disponible el dataset positivo, sin plegar, en http://rfam.sanger.ac.uk
%% 2. no usa “features”, sino estados de transición en la secuencia para entrenar un clasificador HMM
%% Jiang et al.: MiPred
%% 1. dataset positivo: mirna registry database, release 8.2
%% 2. dataset negativo: CODING de Xue
%% 3. features(34): Xue + MFE + P-value
%% 4. disponible: 163 hsa (+), 168 random (-), sin plegar, sin características extraídas. Aparentemente idem Xue.



%% Huang et al.: MiRFinder
%% 1. datos no tan disponibles (train sin etiquetar, sólo vectores libsvm, test sólo secuencias, sin features) 
%% 2. features (18):
%% 1. 1: Minimum Free Energy
%% 2. 2: The difference of the MFE of the sequence pair
%% 3. 3: The difference of the structure of the sequence pair
%% 4. 4–7: Base pairing and other properties of the 22 mer hypothesized mature miRNA
%% 5. 8: The mutation frequency of the sequence segment pair
%% 6. 9–18: The frequency of the 10 possible secondary structure elements (combinations of 2 adjacent characters) in the pseudo code of stem region (represented by the new syntax)
%% 1. training (+): vectores libSVM sin etiquetar: mirBase 8.2 human, mouse, pig, cattle, dog, sheep
%% 2. training (-): sequence segments extracted from UCSC genome pair-wise alignments (human, mouse)
%% Lim et al.: mirCheck/mirScan
%% 1. no hay datos
%% 2. especies: C. elegans
%% 3. features:
%% 1. base pairing of the miRNA portion of the fold-back
%% 2. base pairing of the rest of the fold-back
%% 3. stringent sequence conservation in the 5Ј half of the miRNA
%% 4. slightly less stringent sequence conservation in the 3Ј half of the miRNA
%% 5.  sequence biases in the first five bases of the miRNA (especially a U at the first position)
%% 6. a tendency toward having symmetric rather than asymmetric internal loops and bulges in the miRNA region
%% 7. and the presence of two to nine consensus base pairs between the miRNA and the terminal loop region, with a preference for 4–6 bp.
%% Gkirtzou et al: MatureBayes
%% 1. datos no disponibles
%% 2. conjunto train:
%% 1. 533 hsa, 422 rno de mirBase 10.0
%% 1. test:
%% 1. entradas nuevas de hsa, rno en mirBase 11.0-14.0
%% 2. especies dme, zebrafish en mirBase 14
%% 1. no se usan features útiles para un clasificador svm
%% Lai et al.: MiRSeeker
%% 1. no hay datos
%% 2. scope del paper distinto al nuestro
%% Ding et al.: MiRenSVM
%% 1. training:
%% 1. (+) 692 hsa, 52 aga de mirBase 12.0
%% 2. (-) 9225 hsa, 92 aga de UTRdb release 22.0 (long 70-150nt)
%% 3. (-) 754 ncRNAs hsa usados en microPred
%% 4. (-) 256 ncRNAs aga de Rfam 9.1 (<150nt)
%% 1. test:
%% 1. (+) 14 hsa, 14 aga nuevos en mirBase 13.0
%% 2. (+) 5328 pre-miRNAs de otras 27 especies
%% 3. (-) ??? aparentemente parte de los de UTRdb en train (usa sólo 5428 para entrenar)
%% 1. datos no disponibles, aunque se pueden armar los datasets a mano (secuencias disponibles, hace falta plegar, armar conjuntos train/test)
%% 2. features de miPred + features de microPred
%% miPredGA
%% 1. train:
%% 1. (+) idem microPred
%% 2. (-) idem microPred (CODING de Xue + 754 ncRNAs)
%% 1. test:
%% 1. separa del dataset de train
%% 1. features:
%% 1. idem microPred, rankeadas según libSVM
%% 1. datos disponibles de microPred. no aporta nada nuevo en lo que hace los datos. 
%% Sheng et al.: mirCos
%% 1. no sirve
%% Xu et al.: miRank
%% 1. features:
%% 1. 1 MFE normalizada
%% 2. 2 base pairing propensity normalizada (1para c/brazo)
%% 3. 1 long del loop normalizada
%% 4. 32 triplets como Xue
%% 1. train:
%% 1. (+) 533 hsa, 38 aga de mirBase 1/9/2007
%% 2. (-) 1000 hsa, 20000+ aga obtenidos escaneando el genoma con ventanas de 90nt y filtrando por caracts. de plegado
%% 1. test: no especificado
%% 2. datos disponibles: sólo hsa, formato svm, sin info de secuencia/estructura secundaria

