\documentclass[12pt,bibliography=oldstyle,DIV=12,parskip=half-,titlepage]{scrartcl}
% esto me setea la variable pdf dependiendo del valor de \pdfoutput, que es >0
% sólo cuando estoy usando pdflatex para compilar el documento
%% \newif\ifpdf
%% \ifnum\pdfoutput<0
%% \pdffalse\fi
%% \ifnum\pdfoutput=0
%% \pdffalse\fi
%% \ifnum\pdfoutput>0
%% \pdftrue\fi


%
% ===
% === Trick para detectar si el documento está siendo compilado con pdflatex
% ===
%
% Esto me setea la variable pdf dependiendo del valor de \pdfoutput, que es >0
% sólo cuando estoy usando pdflatex para compilar el documento. Con esto puedo
% hacer  \ifpdf {...} \fi, que se ejecuta colo cuando compilo con pdflatex.
%% \newif\ifpdf
%% \ifnum\pdfoutput<0
%% \pdffalse\fi
%% \ifnum\pdfoutput=0
%% \pdffalse\fi
%% \ifnum\pdfoutput>0
%% \pdftrue\fi
%
% ===
% === I18n / L10n
% ===
%
% babel me da separación de sílabas para palabras en el idioma que le paso como
%       argumento opcional.
\usepackage[spanish,es-tabla,english]{babel}
%
% inputenc define la codificación de caracteres del código fuente, acá utf8.
\usepackage[utf8]{inputenc}
%
% ===
% === Gráficos
% ===
% 
% pst-pdf me permite usar PSTricks con pdflatex. Necesito cargarlo sólo si está
%         definida la variable pdf, por eso está entre \ifpdf ... \fi
%\ifpdf\usepackage{pst-pdf}\fi
%
% color me permite usar colores en el documento.
\usepackage{color}
%
% graphicx me da el comando \includegraphics para insertar imágenes (?)
\usepackage{graphicx}
%
% pstricks es un conjunto de macros basadas en PostScript para TeX, en
%          castellano: me da un entorno pstricks y comandos que uso dentro de
%          éste, que me sirven para dibujar figuras/diagramas/etc de manera
%          relativamente simple.
%\usepackage{pstricks}
%
% pst-circ me da macros para pstricks que me dibujan elementos de circuitos
%\usepackage{pst-circ}
%
% pst-plot me provee de funciones de ploteo para pstricks
%\usepackage{pst-plot}
%
% pst-2dplot me sirve para plotear en pstricks, entorno pstaxes
%\usepackage{pst-2dplot}
%
% ===
% === Verbatims
% ===
%
% verbatim es una reimplementación de los entornos verbatim[*]
%          provee el comando \verbatiminput{archivo} y el entorno comment, que
%          hace que LaTeX ignore directamente todo lo que está adentro
%\usepackage{verbatim}
%
% moreverb implementa el entorno verbatimtab indentando los tabs que encuentre,
%          y también el entorno listing, que pone números de línea al verbatim.
%          Para cambiar el ancho de la tabulacion, uso
%          \renewcommand\verbatimtabsize{<ancho del tab>\relax}
%          También define el entorno boxedverbatim.
%\usepackage{moreverb}
%
% listings me da el entorno lstlisting con resaltado de sintaxis.
%          Para setear el lenguaje del código, hago \lstset{language=<lang>}
%\usepackage{listings}
%
% url es un verbatim para escribir URL's que permite linebreaks dentro de ésta.
%     para usarlo, \url{<URL>}
\usepackage{url}
%
% ===
% === Más packages
% ===
%
%% \usepackage{mdwlist}		%Para listas mas compactas
%% \usepackage{textcomp}		%Para algunos símbolos
%% \usepackage{colortbl}		%Para celdas de colores en tablas
%% \usepackage{fancyhdr}		%Para encabezados/pie
\usepackage{bbold}		%Fuente bb para modo math: \mathbb{R} = reales
\usepackage{dsfont}		%Fuente ds para modo math: \mathds{R} = reales
\usepackage{multirow}		%Para "combinar" celdas en tablas
\usepackage{float}		%Para mejorar cuadros, figuras, etc
%% \usepackage{fancybox}		%Para recuardos de texto con bordes "fancy"
%% \usepackage{dingbat}		%Para dingbats
%\usepackage{marginal}		%Para  notas al margen que no puedo hacer andar
\usepackage{amsmath}		%Para enornos matemáticos mas flexibles
%\usepackage{varwidth}		%varwidth es un minipage que se ajusta al ancho mínimo


\usepackage[backend=biber,sorting=none,style=ieee,eprint=false,url=false]{biblatex} %% style=ieee
%% requiere texlive-bibtex-extra en debian


\usepackage{enumitem}
\setlist{noitemsep}
%% \setlist[description]{noitemsep}
%% \setlist[enumerate]{noitemsep}
%% \setlist[itemize]{noitemsep}

\usepackage{tikz}
\usepackage{pgfkeys}
\usepackage{pgfgantt}

% typearea: uso con koma-script para ajustar márgenes de página.
% vars globales a setear en la clase koma-script: DIV=12, BCOR=margen de ``binding'' para double side
\usepackage{typearea}

% para poder usar footnotes p.ej, adentro de un tabular
\usepackage{footnote}
\makesavenoteenv{tabular}

% para tabulars mas lindos/legibles
\usepackage{booktabs}

%\usepackage{glossaries}

\usepackage[spanish]{algorithm2e}

% para highlight (comando \hl{})
\usepackage{soulutf8}

% para teoremans etc
\usepackage{amsthm}

% para tunear citations
%\usepackage[square,comma,numbers,sort&compress]{natbib}


% config.tex: configuraciones del documento
%\selectlanguage{spanish}		%Elijo idioma español

%Permitir que los entornos equation, align, etc permitan saltos de página
%\allowdisplaybreaks[1]

%Tweaks
%% \setlength{\parindent}{0mm}		%Sangría de 1a. línea
%% \setlength{\hoffset}{2.6mm}		%
%% \setlength{\voffset}{-5.4mm}		%
%% \setlength{\topmargin}{0mm}		%
%% \setlength{\oddsidemargin}{5mm}	%
%% \setlength{\evensidemargin}{5mm}	%
%% \setlength{\marginparsep}{5mm}	%
%% \setlength{\headheight}{12.5mm}	%
%% \setlength{\headsep}{2.5mm}		%
%% \setlength{\footskip}{10mm}		%
%% \setlength{\textwidth}{14.1cm}		%
%% \setlength{\textheight}{232mm}	%
%% \setlength{\fboxrule}{.1pt}
%% \setlength{\parskip}{.5\baselineskip}

%Colores
\definecolor{negro}	{cmyk}{0,0,0,1}
\definecolor{marron}	{cmyk}{0,.5,1,.41}
\definecolor{rojo}	{cmyk}{0,1,1,0}
\definecolor{naranja}	{cmyk}{0,.35,1,0}
\definecolor{amarillo}	{cmyk}{0,0,1,0}
\definecolor{verde}	{cmyk}{1,0,1,0}
\definecolor{azul}	{cmyk}{1,1,0,0}
\definecolor{violeta}	{cmyk}{.45,1,0,0}
\definecolor{gris}	{cmyk}{0,0,0,.5}
\definecolor{blanco}	{cmyk}{0,0,0,0}
\definecolor{dorado}	{cmyk}{0,.16,1,0}
\definecolor{plateado}	{cmyk}{0,0,0,.25}

%% \title{\titulo}
%% \author{\autor}
%% \date{\fecha}

% si uso pdflatex, me setea las propiedades del pdf de salida
%% \ifpdf\pdfinfo{/Title    (\tituloPDF)
%%                /Author   (\autorPDF)
%%                /Subject  (\asuntoPDF)
%%                /Keywords (\clavesPDF)}\fi

% comandos.tex
% en este archivo defino todos los comandos/environment que quiera usar en mi documento.
%
% ===
% === Comandos
% ===
% 
% T: para escribir texto común cuando en modo math
%    uso: \T{texto que aparecerá en letra normal}
\newcommand{\T}{\textrm}
%
% aclaracion: dibuja un recuadrito aclaratorio, como <quote> en HTML.
%             uso: \aclaracion{Texto...}
\newcommand{\aclaracion}[1]{%
\smallpencil\-\begin{minipage}{0.9\textwidth}
%\vspace*{6pt}
{#1}\smallskip\end{minipage}}
%
% consigna: parecido a aclaración, pero con texto _slanted_
%           uso: \consigna{Consigna...}
\newcommand{\consigna}[1]{%
\leftpointright\ \parbox[t]{0.9\textwidth}{\textsl{#1}\vspace{8pt}}}
%
% pinterno: para representar el producto interno entre los dos argumentos
%           uso: \pinterno{X}{Y}
\newcommand{\pinterno}[2]{%
\left\langle #1 , #2 \right\rangle}
%
% === Estilos de texto
%
% resalt: resaltado con fondo verde
%         uso: \resalt{texto resaltado}
\newcommand{\resalt}{\colorbox{yellow}}
%
% sfbf: texto en negrita + slanted
%       uso:
\newcommand{\sfbf}[1]{\textsf{\bfseries #1}}
%
% small bold sans-serif
\newcommand{\sbs}[1]{\textsf{\small\bfseries #1}}
%
% eng: itálica (para palabras en inglés)
%      uso: \eng{some English text}
\newcommand{\eng}{\textit}
%
% mean: significado de una sigla - slanted
%       uso: (...) SNCF: \mean{Société Nationale des Chemins de Fer Francais} ...
\newcommand{\mean}{\textsl}
\newcommand{\desc}{\textsl}
%
% defin: pone en negrita el texto, útil para definiciones
%        uso: \defin{asshole}: vulgar slang for anus
\newcommand{\defin}{\textbf}
%
% R, N: cambia la tipografía en modo math, probar para ver cómo quedan
%       uso: \R{R} , \N{N}
\newcommand{\R}{\mathds}
\newcommand{\N}{\mathbf}
\newcommand{\C}{\mathcal}
\newcommand{\B}{\boldsymbol}
%
% dx: para escribir d2y/dx2, etc
\newcommand{\dx}[2]{\frac{d^{#2}\!#1}{d\!x^{#2}}}
%
% dp: para escribir derivadas parciales d2y/dx2, etc
\newcommand{\dpar}[3]{\frac{\partial^{#3}#1}{\partial{#2}^{#3}}}
%
% dvar: para escribir derivadas totales d2y/d(VAR)2, etc
\newcommand{\dvar}[3]{\frac{d^{#3}#1}{d{#2}^{#3}}}
%
% evalen: para escribir (loquesea)|_{evaluado_en}
\newcommand{\evalen}[2]{\left.{#2}\right|_{#1}}
%
% lil: para escribir texto pequeño. más cómodo que { \footnotesize texto pequeño... }
%      uso: \lil{texto pequeño... }
\newcommand{\lil}[1]{\footnotesize #1}  %Para texto pequeñooo
%
% mono: escribe el texto que le paso como parámetro con letra de ancho fijo
%       uso: \mono{texto monoespaciado}
\newcommand{\mono}[1]{{\texttt{#1}}}
%
% === Símbolos
%
\newcommand{\y}{\wedge}			%Y (Lógica)
\newcommand{\ve}{\vee}			%O (Lógica)
\newcommand{\ent}{\supset}		%Entonces (Lógica)
\newcommand{\dimp}{\leftrightarrow}	%Doble implicativo, equivalencia (Lógica)
\newcommand{\sii}{\leftrightarrow}	%Si y sólo si (Lógica)
\newcommand{\equi}{\equiv}		%Equivalencia (Lógica)
\newcommand{\portanto}{\vdash}		%Por lo tanto (Lógica)
\newcommand{\por}{\cdot}		%Producto punto
\newcommand{\RR}[1][1]{\mathds{R}}	%R de reales
\newcommand{\hfi}{\hat{\phi}}           %fi con gorrito arriba
\newcommand{\bfi}{\bar{\phi}}           %fi con raya arriba
\newcommand{\hpsi}{\hat{\psi}}          %Letra griega psi con gorrito arriba
\newcommand{\II}{\B{I}}                 %w negrita
\newcommand{\KK}{\B{K}}                 %w negrita
\newcommand{\QQ}{\B{Q}}                 %w negrita
\newcommand{\YY}{\B{Y}}                 %w negrita
\newcommand{\Bg}{\B{g}}                 %w negrita
\newcommand{\nn}{\B{n}}                 %w negrita
\newcommand{\uu}{\B{u}}                 %w negrita
\newcommand{\vv}{\B{v}}                 %w negrita
\newcommand{\ww}{\B{w}}                 %w negrita
\newcommand{\xx}{\B{x}}                 %x negrita
\newcommand{\yy}{\B{y}}                 %y negrita
\newcommand{\zz}{\B{z}}                 %z negrita
\newcommand{\BPhi}{\B{\Phi}}            %\Phi negrita
\newcommand{\Balpha}{\B{\alpha}}        %\alpha negrita
\newcommand{\Bbeta}{\B{\beta}}          %\beta negrita
\newcommand{\Btheta}{\B{\theta}}        %\theta negrita
\newcommand{\Bxi}{\B{\xi}}              %\xi negrita
%
%
% ===
% === Environments
% ===
% 
% enunciado: un environment que básicamente tiene el mismo efecto que el
%            comando consigna.
%            uso: \begin{enunciado} ... contenido ... \end{enunciado}
\newenvironment{enunciado}
{\leftpointright\ \begin{varwidth}[t]{0.9\textwidth}\textsl}
{\end{varwidth}\vspace{8pt}}
%
% pvi: para tipear la definición de un problema de valor inicial/funciones
%      definidas de a trozos/etc directamente en el texto (sin necesidad de
%      cambiar a un modo matemático.
%      uso: \begin{pvi} linea1 \\ linea 2 \\ ... \end{pvi}
\newenvironment{pvi}{\begin{equation}\begin{cases}}
{\end{cases}\end{equation}}
%
% pvi*: comp pvi, pero sin número de ecuación
\newenvironment{pvi*}{\begin{equation*}\begin{cases}}
{\end{cases}\end{equation*}}
%
% verbatimsmall: un verbatim con letra más chica. usualmente queda bastante
%                mejor que el verbatim pelado.
%                uso: \begin{verbatimsmall} ........ \end{verbatimsmall}
\newenvironment{verbatimsmall}{\small\begin{verbatim*}}
{\end{verbatim*}}
%
% nota: escribe una aclaracion dentro del texto
\newenvironment{nota}{$$\left[\;\begin{minipage}{0.95\textwidth}\slshape}
{\end{minipage}\;\right]$$}
%
%
% ===
% === Comandos ``históricos''
% ===
%
%% %\begin{pspicture}
%% \def\tierra(#1){%Para dibujar el símbolo de tierra en el entorno PSTricks
%% 	\rput(#1){
%% 		\psdot(0,0)
%% 		\psline(0,0)(0,-0.45)
%% 		\psline(-0.5,-0.45)(0.5,-0.45)
%% 		\psline(-0.35,-0.6)(0.35,-0.6)
%% 		\psline(-0.2,-0.75)(0.2,-0.75)
%% 	}%
%% }
%% %\end{pspicture}

\newcommand{\codigo}[2]{%Para generar un recuadro con código
	%\setlength{\hrulewidth}{0.1pt}
	\begin{flushleft}
	\underline{#1}
	\begin{tabular}{@{\quad}|l}
		\begin{minipage}{.85\textwidth}\smallskip{#2}
	\end{minipage}\end{tabular}\end{flushleft}%
}

\newcommand{\filecodigo}[1]{%Insertar código verbatim desde un archivo
\codigo{#1}{\verbatiminput{#1}}}%Requiere el paquete verbatim
\newcommand{\filecodigobis}[1]{{\verbatiminput{#1}}}%Requiere el paquete verbatim

%% \newcommand{\grafico}[3][1]{%Para generar un plot de un archivo con coords.
%% %\def\deequis=#1
%% \begin{minipage}{0.5\textwidth}\begin{center}
%% \begin{pspicture}(6,5)
%% 	\psgrid[subgriddiv=1,gridlabels=0pt,gridwidth=.1pt](1,3)(1,1)(6,5)
%% 	\psset{xunit=5cm,yunit=2cm}
%% 	\fileplot[linewidth=1pt,linecolor=blue,origin={0.2,1.5}]{#2}
%% 	\psset{xunit=1cm,yunit=1cm}
%% 	\psaxes[Dx=#1,dx=5,Oy=-1,Dy=1,dy=2]{-}(0.9,1)(6,5)
%% 	\rput(4,0.4){\textsl{#3}}
%% \end{pspicture}\end{center}\end{minipage}}

%% \newcommand{\eqncode}[2]{%
%% \begin{center}
%% \begin{tabular}{l@{\hspace{0.5cm}}r}
%% \begin{minipage}{.4\textwidth}
%% \begin{equation*}
%% #1
%% \end{equation*}
%% \end{minipage}
%% &
%% \fbox{\begin{minipage}{.4\textwidth}
%% %\setlength{\parskip}{4mm}
%% \filecodigobis{#2}
%% \end{minipage}}
%% \end{tabular}
%% \end{center}
%% }

%% \newcommand{\eqncodeb}[2]{%
%% \begin{center}\begin{tabular}{l@{\hspace{0.5cm}}r}
%% \begin{minipage}{.4\textwidth}#1\end{minipage} &
%% \fbox{\begin{minipage}{.4\textwidth}\filecodigobis{#2}\end{minipage}}
%% \end{tabular}\end{center}}

%% \newenvironment{matemcode}[1]{\newline
%% \begin{tabular}{l@{\hspace{0.5cm}}r}
%% \begin{minipage}{.4\textwidth}
%% \parbox[t]{.4\textwidth}{\begin{equation*}#1\end{equation*}}\end{minipage}
%% &\begin{Sbox}\begin{minipage}{.4\textwidth}}
%% {\end{minipage}\end{Sbox}\fbox{\TheSbox}\end{tabular}\newline}

%% \newenvironment{encuadrar}[1]{\begin{Sbox}\begin{varwidth}{#1\textwidth}}
%% {\end{varwidth}\end{Sbox}\fbox{\TheSbox}}

%% \newenvironment{parboxenv}{\begin{Sbox}}
%% {\end{Sbox}\parbox[t]{.9\textwidth}{\TheSbox}}

% multicolumn y multirow
\newcommand{\mcol}[3]{\multicolumn{#1}{#2}{#3}}
\newcommand{\mrow}[3]{\multirow{#1}{#2}{#3}}

%% Serif .......................................................................
%%
%% New Century Schoolbook
%% \usepackage[T1]{fontenc}
%% \usepackage{fouriernc}
%%
%%
%% TeX Gyre Schola (New Century extendida)
%% \usepackage[T1]{fontenc}
%% \usepackage{tgschola}
%%
%%
%% Utopia
%% \usepackage[T1]{fontenc}
%% \usepackage{fourier}
%%
%%
%% Utopia (con MathDesign)
%% \usepackage[T1]{fontenc}
%% \usepackage[adobe-utopia]{mathdesign}
%%
%%
%% Computer Concrete
%% \usepackage[T1]{fontenc}
%% \usepackage{concmath}
%%
%%
%% Charter BT
 \usepackage[T1]{fontenc}
 \usepackage[bitstream-charter]{mathdesign}
%%
%%
%% Nimbus Roman (clon de Times)
%% \usepackage[T1]{fontenc}
%% \usepackage{nimbus}
%%
%%
%% TeX Gyre Termes (version mejorada de Nimbus Roman)
%% \usepackage[T1]{fontenc}
%% \usepackage{tgtermes}
%%
%%
%% GFS Bodoni
%% \usepackage[T1]{fontenc}
%% \usepackage[default]{gfsbodoni}
%%
%%
%% Baskervald ADF
%% \usepackage[T1]{fontenc}
%% \usepackage{baskervald}
%%
%%
%% Efont Serif -- descargar de http://openlab.jp/efont/serif/
%% \usepackage[T1]{fontenc}
%% \usepackage{efont,mathesf}
%% \renewcommand*\oldstylenums[1]{{\fontfamily{esfod}\selectfont#1}}
%%
%%
%%
%%
%%
%% Sans-Serif ..................................................................
%%
%%
%% Optima (clon de, URW Classico)
%% \usepackage[T1]{fontenc}
%% \renewcommand*\sfdefault{uop}
%%
%%
%% Avantgarde (clon de, URW Gothic)
%% \usepackage[T1]{fontenc}
%% \usepackage{avant}
%%
%%
%% TeX Gyre Adventor (version mejorada de Avantgarde)
%% \usepackage[T1]{fontenc}
%% \usepackage{tgadventor}
%%
%%
%% Nimbus Sans (clon de Helvetica)
%% \usepackage[T1]{fontenc}
%% \usepackage{nimbus}
%%
%%
%% Helvetica (clon de, Nimbus Sans)
%% \usepackage[T1]{fontenc}
%% \usepackage[scaled]{helvet}
%%
%%
%% TeX Gyre Heros (version mejorada de Nimbus Sans)
%% \usepackage[T1]{fontenc}
%% \usepackage{tgheros}
%%
%%
%% Boilinum
%% \usepackage[T1]{fontenc}
%% \usepackage{libertine}
%%
%%
%% Computer Modern Bright
%% \usepackage[T1]{fontenc}
%% \usepackage{cmbright}
%%
%%
%% Latin Modern Sans
%% \usepackage[T1]{fontenc}
%% \usepackage{lmodern}
%%
%%
%% Epigrafica
%% \usepackage[OT1]{fontenc}
%% \usepackage{epigrafica}
%%
%%
%%
%% Si quiero el documento en sans en vez de Roman:
%% \renewcommand*\familydefault{\sfdefault}
%% ...............................................
%% 
%%
%%
%% Monospaced ..................................................................
%%
%%
%% Pandora Typewriter
%% \usepackage[T1]{fontenc}
%% \usepackage{pandora}
%%
%%
%% Letter Gothic
%% \usepackage[T1]{fontenc}
%% \usepackage{ulgothic}
%%
%%
%% Inconsolata
%% \usepackage[T1]{fontenc}
%% \usepackage{inconsolata}
%%

%
\addbibresource{res/bibliografia.bib}
%
\selectlanguage{spanish}
\hyphenation{micro-RNA}
\hyphenation{micro-RNAs}
\hyphenation{mi-RNA}
\hyphenation{mi-RNAs}
%
%
%
%
\addtokomafont{descriptionlabel}{\small}
\setkomafont{subject}{\LARGE\usekomafont{disposition}}
\setkomafont{title}{\normalfont\slshape}
\setkomafont{subtitle}{\LARGE\usekomafont{disposition}}
%
\begin{document}
\selectlanguage{spanish}
%
% pagina de titulo
%
\titlehead{\center\large
    Universidad Nacional del Litoral\\
    Facultad de Ingeniería y Ciencias Hídricas
}
%
%
\title{\LARGE ``Desarrollo de un clasificador de secuencias de pre-microRNA
  mediante técnicas de Inteligencia Computacional''}
\subject{Proyecto Final de Carrera\\Ingeniería en
  Informática}
\subtitle{~\\[.2ex]Informe entregable 2\\[.2ex]~}
\author{{Alumno: Mauro Javier Torrez}\and{Director: Dr. Diego H. Milone}}
%
\date{~\\[2em]\today}
%
\renewcommand*{\titlepagestyle}{empty}
%\thispagestyle{empty}
\maketitle
\setcounter{page}{1}
%
%
%
%
\section{Introducción}
En este informe entregable se detalla el trabajo realizado \resalt{\ldots}

Se realizaron pruebas de clasificación con clasificadores SVM y MLP,
variando los parámetros de cada uno, para distintas configuraciones de
datos de entrenamiento/prueba y de conjuntos de características
utilizados.

\section{Clasificadores}
Para las pruebas se utilizaron clasificadores SVM y MLP según se
detallan a continuación.
%
\subsection{SVM}
Para el clasificador SVM se implementó un algoritmo de búsqueda de
parámetros óptimos de tipo \eng{grid-search} \ref{ref-svm-gridsearch}
o \emph{búsqueda en la grilla}.  Este algoritmo consiste en, a partir
de valores iniciales para los parámetros, en cada iteración se genera
un nuevo conjunto de valores nuevo en las cercanías de aquellos que
obtienen el mejor rendimiento.

%% \begin{algorithm}[H]
%% %% \SetKwInput{Datos}{Datos}
%% %% \SetKwInput{Resultado}{Resultado}
%% %% \SetKwInput{Entrada}{Entrada}
%% %% \SetKwInput{Salida}{Salida}
%% %% \SetKw{KwA}{a}
%% %% \SetKw{KwDevolver}{devolver}
%% %% \SetKw{Devolver}{devolver}
%% %% \SetKwBlock{Inicio}{inicio}{fin}
%% %% \SetKwIF{SSi}{EnOtroCasoSi}{EnOtroCaso}{si}{entonces}{sin\'o, si}{sin\'o}{fin si}
%% %% \SetKwSwitch{Seleccionar}{Caso}{Otro}{seleccionar}{hacer}{caso}{sin\'o}{fin caso}{fin seleccionar}
%% %% \SetKwFor{Para}{para}{hacer}{fin para}
%% %% \SetKwFor{ParaPara}{par}{hacer en paralelo}{fin para}
%% %% \SetKwFor{EnParalelo}{para}{hacer en paralelo}{fin para}
%% %% \SetKwFor{Mientras}{mientras}{hacer}{fin mientras}
%% %% \SetKwFor{ParaCada}{para cada}{hacer}{fin para cada}
%% %% \SetKwFor{ParaTodo}{para todo}{hacer}{fin para todo}
%% %% \SetKwRepeat{Repetir}{repetir}{hasta que}
%% \SetAlgoLined
%% \Entrada{clasificador RBF, datos, $\sigma_0$, $C_0$, $R$}
%% \Resultado{$\sigma_{ni},C_{nj} | G_m(\T{Se}(\sigma_{ki},C_{kj}),\T{Sp}(\sigma_{ki},C_{kj})) = \max(G_m) \forall k,i,j$}
%% $C\leftarrow C_0$\;
%% $sigma\leftarrow\sigma_0$\;
%% \Para{$r\in [1,R]$}{ 
%% \ParaCada{$C_l, \sigma_m$}{
%% Se$_k$, Sp$_k$ $\leftarrow$ entrenar($C_l$, $\sigma_m$, datos, \ldots) \; 
%% $G_m(k)\leftarrow\sqrt{\T{Se}_k\cdot\T{Sp}_k}$
%% }
%% $G_m*\leftarrow\{G_m(k):(G_m(k)-\max(Gm(k)))<\T{tolerancia}\}$
%% }
%% \caption{How to write algorithms}
%% \end{algorithm}

Como medida para el rendimiento se utilizó la media geométrica entre
la sensibilidad (tasa de clasificación del conjunto de datos positivo)
y la especificidad (tasa de clasificación del conjunto de datos negativo)
obtenida en cada prueba.

Se realizaron las pruebas con cuatro etapas de refinamiento de los parámetros,
partiendo de una grilla logarítmicamente espaciada en $10^2$ se llega a una
grilla refinada en $10^{0.25}$ con una progresión de $2^{(2-r)}$, con $r$ 
el número de iteración.
\subsubsection{Parámetros}
En el caso del clasificador SVM con kernel de función de base radial,
se refinaron los parametros \emph{boxconstraint} y \emph{sigma}.

En el caso SVM lineal, el parámetro que se procedió a refinar es
el \emph{boxconstraint}.
\subsection{MLP}
Se utilizaron clasificadores MLP con una única capa oculta, variando
el número de neuronas en la capa oculta entre los valores extremos 5 y 25.

Como función de entrenamiento del MLP se utilizó el método \emph{gradiente conjugado escalado}.
%
\section{Conjuntos de datos}
Se generaron 5 configuraciones diferentes para los conjuntos de datos, 
cada una de las cuales especifica los conjuntos de la base de datos 
utilizados tanto para entrenamiento como para prueba.

En la tabla \ref{datasetup} se detallan los conjuntos utilizados en
cada configuración.  A continuación se describen cada una de las
configuraciones.

%
\begin{table}[H]
  \caption{Configuraciones de los conjuntos de datos utilizados para las pruebas}
  \center%\sffamily
  \begin{tabular}{llrlrrrr}\toprule
Config. &
Fuente             & Clase & Especies  & Total  & Train & Test  \\\midrule
\mrow{4}{*}{HSA}
& mirbase20        & +1    & humano    & 1265   & 1200  & 65    \\
& coding           & -1    & humano    & 8494   & 2271  & 6223  \\
& other-ncrna      & -1    & humano    & 129    & 129   & 0     \\
& mirbase20        & +1    & no humano & 11805  & 0     & 11805 \\\midrule
\mrow{4}{*}{ALL}
& mirbase20        & +1    & todas     & 13070  & 2200  & 10870 \\
& functional-ncrna & -1    & varias    & 2650   & 2200  & 450   \\
& coding           & -1    & humano    & 8494   & 0     & 8494  \\
& other-ncrna      & -1    & humano    & 129    & 0     & 129   \\\midrule
\mrow{7}{*}{XUE}
& mirbase50        & +1    & humano    & 193    & 163   &  30   \\
& coding           & -1    & humano    & 8494   & 168   & 1000  \\
& conserved-hairpin& -1?   & humano    & 2444   & 0     & 2444  \\
& updated          & +1    & humano    & 39     & 0     &  39   \\
& mirbase50        & +1    & no-humano & 1017   & 0     & 1017  \\
& mirbase20        & +1    & humano    & 1265   & 0     & 1265  \\
& mirbase20        & +1    & no-humano & 11805  & 0     & 11805 \\\midrule
\mrow{7}{*}{NG}
& mirbase82        & +1    & humano    & 308    & 200   & 108   \\
& coding           & -1    & humano    & 8494   & 400   & 8094  \\
& mirbase82        & +1    & no-humano & 1677   & 0     & 1677  \\
& functional-ncrna & -1    & varias    & 2650   & 0     & 2650  \\
& mirbase20        & +1    & humano    & 1265   & 0     & 1265  \\
& mirbase20        & +1    & no-humano & 11805  & 0     & 11805 \\\midrule
\mrow{5}{*}{BTW}
& mirbase12        & +1    & humano    & 660    & 561   & 99    \\
& coding           & -1    & humano    & 8494   & 1012  & 7482  \\
& other-ncrna      & -1    & humano    & 129    & 110   & 19    \\
& mirbase20        & +1    & humano    & 1265   & 0     & 1265  \\
& mirbase20        & +1    & no-humano & 11805  & 0     & 11805 \\
\bottomrule
  \end{tabular}
  \label{datasetup}
\end{table}
%
\subsection{Configuración HSA}
En esta configuración se utilizan para entrenamiento 1200 entradas del
conjunto de datos positivo \sbs{mirbase20} (humano), y 2400 entradas
de los conjuntos de datos negaivos \sbs{other-ncrna} (humano), y
\sbs{coding} (humano).

El clasificador se evalúa contra las 65 entradas restantes de humanos
de \sbs{mirbase20}, contra otras 6223 del conjunto negativo
\sbs{coding}, y contra las 11805 entradas de \sbs{mirbase20} que no
pertenecen a humanos.
%
\subsection{Configuración ALL}
En esta configuración se utilizan como datos de entrenamiento positivo
2200 entradas del conjunto \sbs{mirbase20} para todas las especies,
tomadas aleatoriamente. Como entrenamiento negativo, se utilizan 2200
entradas del conjunto \sbs{functional-ncrna}, que contiene ncRNAs
funcionales (exceptuando pre-miRNAs) para todas las especies.

El clasificador de prueba contra las 10870 entradas restantes de
\sbs{mirbase20}, discriminando entre entradas de humano/no-humano,
contra las 450 entradas de \sbs{functional-ncrna} no utilizadas para
entrenamiento, y contra las bases de datos negativas \sbs{coding}
(humano) y \sbs{other-ncrna} (humano).
%
\subsection{Configuración XUE}
En esta configuración se replican los conjuntos de entrenamiento y
prueba utilizados en \cite{xue}: entrenamiento con 163 entradas de
humanos de \sbs{mirbase50} y 168 de \sbs{coding}, y prueba contra
\sbs{mirbase50} (30 entradas de humanos, 1017 otras especies),
\sbs{coding} (1000 entradas de humanos), \sbs{conserved-hairpin} (2444
entradas de humanos) y \sbs{updated} (39 entradas humanos).

Se prueba además el clasificador contra la base de datos positiva
\sbs{mirbase20} discriminando entre entradas pertenecientes a humanos
y a otras especies.
%
\subsection{Configuración NG}
En esta configuración se replican los conjuntos de datos utilizados en
\cite{ng}.  Para entrenamiento se utilizan 200 entradas del conjunto
positivo \sbs{mirbase82} y 400 del conjunto negativo \sbs{coding}.

Se prueba el clasificador con las 108 entradas de humanos restantes de
\sbs{mirbase82}, 8094 de \sbs{coding}, 1677 entradas de
\sbs{mirbase82} pertenecientes a otras especies, 2650 entradas
negativas del conjunto \sbs{functional-ncrna}, además de las 13070
entradas del conjunto positivo \sbs{mirbase20} discriminando entre
humano y otras especies.
%
\subsection{Configuración BTW}
Se entrena utilizando los conjuntos de datos utilizados en
\cite{batuwita}, tomando como conjunto positivo el 85\% (561) de las
entradas de humanos disponibles en \sbs{mirbase12}, y como conjunto de
datos negativos se toman 1122 entradas de los conjuntos \sbs{coding}
(1012) y \sbs{other-ncrna} (110).

Se prueba el clasificador con las entradas restantes de los conjuntos
utilizados para entrenamiento: \sbs{mirbase12} (99), \sbs{coding}
(7482) y \sbs{other-ncrna} (19), además de las 13070 disponibles en
\sbs{mirbase20} discriminando entre humano y no-humano.
%


\section{Conjuntos de características}
Considerando los conjuntos de características definidos en la etapa anterior,
se probaron 15 configuraciones diferentes que abarcan todas las combinaciones
posibles. La composición de cada configuración se muestra en la Tabla
\ref{featsets}, marcando con un asterisco (*) el conjunto incluido en la respectiva configuración.

\begin{table}[H]
  \caption{Configuraciones de prueba para las características.}
  \center%\sffamily
  \begin{tabular}{lccccccccccccccc}\toprule
Características / Configuración & 1& 2& 3& 4& 5& 6& 7& 8& 9&10&11&12&13&14&15\\
\midrule
Tripletes                       & *& *&  &  &  & *& *&  & *&  &  &  & *& *& *\\
Tripletes ``extra''             & *&  & *&  &  & *& *&  & *& *& *& *&  &  &  \\
Medidas de la secuencia         & *&  &  & *&  &  &  & *& *& *&  & *&  & *& *\\
Medidas estructura secundaria   & *&  &  &  & *&  & *& *&  & *& *&  & *&  & *\\
\bottomrule
  \end{tabular}
  \label{featsets}
\end{table}


\section{Pruebas de clasificación}

Las pruebas de clasificación se realizaron con los tres
clasificadores, para todas las combinaciones de configuraciones de los
conjuntos de datos y de los conjuntos de características.

Para cada cada prueba, el conjunto de datos de entrenamiento se
particionó en 5, 80\% entrenamiento y 20\% prueba (validación
cruzada), tal que cada conjunto de prueba no contenga entradas en
común con los otros 4. Se repitió el entrenamiento/validación cruzada
para las 5 particiones, y el resultado de la prueba se consideró como
la media aritmética de los 5 resultados obtenidos.


\end{document}


\subsubsection{Características de tripletes}
Vector de 32 elementos de tripletes, calculado según \cite{xue}.
Considerando la región del tallo, contiene el número de ocurrencias
del elemento triplete correspondiente, normalizado entre el número
total de ocurrencias.  El orden es el siguiente: \mono{A..., A..(,
  A.(., A.((, A(.., A(.(, A((., A(((, G..., G..(, G.(., G.((, G(..,
  G(.(, G((., G(((, C..., C..(, C.(., C.((, C(.., C(.(, C((., C(((,
  U..., U..(, U.(., U.((, U(.., U(.(, U((., U(((}.
%
\subsubsection{Características de tripletes ``extra''}
Características auxiliares que se obtienen
al calcular el vector de tripletes.
\begin{description}[style=sameline,leftmargin=3cm]
\item[length3] longitud de la secuencia considerada al
  extraer los elementos de triplete (número de bases que
  conforman el tallo de la estructura de horquilla).
\item[basepairs] número de pares de bases en
  el pre-miRNA.
\item[{length3/basepairs}] grado de complementariedad
  entre los dos brazos de la estructura de horquilla. Para una
  complementariedad perfecta se da el valor mínimo de 2, aumentando cuanto más
  bases ``sueltas'' contenga el tallo.
\item[gc\_content] $=(\T{cant(\mono{G})}+\T{cant(\mono{C})})/\mono{length3}$. Proporción de nucleótidos \mono{G} y
  \mono{C} en el tallo.
\end{description}
%
\subsubsection{Medidas de la secuencia}
%Este grupo contiene las siguientes medidas de la secuencia del pre-miRNA:
\begin{description}[style=sameline,leftmargin=3cm]
\item[length] longitud del pre-miRNA, incluyendo extremos sueltos y el
  bucle central.
\item[A, C, G, U] (x4) número de nucleótidos \mono{A}, \mono{C},
  \mono{G} y \mono{U}, respectivamente.
\item[G+C, A+U] (x2) número de nucleótidos \mono{G} y \mono{C}, y
  \mono{A} y \mono{U} respectivamente.
\item[XY] (x16) número de dinucleótidos (dos nucleótidos contiguos)
  \mono{XY}, con $\mono{X,Y}\in\{\mono{A,C,G,U}\}$. El orden es el
  siguiente: \mono{AC, AG, AU, CA, CC, CG, CU, GA, GC, GG, GU, UA, UC,
    UG, UU}.
\end{description}
%
\subsubsection{Medidas de la estructura secundaria}
%En este grupo se presentan las siguientes características relativas al
%plegado del pre-miRNA en su estructura de horquilla:
\begin{description}[style=sameline,leftmargin=3cm]
\item[MFE] Mínima Energía Libre obtenida al plegar la secuencia con
  \mono{RNAfold}.
\item[MFEI1] $={\mono{MFE}}/{(\mono{G+C})\cdot 100}$, con \mono{G+C}
  de las medidas de secuencia.
\item[MFEI4] $=\mono{MFE}/\mono{basepairs}$ con \mono{basepairs} de
  las características de triplete extra.
\item[dP] \mono{basepairs} normalizada con la longitud total
  \mono{length}.
\item[|A-U|/length] número de pares \mono{A-U} normalizado.
\item[|G-C|/length] número de pares \mono{G-C} normalizado.
\item[|G-U|/length] número de pares \mono{G-U} normalizado.
\end{description}


Para el entrenamiento y validación 


En la fase de pruebas se trabajará con dos conjuntos de datos diferentesgenerados a
partir 



Para las pruebas a realizar en esta etapa se procederá a la generación
de un conjunto de datos propio, a partir del conjunto de datos \sbs{mirbase20}
habiendo aplicado el algoritmo de \emph{clustering} \emph{CD-HIT}\resalt{ref},
y como conjuntos de datos negativos se utilizarán elementos de los conjuntos 
\sbs{coding, functional-ncrna, other-ncrna}.

\subsection*{Clasificadores}

svm -- grid search

mlp
\subsection*{Características}
todas las feats, discriminadas por clase
\subsection*{Configuración de las pruebas}
svm - grid search
mlp - no. de neuronas capa oculta
libsvm?
libfann?
\subsubsection*{XUE}
training, test
\subsubsection*{NG}
training,test
\subsubsection*{BTW}

training, test
\subsubsection*{OWN}
training, test

\subsection*{Resultados}

%\% aciertos

tiempos

Los clasificadores se entrenaron con el mismo conjunto de datos de
\cite{xue}, y se probaron con los conjuntos de prueba \emph{Real} (30
elementos), \emph{Pseudo} (1000 elementos) y \emph{Updated} (39
elementos), también de la misma fuente.

En el caso de SVM-Matlab y MLP, se probaron diversos parámetros hasta
encontrar aquellos que obtienen el mejor rendimiento del clasificador.
En la tabla \ref{testresults} se presentan figuras de rendimiento para
los distintos clasificadores. %En el caso de SVM-Matlab se muestran
%tres resultados representativos correspondientes a diferentes
%parámetros. 
En el caso MLP, se encontró que los resultados no varían
significativamente para distintas configuraciones, con diferencias
menores a 3\% en cada caso.  Se muestran entonces para MLP tasas
representativas.
%
\begin{table}
  \caption{Resultados de las pruebas iniciales de clasificación}
  \center%\sffamily
  \begin{tabular}{lrrr}\toprule
    Clasificador  & \% Real (/30) &
                     \% Pseudo (/1000) & \% Updated (/39) \\\midrule
    Triplet-SVM   &  $93.3$    & $88.1$     & $92.3$     \\
    SVM (libSVM)  & $100.0$    & $87.4$     & $92.3$     \\
    SVM (Matlab)  & $100.0$    & $91.2$     & $89.7$     \\
    %% svm (matlab,2) & $96.7$     & $96.6$     & $48.7$     \\
    %% svm (matlab,3) & $100.0$    & $86.4$     & $97.4$     \\
    MLP           &  $90.0$    & $85.0$     & $91.0$  \\\bottomrule
  \end{tabular}
  \label{testresults}
\end{table}
%

Como se puede observar en la tabla \ref{testresults}, las tasas de
clasificación para los distintos conjuntos resultan satisfactorias, e
incluso sensiblemente mejores a aquellas del trabajo original.  Sin
embargo, estos números deberán tomarse con cuidado, ya que han sido
obtenidos entrenando y validando con particiones estáticas, y podrían
ser resultado de un sobreentrenamiento para estos datos en
particular. Se observa también que el perceptrón multicapa presenta
una buena tasa de clasificación incluso cuando se trata de una única
neurona (sin capas ocultas), siempre para este mismo conjunto de
datos.

Se ha implementado el script \mono{triplet\_libsvm.sh} para las
pruebas con \emph{libsvm}, y los scripts \mono{triplet\_svm.m} y
\mono{triplet\_mlp.m} para las pruebas en Matlab de los clasificadores
SVM y MLP respectivamente. Con el software apropiado, estos scripts
se pueden ejecutar directamente para reproducir los resultados de las
pruebas.

%
%
%
%
%
\section{Armado de la base de datos definitiva}
Para el armado de la base de datos definitiva se tomaron como fuente
los conjuntos de datos y características utilizados en \cite{xue},
\cite{ng} y \cite{batuwita}. Se incorporó al conjunto de datos
la última versión disponible de miRBase \cite{mirbase}.

Como primer paso se procedió a validar el plegado de las secuencias en
las bases de datos originales aplicando el programa RNAfold sobre las
mismas y comprobando que el plegado obtenido fuera el mismo que el
presente en la base de datos original. Se utilizó la versión 1.8.5 de
RNAfold, ya que con la versión actual (2.1) se obtienen plegados
diferentes en la mayoría de los casos.

Una vez validadas las estructuras secundarias se procedió a filtrar
aquellas entradas con bucles múltiples en la estructura secundaria, ya
que las características de triplete no están definidas en estos casos.
De esta manera se eliminó la base de datos ``mRNA'' de \cite{ng} por
completo, ya que todas las entradas en este caso poseen bucles
múltiples.

Luego se procedió a la ectracción y validación de características
extraídas contra las bases de datos respectivas: las características
de triplete y triplete-extra contra la base de datos de \cite{xue},
las características de la secuencia contra la base de \cite{ng}, y las
de estructura secundaria con la base de datos en \cite{batuwita}.

Además de las características calculadas, se incorporan a la base de
datos los datos de la secuencia y estructura secundaria en formato
compatible RNAfold, junto con otro archivo donde se indica la clase
(pre-miRNA real, pseudo-pre-miRNA o indefinido) de cada entrada
correspondiente.  En el archivo \mono{README.md} de la base de datos
se detalla la estructura de directorios generada junto con el formato
de los archivos para cada caso.

Se codificó la herramienta \mono{feats.py} a partir de la utilidad
\mono{fautil.py}, para la extracción de características de archivos
con formato RNAfold. Se ignoraron aquellas características que no se
pueden calcular directamente con la información de la secuencia y de
la estructura secundaria.

Se codificó la herramienta \mono{tests.py} para la validación, así
como los scripts en Bash \mono{validate.sh} y \mono{generate\_db.sh},
los que pueden ser ejecutados directamente, siempre con el software
requerido, para validar y regenerar la misma base de datos a partir de
las fuentes.
%
\subsection{Características extraídas}
%
A continuación se enumeran las características disponibles en la base
de datos definitiva.
%
\subsubsection{Características de tripletes}
Vector de 32 elementos de tripletes, calculado según \cite{xue}.
Considerando la región del tallo, contiene el número de ocurrencias
del elemento triplete correspondiente, normalizado entre el número
total de ocurrencias.  El orden es el siguiente: \mono{A..., A..(,
  A.(., A.((, A(.., A(.(, A((., A(((, G..., G..(, G.(., G.((, G(..,
  G(.(, G((., G(((, C..., C..(, C.(., C.((, C(.., C(.(, C((., C(((,
  U..., U..(, U.(., U.((, U(.., U(.(, U((., U(((}.
%
\subsubsection{Características de tripletes ``extra''}
Características auxiliares que se obtienen
al calcular el vector de tripletes.
\begin{description}[style=sameline,leftmargin=3cm]
\item[length3] longitud de la secuencia considerada al
  extraer los elementos de triplete (número de bases que
  conforman el tallo de la estructura de horquilla).
\item[basepairs] número de pares de bases en
  el pre-miRNA.
\item[{length3/basepairs}] grado de complementariedad
  entre los dos brazos de la estructura de horquilla. Para una
  complementariedad perfecta se da el valor mínimo de 2, aumentando cuanto más
  bases ``sueltas'' contenga el tallo.
\item[gc\_content] $=(\T{cant(\mono{G})}+\T{cant(\mono{C})})/\mono{length3}$. Proporción de nucleótidos \mono{G} y
  \mono{C} en el tallo.
\end{description}
%
\subsubsection{Medidas de la secuencia}
%Este grupo contiene las siguientes medidas de la secuencia del pre-miRNA:
\begin{description}[style=sameline,leftmargin=3cm]
\item[length] longitud del pre-miRNA, incluyendo extremos sueltos y el
  bucle central.
\item[A, C, G, U] (x4) número de nucleótidos \mono{A}, \mono{C},
  \mono{G} y \mono{U}, respectivamente.
\item[G+C, A+U] (x2) número de nucleótidos \mono{G} y \mono{C}, y
  \mono{A} y \mono{U} respectivamente.
\item[XY] (x16) número de dinucleótidos (dos nucleótidos contiguos)
  \mono{XY}, con $\mono{X,Y}\in\{\mono{A,C,G,U}\}$. El orden es el
  siguiente: \mono{AC, AG, AU, CA, CC, CG, CU, GA, GC, GG, GU, UA, UC,
    UG, UU}.
\end{description}
%
\subsubsection{Medidas de la estructura secundaria}
%En este grupo se presentan las siguientes características relativas al
%plegado del pre-miRNA en su estructura de horquilla:
\begin{description}[style=sameline,leftmargin=3cm]
\item[MFE] Mínima Energía Libre obtenida al plegar la secuencia con
  \mono{RNAfold}.
\item[MFEI1] $={\mono{MFE}}/{(\mono{G+C})\cdot 100}$, con \mono{G+C}
  de las medidas de secuencia.
\item[MFEI4] $=\mono{MFE}/\mono{basepairs}$ con \mono{basepairs} de
  las características de triplete extra.
\item[dP] \mono{basepairs} normalizada con la longitud total
  \mono{length}.
\item[|A-U|/length] número de pares \mono{A-U} normalizado.
\item[|G-C|/length] número de pares \mono{G-C} normalizado.
\item[|G-U|/length] número de pares \mono{G-U} normalizado.
\end{description}
%
\subsection{Conjuntos de datos}
A continuación se listan los conjuntos de datos obtenidos en el armado
de la base de datos.
\begin{description}
\item[mirbase50] de \cite{xue}, contiene 1210 pre-miRNAs reales para
  diferentes especies, entre ellas 193 humanos, 112 de gallina, 207 de
  ratón, 172 de rata y 96 de arroz.
\item[updated] de \cite{xue}, contiene 39 pre-miRNAs humanos.
\item[coding] de \cite{xue}, contiene 8494 pseudo pre-miRNAs humanos.
\item[conserved-hairpin] de \cite{xue}, contiene 2444 pseudo
  pre-miRNAs humanos, aunque se conoce que algunos de ellos son
  pre-miRNAs reales. Se establece la clase para estas entradas como
  indeterminada, representada por el valor ambiguo 0.
\item[mirbase82-nr] de \cite{ng}, contiene 1985 pre-miRNAs de miRBase
  8.2 filtrados a 90\% de similaridad, para 40 especies incluyendo
  vertebrados, plantas y virus.
\item[functional-ncrna] de \cite{ng}, contiene 2657 ncRNAs
  funcionales, excepto pre-miRNAs, para varias
  especies.\item[mirbase12] de \cite{batuwita}, contiene 660
  pre-miRNAs humanos no redundantes originalmente de miRBase 12.0.
\item[other-ncrna] de \cite{batuwita}, contiene 129 ncRNAs humanos que
  no son pre-miRNAs.
\item[mirbase20] de \cite{mirbase}, incorpora 21433 pre-miRNAs reales
  que no tienen bucles múltiples de miRBase 20, para 204 especies,
  entre ellos 1801 humanos, 1121 de ratón, 423 de rata y 392 de arroz.
\end{description}
%
%
\renewcommand{\bibfont}{\normalfont\footnotesize}
\printbibliography
%
\end{document}



%% Batuwita et al.: microPred
%% 1. datos disponibles sin plegar, pero con las features calculadas
%% 2. dataset positivo: 695 pre-miRNAs hsa (mirBase 12)
%% 3. datasets negativos:
%% 1. CODING de Xue (8494)
%% 2. otros ncRNAs hsa: 754 (695 con secstruct multi-branched)
%% 1. features (48):
%% 1. 29 idem miPred
%% 2. 2 MFE-related
%% 3. 4 RNAfold-related
%% 4. 6 Mfold-related
%% 5. 7 calculadas con scripts propios



%% Sewer et al.: Mir-abela
%% 1. extrae candidatos de miRNAs usando sliding windows en las regiones del genoma que se sabe hay miRNAs
%% 2. especies: human, mouse, rat
%% 3. algunas features “calculables”:
%% 1. energía
%% 2. long del stem simple más largo
%% 3. long del loop del hairpin
%% 4. proporción de nt A/C/G/U en el stem
%% 5. proporción de pares A-U/C-G/G-U en el stem
%% 1. no hay datos de entrenamiento





%% Hertel & Stadler: RNAmicro
%% 1. no hay datos
%% 2. features sacables:
%% 1. long stem
%% 2. long loop
%% 3. G+C

%% Helvik et al.: Microprocessor SVM
%% 1. no hay datos
%% 2. valida con 332 miRNAs hsa (miRBase 8.0) + 130 miRNAs (miRBase 8.1)
%% 3. plega con RNAfold default
%% 4. muchas features

%% Yousef et al.: BayesmiRNAfind
%% 1. no hay datos (dice que estan como supplementary data pero en Bioinformatics no están, tampoco en el sitio bioinfo.wistar.upenn.edu)
%% 2. dataset positivo: no queda claro de dónde lo sacó (pasando un sliding window por las regiones candidatas? de mirbase?)
%% 3. dataset negativo: 190739 no-miRNAs sacados aleatoriamente de las 3’UTR de mRNAs humanos.

%% Nam et al: ProMiR
%% 1. no hay datos, solo disponible el dataset positivo, sin plegar, en http://rfam.sanger.ac.uk
%% 2. no usa “features”, sino estados de transición en la secuencia para entrenar un clasificador HMM
%% Jiang et al.: MiPred
%% 1. dataset positivo: mirna registry database, release 8.2
%% 2. dataset negativo: CODING de Xue
%% 3. features(34): Xue + MFE + P-value
%% 4. disponible: 163 hsa (+), 168 random (-), sin plegar, sin características extraídas. Aparentemente idem Xue.



%% Huang et al.: MiRFinder
%% 1. datos no tan disponibles (train sin etiquetar, sólo vectores libsvm, test sólo secuencias, sin features) 
%% 2. features (18):
%% 1. 1: Minimum Free Energy
%% 2. 2: The difference of the MFE of the sequence pair
%% 3. 3: The difference of the structure of the sequence pair
%% 4. 4–7: Base pairing and other properties of the 22 mer hypothesized mature miRNA
%% 5. 8: The mutation frequency of the sequence segment pair
%% 6. 9–18: The frequency of the 10 possible secondary structure elements (combinations of 2 adjacent characters) in the pseudo code of stem region (represented by the new syntax)
%% 1. training (+): vectores libSVM sin etiquetar: mirBase 8.2 human, mouse, pig, cattle, dog, sheep
%% 2. training (-): sequence segments extracted from UCSC genome pair-wise alignments (human, mouse)
%% Lim et al.: mirCheck/mirScan
%% 1. no hay datos
%% 2. especies: C. elegans
%% 3. features:
%% 1. base pairing of the miRNA portion of the fold-back
%% 2. base pairing of the rest of the fold-back
%% 3. stringent sequence conservation in the 5Ј half of the miRNA
%% 4. slightly less stringent sequence conservation in the 3Ј half of the miRNA
%% 5.  sequence biases in the first five bases of the miRNA (especially a U at the first position)
%% 6. a tendency toward having symmetric rather than asymmetric internal loops and bulges in the miRNA region
%% 7. and the presence of two to nine consensus base pairs between the miRNA and the terminal loop region, with a preference for 4–6 bp.
%% Gkirtzou et al: MatureBayes
%% 1. datos no disponibles
%% 2. conjunto train:
%% 1. 533 hsa, 422 rno de mirBase 10.0
%% 1. test:
%% 1. entradas nuevas de hsa, rno en mirBase 11.0-14.0
%% 2. especies dme, zebrafish en mirBase 14
%% 1. no se usan features útiles para un clasificador svm
%% Lai et al.: MiRSeeker
%% 1. no hay datos
%% 2. scope del paper distinto al nuestro
%% Ding et al.: MiRenSVM
%% 1. training:
%% 1. (+) 692 hsa, 52 aga de mirBase 12.0
%% 2. (-) 9225 hsa, 92 aga de UTRdb release 22.0 (long 70-150nt)
%% 3. (-) 754 ncRNAs hsa usados en microPred
%% 4. (-) 256 ncRNAs aga de Rfam 9.1 (<150nt)
%% 1. test:
%% 1. (+) 14 hsa, 14 aga nuevos en mirBase 13.0
%% 2. (+) 5328 pre-miRNAs de otras 27 especies
%% 3. (-) ??? aparentemente parte de los de UTRdb en train (usa sólo 5428 para entrenar)
%% 1. datos no disponibles, aunque se pueden armar los datasets a mano (secuencias disponibles, hace falta plegar, armar conjuntos train/test)
%% 2. features de miPred + features de microPred
%% miPredGA
%% 1. train:
%% 1. (+) idem microPred
%% 2. (-) idem microPred (CODING de Xue + 754 ncRNAs)
%% 1. test:
%% 1. separa del dataset de train
%% 1. features:
%% 1. idem microPred, rankeadas según libSVM
%% 1. datos disponibles de microPred. no aporta nada nuevo en lo que hace los datos. 
%% Sheng et al.: mirCos
%% 1. no sirve
%% Xu et al.: miRank
%% 1. features:
%% 1. 1 MFE normalizada
%% 2. 2 base pairing propensity normalizada (1para c/brazo)
%% 3. 1 long del loop normalizada
%% 4. 32 triplets como Xue
%% 1. train:
%% 1. (+) 533 hsa, 38 aga de mirBase 1/9/2007
%% 2. (-) 1000 hsa, 20000+ aga obtenidos escaneando el genoma con ventanas de 90nt y filtrando por caracts. de plegado
%% 1. test: no especificado
%% 2. datos disponibles: sólo hsa, formato svm, sin info de secuencia/estructura secundaria

