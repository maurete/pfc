\documentclass[12pt,bibliography=oldstyle,DIV=14,parskip=full-,titlepage]{scrartcl}
% esto me setea la variable pdf dependiendo del valor de \pdfoutput, que es >0
% sólo cuando estoy usando pdflatex para compilar el documento
%% \newif\ifpdf
%% \ifnum\pdfoutput<0
%% \pdffalse\fi
%% \ifnum\pdfoutput=0
%% \pdffalse\fi
%% \ifnum\pdfoutput>0
%% \pdftrue\fi


%
% ===
% === Trick para detectar si el documento está siendo compilado con pdflatex
% ===
%
% Esto me setea la variable pdf dependiendo del valor de \pdfoutput, que es >0
% sólo cuando estoy usando pdflatex para compilar el documento. Con esto puedo
% hacer  \ifpdf {...} \fi, que se ejecuta colo cuando compilo con pdflatex.
%% \newif\ifpdf
%% \ifnum\pdfoutput<0
%% \pdffalse\fi
%% \ifnum\pdfoutput=0
%% \pdffalse\fi
%% \ifnum\pdfoutput>0
%% \pdftrue\fi
%
% ===
% === I18n / L10n
% ===
%
% babel me da separación de sílabas para palabras en el idioma que le paso como
%       argumento opcional.
\usepackage[spanish,es-tabla,english]{babel}
%
% inputenc define la codificación de caracteres del código fuente, acá utf8.
\usepackage[utf8]{inputenc}
%
% ===
% === Gráficos
% ===
% 
% pst-pdf me permite usar PSTricks con pdflatex. Necesito cargarlo sólo si está
%         definida la variable pdf, por eso está entre \ifpdf ... \fi
%\ifpdf\usepackage{pst-pdf}\fi
%
% color me permite usar colores en el documento.
\usepackage{color}
%
% graphicx me da el comando \includegraphics para insertar imágenes (?)
\usepackage{graphicx}
%
% pstricks es un conjunto de macros basadas en PostScript para TeX, en
%          castellano: me da un entorno pstricks y comandos que uso dentro de
%          éste, que me sirven para dibujar figuras/diagramas/etc de manera
%          relativamente simple.
%\usepackage{pstricks}
%
% pst-circ me da macros para pstricks que me dibujan elementos de circuitos
%\usepackage{pst-circ}
%
% pst-plot me provee de funciones de ploteo para pstricks
%\usepackage{pst-plot}
%
% pst-2dplot me sirve para plotear en pstricks, entorno pstaxes
%\usepackage{pst-2dplot}
%
% ===
% === Verbatims
% ===
%
% verbatim es una reimplementación de los entornos verbatim[*]
%          provee el comando \verbatiminput{archivo} y el entorno comment, que
%          hace que LaTeX ignore directamente todo lo que está adentro
%\usepackage{verbatim}
%
% moreverb implementa el entorno verbatimtab indentando los tabs que encuentre,
%          y también el entorno listing, que pone números de línea al verbatim.
%          Para cambiar el ancho de la tabulacion, uso
%          \renewcommand\verbatimtabsize{<ancho del tab>\relax}
%          También define el entorno boxedverbatim.
%\usepackage{moreverb}
%
% listings me da el entorno lstlisting con resaltado de sintaxis.
%          Para setear el lenguaje del código, hago \lstset{language=<lang>}
%\usepackage{listings}
%
% url es un verbatim para escribir URL's que permite linebreaks dentro de ésta.
%     para usarlo, \url{<URL>}
\usepackage{url}
%
% ===
% === Más packages
% ===
%
%% \usepackage{mdwlist}		%Para listas mas compactas
%% \usepackage{textcomp}		%Para algunos símbolos
%% \usepackage{colortbl}		%Para celdas de colores en tablas
%% \usepackage{fancyhdr}		%Para encabezados/pie
\usepackage{bbold}		%Fuente bb para modo math: \mathbb{R} = reales
\usepackage{dsfont}		%Fuente ds para modo math: \mathds{R} = reales
\usepackage{multirow}		%Para "combinar" celdas en tablas
\usepackage{float}		%Para mejorar cuadros, figuras, etc
%% \usepackage{fancybox}		%Para recuardos de texto con bordes "fancy"
%% \usepackage{dingbat}		%Para dingbats
%\usepackage{marginal}		%Para  notas al margen que no puedo hacer andar
\usepackage{amsmath}		%Para enornos matemáticos mas flexibles
%\usepackage{varwidth}		%varwidth es un minipage que se ajusta al ancho mínimo


\usepackage[backend=biber,sorting=none,style=ieee,eprint=false,url=false]{biblatex} %% style=ieee
%% requiere texlive-bibtex-extra en debian


\usepackage{enumitem}
\setlist{noitemsep}
%% \setlist[description]{noitemsep}
%% \setlist[enumerate]{noitemsep}
%% \setlist[itemize]{noitemsep}

\usepackage{tikz}
\usepackage{pgfkeys}
\usepackage{pgfgantt}

% typearea: uso con koma-script para ajustar márgenes de página.
% vars globales a setear en la clase koma-script: DIV=12, BCOR=margen de ``binding'' para double side
\usepackage{typearea}

% para poder usar footnotes p.ej, adentro de un tabular
\usepackage{footnote}
\makesavenoteenv{tabular}

% para tabulars mas lindos/legibles
\usepackage{booktabs}

%\usepackage{glossaries}

\usepackage[spanish]{algorithm2e}

% para highlight (comando \hl{})
\usepackage{soulutf8}

% para teoremans etc
\usepackage{amsthm}

% para tunear citations
%\usepackage[square,comma,numbers,sort&compress]{natbib}


% config.tex: configuraciones del documento
%\selectlanguage{spanish}		%Elijo idioma español

%Permitir que los entornos equation, align, etc permitan saltos de página
%\allowdisplaybreaks[1]

%Tweaks
%% \setlength{\parindent}{0mm}		%Sangría de 1a. línea
%% \setlength{\hoffset}{2.6mm}		%
%% \setlength{\voffset}{-5.4mm}		%
%% \setlength{\topmargin}{0mm}		%
%% \setlength{\oddsidemargin}{5mm}	%
%% \setlength{\evensidemargin}{5mm}	%
%% \setlength{\marginparsep}{5mm}	%
%% \setlength{\headheight}{12.5mm}	%
%% \setlength{\headsep}{2.5mm}		%
%% \setlength{\footskip}{10mm}		%
%% \setlength{\textwidth}{14.1cm}		%
%% \setlength{\textheight}{232mm}	%
%% \setlength{\fboxrule}{.1pt}
%% \setlength{\parskip}{.5\baselineskip}

%Colores
\definecolor{negro}	{cmyk}{0,0,0,1}
\definecolor{marron}	{cmyk}{0,.5,1,.41}
\definecolor{rojo}	{cmyk}{0,1,1,0}
\definecolor{naranja}	{cmyk}{0,.35,1,0}
\definecolor{amarillo}	{cmyk}{0,0,1,0}
\definecolor{verde}	{cmyk}{1,0,1,0}
\definecolor{azul}	{cmyk}{1,1,0,0}
\definecolor{violeta}	{cmyk}{.45,1,0,0}
\definecolor{gris}	{cmyk}{0,0,0,.5}
\definecolor{blanco}	{cmyk}{0,0,0,0}
\definecolor{dorado}	{cmyk}{0,.16,1,0}
\definecolor{plateado}	{cmyk}{0,0,0,.25}

%% \title{\titulo}
%% \author{\autor}
%% \date{\fecha}

% si uso pdflatex, me setea las propiedades del pdf de salida
%% \ifpdf\pdfinfo{/Title    (\tituloPDF)
%%                /Author   (\autorPDF)
%%                /Subject  (\asuntoPDF)
%%                /Keywords (\clavesPDF)}\fi

% comandos.tex
% en este archivo defino todos los comandos/environment que quiera usar en mi documento.
%
% ===
% === Comandos
% ===
% 
% T: para escribir texto común cuando en modo math
%    uso: \T{texto que aparecerá en letra normal}
\newcommand{\T}{\textrm}
%
% aclaracion: dibuja un recuadrito aclaratorio, como <quote> en HTML.
%             uso: \aclaracion{Texto...}
\newcommand{\aclaracion}[1]{%
\smallpencil\-\begin{minipage}{0.9\textwidth}
%\vspace*{6pt}
{#1}\smallskip\end{minipage}}
%
% consigna: parecido a aclaración, pero con texto _slanted_
%           uso: \consigna{Consigna...}
\newcommand{\consigna}[1]{%
\leftpointright\ \parbox[t]{0.9\textwidth}{\textsl{#1}\vspace{8pt}}}
%
% pinterno: para representar el producto interno entre los dos argumentos
%           uso: \pinterno{X}{Y}
\newcommand{\pinterno}[2]{%
\left\langle #1 , #2 \right\rangle}
%
% === Estilos de texto
%
% resalt: resaltado con fondo verde
%         uso: \resalt{texto resaltado}
\newcommand{\resalt}{\colorbox{yellow}}
%
% sfbf: texto en negrita + slanted
%       uso:
\newcommand{\sfbf}[1]{\textsf{\bfseries #1}}
%
% small bold sans-serif
\newcommand{\sbs}[1]{\textsf{\small\bfseries #1}}
%
% eng: itálica (para palabras en inglés)
%      uso: \eng{some English text}
\newcommand{\eng}{\textit}
%
% mean: significado de una sigla - slanted
%       uso: (...) SNCF: \mean{Société Nationale des Chemins de Fer Francais} ...
\newcommand{\mean}{\textsl}
\newcommand{\desc}{\textsl}
%
% defin: pone en negrita el texto, útil para definiciones
%        uso: \defin{asshole}: vulgar slang for anus
\newcommand{\defin}{\textbf}
%
% R, N: cambia la tipografía en modo math, probar para ver cómo quedan
%       uso: \R{R} , \N{N}
\newcommand{\R}{\mathds}
\newcommand{\N}{\mathbf}
\newcommand{\C}{\mathcal}
\newcommand{\B}{\boldsymbol}
%
% dx: para escribir d2y/dx2, etc
\newcommand{\dx}[2]{\frac{d^{#2}\!#1}{d\!x^{#2}}}
%
% dp: para escribir derivadas parciales d2y/dx2, etc
\newcommand{\dpar}[3]{\frac{\partial^{#3}#1}{\partial{#2}^{#3}}}
%
% dvar: para escribir derivadas totales d2y/d(VAR)2, etc
\newcommand{\dvar}[3]{\frac{d^{#3}#1}{d{#2}^{#3}}}
%
% evalen: para escribir (loquesea)|_{evaluado_en}
\newcommand{\evalen}[2]{\left.{#2}\right|_{#1}}
%
% lil: para escribir texto pequeño. más cómodo que { \footnotesize texto pequeño... }
%      uso: \lil{texto pequeño... }
\newcommand{\lil}[1]{\footnotesize #1}  %Para texto pequeñooo
%
% mono: escribe el texto que le paso como parámetro con letra de ancho fijo
%       uso: \mono{texto monoespaciado}
\newcommand{\mono}[1]{{\texttt{#1}}}
%
% === Símbolos
%
\newcommand{\y}{\wedge}			%Y (Lógica)
\newcommand{\ve}{\vee}			%O (Lógica)
\newcommand{\ent}{\supset}		%Entonces (Lógica)
\newcommand{\dimp}{\leftrightarrow}	%Doble implicativo, equivalencia (Lógica)
\newcommand{\sii}{\leftrightarrow}	%Si y sólo si (Lógica)
\newcommand{\equi}{\equiv}		%Equivalencia (Lógica)
\newcommand{\portanto}{\vdash}		%Por lo tanto (Lógica)
\newcommand{\por}{\cdot}		%Producto punto
\newcommand{\RR}[1][1]{\mathds{R}}	%R de reales
\newcommand{\hfi}{\hat{\phi}}           %fi con gorrito arriba
\newcommand{\bfi}{\bar{\phi}}           %fi con raya arriba
\newcommand{\hpsi}{\hat{\psi}}          %Letra griega psi con gorrito arriba
\newcommand{\II}{\B{I}}                 %w negrita
\newcommand{\KK}{\B{K}}                 %w negrita
\newcommand{\QQ}{\B{Q}}                 %w negrita
\newcommand{\YY}{\B{Y}}                 %w negrita
\newcommand{\Bg}{\B{g}}                 %w negrita
\newcommand{\nn}{\B{n}}                 %w negrita
\newcommand{\uu}{\B{u}}                 %w negrita
\newcommand{\vv}{\B{v}}                 %w negrita
\newcommand{\ww}{\B{w}}                 %w negrita
\newcommand{\xx}{\B{x}}                 %x negrita
\newcommand{\yy}{\B{y}}                 %y negrita
\newcommand{\zz}{\B{z}}                 %z negrita
\newcommand{\BPhi}{\B{\Phi}}            %\Phi negrita
\newcommand{\Balpha}{\B{\alpha}}        %\alpha negrita
\newcommand{\Bbeta}{\B{\beta}}          %\beta negrita
\newcommand{\Btheta}{\B{\theta}}        %\theta negrita
\newcommand{\Bxi}{\B{\xi}}              %\xi negrita
%
%
% ===
% === Environments
% ===
% 
% enunciado: un environment que básicamente tiene el mismo efecto que el
%            comando consigna.
%            uso: \begin{enunciado} ... contenido ... \end{enunciado}
\newenvironment{enunciado}
{\leftpointright\ \begin{varwidth}[t]{0.9\textwidth}\textsl}
{\end{varwidth}\vspace{8pt}}
%
% pvi: para tipear la definición de un problema de valor inicial/funciones
%      definidas de a trozos/etc directamente en el texto (sin necesidad de
%      cambiar a un modo matemático.
%      uso: \begin{pvi} linea1 \\ linea 2 \\ ... \end{pvi}
\newenvironment{pvi}{\begin{equation}\begin{cases}}
{\end{cases}\end{equation}}
%
% pvi*: comp pvi, pero sin número de ecuación
\newenvironment{pvi*}{\begin{equation*}\begin{cases}}
{\end{cases}\end{equation*}}
%
% verbatimsmall: un verbatim con letra más chica. usualmente queda bastante
%                mejor que el verbatim pelado.
%                uso: \begin{verbatimsmall} ........ \end{verbatimsmall}
\newenvironment{verbatimsmall}{\small\begin{verbatim*}}
{\end{verbatim*}}
%
% nota: escribe una aclaracion dentro del texto
\newenvironment{nota}{$$\left[\;\begin{minipage}{0.95\textwidth}\slshape}
{\end{minipage}\;\right]$$}
%
%
% ===
% === Comandos ``históricos''
% ===
%
%% %\begin{pspicture}
%% \def\tierra(#1){%Para dibujar el símbolo de tierra en el entorno PSTricks
%% 	\rput(#1){
%% 		\psdot(0,0)
%% 		\psline(0,0)(0,-0.45)
%% 		\psline(-0.5,-0.45)(0.5,-0.45)
%% 		\psline(-0.35,-0.6)(0.35,-0.6)
%% 		\psline(-0.2,-0.75)(0.2,-0.75)
%% 	}%
%% }
%% %\end{pspicture}

\newcommand{\codigo}[2]{%Para generar un recuadro con código
	%\setlength{\hrulewidth}{0.1pt}
	\begin{flushleft}
	\underline{#1}
	\begin{tabular}{@{\quad}|l}
		\begin{minipage}{.85\textwidth}\smallskip{#2}
	\end{minipage}\end{tabular}\end{flushleft}%
}

\newcommand{\filecodigo}[1]{%Insertar código verbatim desde un archivo
\codigo{#1}{\verbatiminput{#1}}}%Requiere el paquete verbatim
\newcommand{\filecodigobis}[1]{{\verbatiminput{#1}}}%Requiere el paquete verbatim

%% \newcommand{\grafico}[3][1]{%Para generar un plot de un archivo con coords.
%% %\def\deequis=#1
%% \begin{minipage}{0.5\textwidth}\begin{center}
%% \begin{pspicture}(6,5)
%% 	\psgrid[subgriddiv=1,gridlabels=0pt,gridwidth=.1pt](1,3)(1,1)(6,5)
%% 	\psset{xunit=5cm,yunit=2cm}
%% 	\fileplot[linewidth=1pt,linecolor=blue,origin={0.2,1.5}]{#2}
%% 	\psset{xunit=1cm,yunit=1cm}
%% 	\psaxes[Dx=#1,dx=5,Oy=-1,Dy=1,dy=2]{-}(0.9,1)(6,5)
%% 	\rput(4,0.4){\textsl{#3}}
%% \end{pspicture}\end{center}\end{minipage}}

%% \newcommand{\eqncode}[2]{%
%% \begin{center}
%% \begin{tabular}{l@{\hspace{0.5cm}}r}
%% \begin{minipage}{.4\textwidth}
%% \begin{equation*}
%% #1
%% \end{equation*}
%% \end{minipage}
%% &
%% \fbox{\begin{minipage}{.4\textwidth}
%% %\setlength{\parskip}{4mm}
%% \filecodigobis{#2}
%% \end{minipage}}
%% \end{tabular}
%% \end{center}
%% }

%% \newcommand{\eqncodeb}[2]{%
%% \begin{center}\begin{tabular}{l@{\hspace{0.5cm}}r}
%% \begin{minipage}{.4\textwidth}#1\end{minipage} &
%% \fbox{\begin{minipage}{.4\textwidth}\filecodigobis{#2}\end{minipage}}
%% \end{tabular}\end{center}}

%% \newenvironment{matemcode}[1]{\newline
%% \begin{tabular}{l@{\hspace{0.5cm}}r}
%% \begin{minipage}{.4\textwidth}
%% \parbox[t]{.4\textwidth}{\begin{equation*}#1\end{equation*}}\end{minipage}
%% &\begin{Sbox}\begin{minipage}{.4\textwidth}}
%% {\end{minipage}\end{Sbox}\fbox{\TheSbox}\end{tabular}\newline}

%% \newenvironment{encuadrar}[1]{\begin{Sbox}\begin{varwidth}{#1\textwidth}}
%% {\end{varwidth}\end{Sbox}\fbox{\TheSbox}}

%% \newenvironment{parboxenv}{\begin{Sbox}}
%% {\end{Sbox}\parbox[t]{.9\textwidth}{\TheSbox}}

% multicolumn y multirow
\newcommand{\mcol}[3]{\multicolumn{#1}{#2}{#3}}
\newcommand{\mrow}[3]{\multirow{#1}{#2}{#3}}

%% Serif .......................................................................
%%
%% New Century Schoolbook
%% \usepackage[T1]{fontenc}
%% \usepackage{fouriernc}
%%
%%
%% TeX Gyre Schola (New Century extendida)
%% \usepackage[T1]{fontenc}
%% \usepackage{tgschola}
%%
%%
%% Utopia
%% \usepackage[T1]{fontenc}
%% \usepackage{fourier}
%%
%%
%% Utopia (con MathDesign)
%% \usepackage[T1]{fontenc}
%% \usepackage[adobe-utopia]{mathdesign}
%%
%%
%% Computer Concrete
%% \usepackage[T1]{fontenc}
%% \usepackage{concmath}
%%
%%
%% Charter BT
 \usepackage[T1]{fontenc}
 \usepackage[bitstream-charter]{mathdesign}
%%
%%
%% Nimbus Roman (clon de Times)
%% \usepackage[T1]{fontenc}
%% \usepackage{nimbus}
%%
%%
%% TeX Gyre Termes (version mejorada de Nimbus Roman)
%% \usepackage[T1]{fontenc}
%% \usepackage{tgtermes}
%%
%%
%% GFS Bodoni
%% \usepackage[T1]{fontenc}
%% \usepackage[default]{gfsbodoni}
%%
%%
%% Baskervald ADF
%% \usepackage[T1]{fontenc}
%% \usepackage{baskervald}
%%
%%
%% Efont Serif -- descargar de http://openlab.jp/efont/serif/
%% \usepackage[T1]{fontenc}
%% \usepackage{efont,mathesf}
%% \renewcommand*\oldstylenums[1]{{\fontfamily{esfod}\selectfont#1}}
%%
%%
%%
%%
%%
%% Sans-Serif ..................................................................
%%
%%
%% Optima (clon de, URW Classico)
%% \usepackage[T1]{fontenc}
%% \renewcommand*\sfdefault{uop}
%%
%%
%% Avantgarde (clon de, URW Gothic)
%% \usepackage[T1]{fontenc}
%% \usepackage{avant}
%%
%%
%% TeX Gyre Adventor (version mejorada de Avantgarde)
%% \usepackage[T1]{fontenc}
%% \usepackage{tgadventor}
%%
%%
%% Nimbus Sans (clon de Helvetica)
%% \usepackage[T1]{fontenc}
%% \usepackage{nimbus}
%%
%%
%% Helvetica (clon de, Nimbus Sans)
%% \usepackage[T1]{fontenc}
%% \usepackage[scaled]{helvet}
%%
%%
%% TeX Gyre Heros (version mejorada de Nimbus Sans)
%% \usepackage[T1]{fontenc}
%% \usepackage{tgheros}
%%
%%
%% Boilinum
%% \usepackage[T1]{fontenc}
%% \usepackage{libertine}
%%
%%
%% Computer Modern Bright
%% \usepackage[T1]{fontenc}
%% \usepackage{cmbright}
%%
%%
%% Latin Modern Sans
%% \usepackage[T1]{fontenc}
%% \usepackage{lmodern}
%%
%%
%% Epigrafica
%% \usepackage[OT1]{fontenc}
%% \usepackage{epigrafica}
%%
%%
%%
%% Si quiero el documento en sans en vez de Roman:
%% \renewcommand*\familydefault{\sfdefault}
%% ...............................................
%% 
%%
%%
%% Monospaced ..................................................................
%%
%%
%% Pandora Typewriter
%% \usepackage[T1]{fontenc}
%% \usepackage{pandora}
%%
%%
%% Letter Gothic
%% \usepackage[T1]{fontenc}
%% \usepackage{ulgothic}
%%
%%
%% Inconsolata
%% \usepackage[T1]{fontenc}
%% \usepackage{inconsolata}
%%

%
%
%
\selectlanguage{spanish}
\hyphenation{micro-RNA}
\hyphenation{micro-RNAs}
\hyphenation{mi-RNA}
\hyphenation{mi-RNAs}
%
%
%
\setkomafont{subject}{\LARGE\usekomafont{disposition}}
\setkomafont{title}{\normalfont\slshape}
%
\raggedbottom
%
\begin{document}
\selectlanguage{spanish}
%
%% \titlehead{\center\large Universidad Nacional del Litoral\\
%%   Facultad de Ingeniería y Ciencias Hídricas}
%% %
%% \subject{Propuesta de Proyecto Final de Carrera\\Ingeniería en
%%   Informática}
%% %
%% \title{\LARGE ``Desarrollo de un clasificador de secuencias de pre-microRNA
%%   mediante técnicas de Inteligencia Computacional''}
%% %
%% %\subtitle{hola que tal}
%% %
%% \author{{Alumno: Mauro Javier Torrez}\and{Director: Dr. Diego. H. Milone}}
%% %
%% %\publishers{Director: Dr. Diego H. Milone}
%% \date{\-\\[2em]\today}
%% \renewcommand*{\titlepagestyle}{empty}
%% \maketitle
%% \setcounter{page}{1}
%
%
%% \section{Puntos de control y entregables}
%% \subsection{Punto de control 1}
%% Resultados de la revisión bibliográfica, pruebas preliminares y armado
%% de la base de datos definitiva.  Abarca las etapas 1 y 2 y la primera
%% parte de la etapa 3.
%% \begin{description}
%%   \item[Fecha:] 7 de agosto de 2013
%%   \item[Entregable:]
%%   \item
%%     \begin{minipage}{\textwidth}
%%       \medskip
%%       \begin{itemize}
%%       \item Bibliografía consultada
%%       \item Bases de datos recopiladas
%%       \item Métodos de clasificación codificados y pruebas preliminares
%%       \item Auxiliares de extracción de características codificados
%%       \item Armado de la base de datos definitiva
%%       \end{itemize}
%%     \end{minipage}
%% \end{description}

%% \subsection{Punto de control 2}
%% Resultados de la implementación del clasificador definitivo e
%% integración del sistema. Abarca la segunda parte de la etapa 3 y la
%% etapa 4 excepto la interfaz web y la documentación para el usuario
%% final.
%% \begin{description}
%%   \item[Fecha:] 31 de octubre de 2013
%%   \item[Entregable:]
%%   \item
%%     \begin{minipage}{\textwidth}
%%       \medskip
%%       \begin{itemize}
%%       \item Resultados de las pruebas de los distintos clasificadores
%%       \item Especificación del algoritmo de clasificación definitivo:
%%         métodos seleccionados, parámetros, conjunto de características
%%         considerado
%%       \item Algoritmo de clasificación codificado
%%       \item Resultados de la validación del algoritmo de clasificación
%%       \item Codificación del sistema integrado
%%       \item Interfaz de usuario de línea de comandos codificada
%%       \item Resultados de la validación de requerimientos
%%       \end{itemize}
%%     \end{minipage}
%% \end{description}
%% \subsection{Punto de control 3}
%% Interfaz de usuario web, puesta en servicio y documentación para el
%% usuario final. Abarca parte de la etapa 4 y la etapa 5.
%% \begin{description}
%%   \item[Fecha:] 7 de enero de 2014
%%   \item[Entregable:]
%%   \item
%%     \begin{minipage}{\textwidth}
%%       \medskip
%%       \begin{itemize}
%%       \item Documentación de la interfaz de línea de comandos
%%       \item Tecnologías web seleccionadas para la interfaz de usuario
%%       \item Interfaz de usuario web codificada
%%       \item Documentación de la interfaz de usuario web
%%       \item Servidor de prueba puesto en servicio
%%       \end{itemize}
%%     \end{minipage}
%% \end{description}
%% %
%% %
\section*{Anexo: Modificación del cronograma del proyecto}
Conforme a lo conversado con la cátedra, se adjunta al presente
informe de avance una modificación del cronograma original del
proyecto que el alumno se propone cumplimentar.  Se consideran las
tareas restantes 5 a 10 tal como se han definido en el Anteproyecto,
con una carga horaria de 356 horas distribuidas en 20 semanas, a
partir del día lunes 28 de julio de 2014.

El siguiente diagrama abarca desde el día 30 de junio de 2014 al día
12 de diciembre del mismo año.
\begin{center}
\definecolor{barblue}{RGB}{153,204,254}
\definecolor{groupblue}{RGB}{51,102,254}
\definecolor{linkred}{RGB}{165,0,33}
\sffamily
\begin{ganttchart}[
y unit chart=6mm,
x unit=1.1mm,
y unit title=7mm,
hgrid style/.append style={draw=black!5, line width=.75pt},
vgrid={*4{draw=black!15, line width=.75pt},%
  *1{draw=black!40, line width=.75pt}},
canvas/.append style={fill=none, draw=black!40, line width=.75pt},
include title in canvas=true,
title height=1,
title/.append style={draw=black!40, line width=.75pt,fill=none},
title label font=\bfseries\footnotesize,
bar label font=\mdseries\small\color{black!70},
bar/.append style={draw=none,fill=black!63},
bar incomplete/.append style={fill=barblue},
bar height=.6,
%bar label node/.append style={left=2cm},
%bar/.append style={draw=none, fill=barblue},
progress label text=,
bar progress label font=\mdseries\footnotesize\color{black!70},
group incomplete/.append style={fill=groupblue},
group left shift=0,
group right shift=0,
group top shift=.3,
group height=.6,
group/.append style={shape=ganttbar},
milestone/.append style={shape=ganttmilestone,fill=linkred,draw=none},
milestone label font=\slshape\bfseries\small\color{black!80},
milestone height=.55,
milestone top shift=.35,
milestone left shift=-0.5,
milestone right shift=1.5,
milestone label node/.append style={below=8pt,left=0pt},
today=11,
today label=HOY,
today rule/.style={
draw=black!64,
%dash pattern=on 3.5pt off 4.5pt,
line width=1.5pt
},
today label font=\slshape\mdseries\small\color{black!70}
]{-8}{111}
\gantttitle{Jul}{24}\gantttitle{Ago}{21}\gantttitle{Sep}{22}
\gantttitle{Oct}{23}\gantttitle{Nov}{20}\gantttitle{Dic}{10}\\
\gantttitle[title label node/.append style={below left=-1.15ex and -1.1ex},
  title label font=\bfseries\scriptsize]{Semana:\quad\,1}{5}%
\gantttitlelist[title label font=\bfseries\scriptsize]{2,...,24}{5} \\
% 20 horas semanales: 1 dia = 4h
% comienzo cronograma: 14 julio
\ganttgroup{Tarea 5}{-8}{13} \\%19 dias
\ganttbar[progress=100]{5.a}{-8}{2}  \\%8h
\ganttbar[progress=82]{5.b}{3}{13} \\%44h
\ganttgroup{Tarea 6}{14}{33} \\%20 dias
\ganttbar[progress=0]{6.a}{14}{17} \\%16h
\ganttbar[progress=0]{6.b}{18}{21} \\%32h
\ganttbar[progress=0]{6.c}{22}{27} \\%24h
\ganttbar[progress=0]{6.d}{28}{33} \\%24h
% entregable 2
\ganttgroup{Tarea 7}{34}{37} \\%4 dias
\ganttmilestone{Control 2}{37}\\
\ganttgroup{Tarea 8}{38}{70} \\%33 dias
\ganttbar[progress=0]{8.a}{38}{41} \\%16h
\ganttbar[progress=0]{8.b}{42}{53} \\%48h
\ganttbar[progress=0]{8.c}{54}{59}\\%24h
\ganttbar[progress=0]{8.d}{60}{70} \\%44h
% entregable 3
\ganttgroup{Tarea 9}{71}{74} \\%4 dias
\ganttmilestone{Control 3}{74}\\
\ganttgroup{Tarea 10}{75}{109} %39 dias
\end{ganttchart}
\end{center}
%
\subsection*{Puntos de control}

Se establece para el segundo punto de control la fecha 2 de septiembre 
de 2014, y para el tercer punto de control la fecha 22 de octubre de 2014.
%
%
%% \section{Riesgos y estrategias de mitigación}
%
%% \subsection{Problemas en el armado de la base de datos}
%% Al trabajar con bases de datos de origen diverso y sobre las
%% cuales no se tienen garantías de calidad, se presenta el riesgo de
%% encontrar más inconsistencias de lo previsto en la etapa de armado de
%% la base de datos.
%% \begin{description}
%%   \item[Probabilidad:] Media
%%   \item[Impacto:] Retraso en el armado de la base de datos definitiva
%%     al requerirse tiempo extra de depuración
%%   \item[Mitigación:] En caso de encontrarse una base de datos
%%     inconsistente o incompleta, se considerará excluirla en el
%%     armado de la base definitiva.
%% \end{description}
%% %
%% \subsection{Retraso en los tiempos previstos por razones ajenas al proyecto}
%% Al encontrarse el estudiante trabajando en un área que es ajena al
%% desarrollo de este proyecto, se considera el riesgo de una carga
%% laboral excesiva que pudiera provocar un retraso en el desarrollo del
%% actual proyecto.
%% \begin{description}
%%   \item[Probabilidad:] Baja
%%   \item[Impacto:] Retraso en el cumplimiento del cronograma del
%%     proyecto
%%   \item[Mitigación:] Replanificar tareas intentando mantener las
%%     fechas previstas, dedicando más horas de trabajo al desarrollo del
%%     proyecto.
%% \end{description}
%% %
%% \subsection{Problemas de portabilidad, compatibilidad y/o
%%   licencias del software de base para el clasificador en el servidor Web}
%% Dado que los servidores Web en general poseen recursos limitados y un
%% entorno de software administrado diferente a aquel de un equipo de
%% escritorio, se da el riesgo de que el software utilizado como soporte
%% en el sistema (lenguajes de programación, software específico) no
%% pueda ser utilizado en el servidor de interfaz Web.
%% \begin{description}
%%   \item[Probabilidad:] Media
%%   \item[Impacto:] Retraso en la implementación de la interfaz Web,
%%     necesidad de volver a codificar partes del sistema en otro
%%     lenguaje compatible
%%   \item[Mitigación:] Búsqueda de software alternativo a utilizar en el
%%     servidor y recodificación de aquellas partes del sistema
%%     incompatibles. Como último recurso, se implementará el servidor en
%%     la misma máquina de desarrollo, aunque tal elección implique que
%%     el servicio no estará disponible para su utilización en un entorno
%%     de producción.
%% \end{description}
%% %
%% %
%% \section{Recursos necesarios y disponibles}
%% Al momento de iniciar el proyecto, todos los recursos necesarios se
%% encuentran disponibles:
%% \begin{itemize}
%% \item Material bibliográfico
%% \item Servicios:
%%   \begin{itemize}
%%   \item Conexión a Internet
%%   \item Servidor Web (disponible en sinc(i))
%%   \end{itemize}
%% \item Hardware:
%%   \begin{itemize}
%%   \item PC de escritorio (Intel Core 2, 4GB RAM)
%%   \item Notebook (Intel Core i5, 4GB RAM)
%%   \end{itemize}
%% \item Software:
%%   \begin{itemize}
%%   \item Sistema operativo Debian GNU/Linux testing ``Jessie''
%%   \item Entorno de desarrollo/Editor GNU Emacs 24
%%   \item Software científico (GNU Octave/MATLAB)
%%   \item Software de scripting (Bash, Python, Perl)
%%   \item Software de servidor Web (Debian GNU/Linux, Apache
%%     httpd, PHP 5)
%%   \end{itemize}
%% %\item Bases de datos para el desarrollo y pruebas
%% \item Insumos varios:
%%   \begin{itemize}
%%   \item Artículos de librería
%%   \item Impresora y tóner
%%   \item Fotocopias
%%   \item Pendrive
%%   \item Pasajes en colectivo
%%   \end{itemize}
%% \item Recursos humanos: alumno y director.
%% \end{itemize}
%% %
%% %
%% \section{Presupuesto}
%% %% En la tabla a continuación se detalla el presupuesto para el Proyecto
%% %% Final. Para la elaboración del mismo, se tuvieron en cuenta las
%% %% siguientes consideraciones:
%% Para la elaboración del presupuesto, se tuvieron en cuenta las
%% siguientes consideraciones:
%% \begin{enumerate}
%% \item El costo de la hora hombre del Alumno se considera al valor de
%%   mercado de un programador Junior en \$40/h.
%% \item Se considera que el Director dedicará 20 horas mensuales en el
%%   seguimiento, asesoramiento y correcciones a la tarea del Alumno. El
%%   costo de la hora hombre del Director se toma en \$200/h.
%% \item No se asigna costo al software a utilizar en el proyecto, ya que
%%   en principio se utilizará software libre. La decisión de utilizar
%%   software comercial incidirá en el costo indirecto con el valor de la
%%   licencia.
%% \item No se consideran intereses así como tampoco el costo de
%%   oportunidad.
%% \end{enumerate}
%% \begin{center}
%% \sffamily\fontsize{11}{13}\selectfont
%% \newcommand\GR[1]{{\bfseries #1}}
%% \newcommand\SL[1]{{\slshape #1}}
%% \begin{tabular}{p{3.5cm}cr@{ }lrr}\toprule
%%   \GR{Tarea/Concepto}&  \GR{Recurso}  & \mcol{2}{c}{\GR{Cantidad}}
%%                                                      &\GR{Costo unitario}
%%                                                                &\GR{Costo total}\\\midrule
%%     \mrow{3}{*}{Tareas 1 a 4}
%%                      & RRHH alumno    &   192 & hs.  & \$  40  & \$ 7680  \\
%%                      & RRHH director  &    40 & hs.  & \$ 200  & \$ 8000  \\
%%                      & Impresión      &   200 & pág. & \$ 0,2  & \$ 40    \\
%%     \mcol{5}{l}{\quad\SL{Subtotal C.D. tareas 1--4}}           & \SL{\$15720}\\\midrule
%%     \mrow{3}{*}{Tareas 5 a 7}
%%                      & RRHH alumno    &   244 & hs.  & \$  40  & \$ 9760  \\
%%                      & RRHH director  &    40 & hs.  & \$ 200  & \$ 8000  \\
%%                      & Impresión      &   100 & pág. & \$ 0,2  & \$ 20    \\
%%     \mcol{5}{l}{\quad\SL{Subtotal C.D. tareas 5--7}}           & \SL{\$17780}\\\midrule
%%     \mrow{4}{*}{Tareas 8 a 9}
%%                      & RRHH alumno    &   148 & hs.  & \$  40  & \$ 5920  \\
%%                      & RRHH director  &    20 & hs.  & \$ 200  & \$ 4000  \\
%%                      & Servidor Web   & 3000  & hs.  & \$   0\footnote{%
%% Servidor Web disponible en \emph{sinc(i)}.}  & \$ 0     \\
%%                      & Impresión      &   100 & pág. & \$ 0,2  & \$ 30    \\
%%     \mcol{5}{l}{\quad\SL{Subtotal C.D. tareas 8--9}}           & \SL{\$9950} \\\midrule
%%     \mrow{4}{*}{Tarea 10}
%%                      & RRHH alumno    &   156 & hs.  & \$  40  & \$ 6240  \\
%%                      & RRHH director  &    40 & hs.  & \$ 200  & \$ 8000  \\
%%                      & Impresión      &   400 & pág. & \$ 0,2  & \$ 80    \\
%%                      & Encuadernado   &     3 & unid.& \$ 30   & \$ 120   \\
%%     \mcol{5}{l}{\quad\SL{Subtotal C.D. tarea 10}}              & \SL{\$14440} \\\midrule
%%     \mrow{5}{*}{Costos Indirectos}
%%                      & Serv. Internet &     7 & mes  & \$  200 & \$ 1400  \\
%%                      & PC escritorio  &     1 & unid.& \$ 6000 & \$ 6000  \\
%%                      & Transporte     &    80 & pas. & \$ 2,9  & \$ 232   \\
%%                 & Elementos de oficina&\mcol{2}{c}{N/A}&       & \$ 150   \\
%%                      & Software     &\mcol{2}{c}{N/A}& \$ 0    & \$ 0     \\
%%     \mcol{5}{l}{\quad\SL{Subtotal C.I.}}                       & \SL{\$7782} \\\midrule
%%     \mcol{5}{l}{\GR{Costo total del Proyecto}}                 & \GR{\$65672}\\\bottomrule
%%   \end{tabular}
%% \end{center}
%
%
%\printglossaries
\renewcommand{\bibfont}{\normalfont\footnotesize}
\printbibliography[heading=bibnumbered,title=Bibliografía]
%
\end{document}
