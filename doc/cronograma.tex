\documentclass[12pt,bibliography=oldstyle,DIV=14,parskip=full-,titlepage]{scrartcl}
\include{conf/preconfig}
\include{conf/packages}
\include{conf/config}
\include{conf/comandos}
\include{conf/fuentes}
%
%
%
\selectlanguage{spanish}
\hyphenation{micro-RNA}
\hyphenation{micro-RNAs}
\hyphenation{mi-RNA}
\hyphenation{mi-RNAs}
%
%
%
\setkomafont{subject}{\LARGE\usekomafont{disposition}}
\setkomafont{title}{\normalfont\slshape}
%
\raggedbottom
%
\begin{document}
\selectlanguage{spanish}
%
%% \titlehead{\center\large Universidad Nacional del Litoral\\
%%   Facultad de Ingeniería y Ciencias Hídricas}
%% %
%% \subject{Propuesta de Proyecto Final de Carrera\\Ingeniería en
%%   Informática}
%% %
%% \title{\LARGE ``Desarrollo de un clasificador de secuencias de pre-microRNA
%%   mediante técnicas de Inteligencia Computacional''}
%% %
%% %\subtitle{hola que tal}
%% %
%% \author{{Alumno: Mauro Javier Torrez}\and{Director: Dr. Diego. H. Milone}}
%% %
%% %\publishers{Director: Dr. Diego H. Milone}
%% \date{\-\\[2em]\today}
%% \renewcommand*{\titlepagestyle}{empty}
%% \maketitle
%% \setcounter{page}{1}
%
%
%% \section{Puntos de control y entregables}
%% \subsection{Punto de control 1}
%% Resultados de la revisión bibliográfica, pruebas preliminares y armado
%% de la base de datos definitiva.  Abarca las etapas 1 y 2 y la primera
%% parte de la etapa 3.
%% \begin{description}
%%   \item[Fecha:] 7 de agosto de 2013
%%   \item[Entregable:]
%%   \item
%%     \begin{minipage}{\textwidth}
%%       \medskip
%%       \begin{itemize}
%%       \item Bibliografía consultada
%%       \item Bases de datos recopiladas
%%       \item Métodos de clasificación codificados y pruebas preliminares
%%       \item Auxiliares de extracción de características codificados
%%       \item Armado de la base de datos definitiva
%%       \end{itemize}
%%     \end{minipage}
%% \end{description}

%% \subsection{Punto de control 2}
%% Resultados de la implementación del clasificador definitivo e
%% integración del sistema. Abarca la segunda parte de la etapa 3 y la
%% etapa 4 excepto la interfaz web y la documentación para el usuario
%% final.
%% \begin{description}
%%   \item[Fecha:] 31 de octubre de 2013
%%   \item[Entregable:]
%%   \item
%%     \begin{minipage}{\textwidth}
%%       \medskip
%%       \begin{itemize}
%%       \item Resultados de las pruebas de los distintos clasificadores
%%       \item Especificación del algoritmo de clasificación definitivo:
%%         métodos seleccionados, parámetros, conjunto de características
%%         considerado
%%       \item Algoritmo de clasificación codificado
%%       \item Resultados de la validación del algoritmo de clasificación
%%       \item Codificación del sistema integrado
%%       \item Interfaz de usuario de línea de comandos codificada
%%       \item Resultados de la validación de requerimientos
%%       \end{itemize}
%%     \end{minipage}
%% \end{description}
%% \subsection{Punto de control 3}
%% Interfaz de usuario web, puesta en servicio y documentación para el
%% usuario final. Abarca parte de la etapa 4 y la etapa 5.
%% \begin{description}
%%   \item[Fecha:] 7 de enero de 2014
%%   \item[Entregable:]
%%   \item
%%     \begin{minipage}{\textwidth}
%%       \medskip
%%       \begin{itemize}
%%       \item Documentación de la interfaz de línea de comandos
%%       \item Tecnologías web seleccionadas para la interfaz de usuario
%%       \item Interfaz de usuario web codificada
%%       \item Documentación de la interfaz de usuario web
%%       \item Servidor de prueba puesto en servicio
%%       \end{itemize}
%%     \end{minipage}
%% \end{description}
%% %
%% %
\section*{Anexo: Modificación del cronograma del proyecto}
Conforme a lo conversado con la cátedra, se adjunta al presente
informe de avance una modificación del cronograma original del
proyecto que el alumno se propone cumplimentar.  Se consideran las
tareas restantes 5 a 10 tal como se han definido en el Anteproyecto,
con una carga horaria de 356 horas distribuidas en 20 semanas, a
partir del día lunes 28 de julio de 2014.

El siguiente diagrama abarca desde el día 30 de junio de 2014 al día
12 de diciembre del mismo año.
\begin{center}
\definecolor{barblue}{RGB}{153,204,254}
\definecolor{groupblue}{RGB}{51,102,254}
\definecolor{linkred}{RGB}{165,0,33}
\sffamily
\begin{ganttchart}[
y unit chart=6mm,
x unit=1.1mm,
y unit title=7mm,
hgrid style/.append style={draw=black!5, line width=.75pt},
vgrid={*4{draw=black!15, line width=.75pt},%
  *1{draw=black!40, line width=.75pt}},
canvas/.append style={fill=none, draw=black!40, line width=.75pt},
include title in canvas=true,
title height=1,
title/.append style={draw=black!40, line width=.75pt,fill=none},
title label font=\bfseries\footnotesize,
bar label font=\mdseries\small\color{black!70},
bar/.append style={draw=none,fill=black!63},
bar incomplete/.append style={fill=barblue},
bar height=.6,
%bar label node/.append style={left=2cm},
%bar/.append style={draw=none, fill=barblue},
progress label text=,
bar progress label font=\mdseries\footnotesize\color{black!70},
group incomplete/.append style={fill=groupblue},
group left shift=0,
group right shift=0,
group top shift=.3,
group height=.6,
group/.append style={shape=ganttbar},
milestone/.append style={shape=ganttmilestone,fill=linkred,draw=none},
milestone label font=\slshape\bfseries\small\color{black!80},
milestone height=.55,
milestone top shift=.35,
milestone left shift=-0.5,
milestone right shift=1.5,
milestone label node/.append style={below=8pt,left=0pt},
today=11,
today label=HOY,
today rule/.style={
draw=black!64,
%dash pattern=on 3.5pt off 4.5pt,
line width=1.5pt
},
today label font=\slshape\mdseries\small\color{black!70}
]{-8}{111}
\gantttitle{Jul}{24}\gantttitle{Ago}{21}\gantttitle{Sep}{22}
\gantttitle{Oct}{23}\gantttitle{Nov}{20}\gantttitle{Dic}{10}\\
\gantttitle[title label node/.append style={below left=-1.15ex and -1.1ex},
  title label font=\bfseries\scriptsize]{Semana:\quad\,1}{5}%
\gantttitlelist[title label font=\bfseries\scriptsize]{2,...,24}{5} \\
% 20 horas semanales: 1 dia = 4h
% comienzo cronograma: 14 julio
\ganttgroup{Tarea 5}{-8}{13} \\%19 dias
\ganttbar[progress=100]{5.a}{-8}{2}  \\%8h
\ganttbar[progress=82]{5.b}{3}{13} \\%44h
\ganttgroup{Tarea 6}{14}{33} \\%20 dias
\ganttbar[progress=0]{6.a}{14}{17} \\%16h
\ganttbar[progress=0]{6.b}{18}{21} \\%32h
\ganttbar[progress=0]{6.c}{22}{27} \\%24h
\ganttbar[progress=0]{6.d}{28}{33} \\%24h
% entregable 2
\ganttgroup{Tarea 7}{34}{37} \\%4 dias
\ganttmilestone{Control 2}{37}\\
\ganttgroup{Tarea 8}{38}{70} \\%33 dias
\ganttbar[progress=0]{8.a}{38}{41} \\%16h
\ganttbar[progress=0]{8.b}{42}{53} \\%48h
\ganttbar[progress=0]{8.c}{54}{59}\\%24h
\ganttbar[progress=0]{8.d}{60}{70} \\%44h
% entregable 3
\ganttgroup{Tarea 9}{71}{74} \\%4 dias
\ganttmilestone{Control 3}{74}\\
\ganttgroup{Tarea 10}{75}{109} %39 dias
\end{ganttchart}
\end{center}
%
\subsection*{Puntos de control}

Se establece para el segundo punto de control la fecha 2 de septiembre 
de 2014, y para el tercer punto de control la fecha 22 de octubre de 2014.
%
%
%% \section{Riesgos y estrategias de mitigación}
%
%% \subsection{Problemas en el armado de la base de datos}
%% Al trabajar con bases de datos de origen diverso y sobre las
%% cuales no se tienen garantías de calidad, se presenta el riesgo de
%% encontrar más inconsistencias de lo previsto en la etapa de armado de
%% la base de datos.
%% \begin{description}
%%   \item[Probabilidad:] Media
%%   \item[Impacto:] Retraso en el armado de la base de datos definitiva
%%     al requerirse tiempo extra de depuración
%%   \item[Mitigación:] En caso de encontrarse una base de datos
%%     inconsistente o incompleta, se considerará excluirla en el
%%     armado de la base definitiva.
%% \end{description}
%% %
%% \subsection{Retraso en los tiempos previstos por razones ajenas al proyecto}
%% Al encontrarse el estudiante trabajando en un área que es ajena al
%% desarrollo de este proyecto, se considera el riesgo de una carga
%% laboral excesiva que pudiera provocar un retraso en el desarrollo del
%% actual proyecto.
%% \begin{description}
%%   \item[Probabilidad:] Baja
%%   \item[Impacto:] Retraso en el cumplimiento del cronograma del
%%     proyecto
%%   \item[Mitigación:] Replanificar tareas intentando mantener las
%%     fechas previstas, dedicando más horas de trabajo al desarrollo del
%%     proyecto.
%% \end{description}
%% %
%% \subsection{Problemas de portabilidad, compatibilidad y/o
%%   licencias del software de base para el clasificador en el servidor Web}
%% Dado que los servidores Web en general poseen recursos limitados y un
%% entorno de software administrado diferente a aquel de un equipo de
%% escritorio, se da el riesgo de que el software utilizado como soporte
%% en el sistema (lenguajes de programación, software específico) no
%% pueda ser utilizado en el servidor de interfaz Web.
%% \begin{description}
%%   \item[Probabilidad:] Media
%%   \item[Impacto:] Retraso en la implementación de la interfaz Web,
%%     necesidad de volver a codificar partes del sistema en otro
%%     lenguaje compatible
%%   \item[Mitigación:] Búsqueda de software alternativo a utilizar en el
%%     servidor y recodificación de aquellas partes del sistema
%%     incompatibles. Como último recurso, se implementará el servidor en
%%     la misma máquina de desarrollo, aunque tal elección implique que
%%     el servicio no estará disponible para su utilización en un entorno
%%     de producción.
%% \end{description}
%% %
%% %
%% \section{Recursos necesarios y disponibles}
%% Al momento de iniciar el proyecto, todos los recursos necesarios se
%% encuentran disponibles:
%% \begin{itemize}
%% \item Material bibliográfico
%% \item Servicios:
%%   \begin{itemize}
%%   \item Conexión a Internet
%%   \item Servidor Web (disponible en sinc(i))
%%   \end{itemize}
%% \item Hardware:
%%   \begin{itemize}
%%   \item PC de escritorio (Intel Core 2, 4GB RAM)
%%   \item Notebook (Intel Core i5, 4GB RAM)
%%   \end{itemize}
%% \item Software:
%%   \begin{itemize}
%%   \item Sistema operativo Debian GNU/Linux testing ``Jessie''
%%   \item Entorno de desarrollo/Editor GNU Emacs 24
%%   \item Software científico (GNU Octave/MATLAB)
%%   \item Software de scripting (Bash, Python, Perl)
%%   \item Software de servidor Web (Debian GNU/Linux, Apache
%%     httpd, PHP 5)
%%   \end{itemize}
%% %\item Bases de datos para el desarrollo y pruebas
%% \item Insumos varios:
%%   \begin{itemize}
%%   \item Artículos de librería
%%   \item Impresora y tóner
%%   \item Fotocopias
%%   \item Pendrive
%%   \item Pasajes en colectivo
%%   \end{itemize}
%% \item Recursos humanos: alumno y director.
%% \end{itemize}
%% %
%% %
%% \section{Presupuesto}
%% %% En la tabla a continuación se detalla el presupuesto para el Proyecto
%% %% Final. Para la elaboración del mismo, se tuvieron en cuenta las
%% %% siguientes consideraciones:
%% Para la elaboración del presupuesto, se tuvieron en cuenta las
%% siguientes consideraciones:
%% \begin{enumerate}
%% \item El costo de la hora hombre del Alumno se considera al valor de
%%   mercado de un programador Junior en \$40/h.
%% \item Se considera que el Director dedicará 20 horas mensuales en el
%%   seguimiento, asesoramiento y correcciones a la tarea del Alumno. El
%%   costo de la hora hombre del Director se toma en \$200/h.
%% \item No se asigna costo al software a utilizar en el proyecto, ya que
%%   en principio se utilizará software libre. La decisión de utilizar
%%   software comercial incidirá en el costo indirecto con el valor de la
%%   licencia.
%% \item No se consideran intereses así como tampoco el costo de
%%   oportunidad.
%% \end{enumerate}
%% \begin{center}
%% \sffamily\fontsize{11}{13}\selectfont
%% \newcommand\GR[1]{{\bfseries #1}}
%% \newcommand\SL[1]{{\slshape #1}}
%% \begin{tabular}{p{3.5cm}cr@{ }lrr}\toprule
%%   \GR{Tarea/Concepto}&  \GR{Recurso}  & \mcol{2}{c}{\GR{Cantidad}}
%%                                                      &\GR{Costo unitario}
%%                                                                &\GR{Costo total}\\\midrule
%%     \mrow{3}{*}{Tareas 1 a 4}
%%                      & RRHH alumno    &   192 & hs.  & \$  40  & \$ 7680  \\
%%                      & RRHH director  &    40 & hs.  & \$ 200  & \$ 8000  \\
%%                      & Impresión      &   200 & pág. & \$ 0,2  & \$ 40    \\
%%     \mcol{5}{l}{\quad\SL{Subtotal C.D. tareas 1--4}}           & \SL{\$15720}\\\midrule
%%     \mrow{3}{*}{Tareas 5 a 7}
%%                      & RRHH alumno    &   244 & hs.  & \$  40  & \$ 9760  \\
%%                      & RRHH director  &    40 & hs.  & \$ 200  & \$ 8000  \\
%%                      & Impresión      &   100 & pág. & \$ 0,2  & \$ 20    \\
%%     \mcol{5}{l}{\quad\SL{Subtotal C.D. tareas 5--7}}           & \SL{\$17780}\\\midrule
%%     \mrow{4}{*}{Tareas 8 a 9}
%%                      & RRHH alumno    &   148 & hs.  & \$  40  & \$ 5920  \\
%%                      & RRHH director  &    20 & hs.  & \$ 200  & \$ 4000  \\
%%                      & Servidor Web   & 3000  & hs.  & \$   0\footnote{%
%% Servidor Web disponible en \emph{sinc(i)}.}  & \$ 0     \\
%%                      & Impresión      &   100 & pág. & \$ 0,2  & \$ 30    \\
%%     \mcol{5}{l}{\quad\SL{Subtotal C.D. tareas 8--9}}           & \SL{\$9950} \\\midrule
%%     \mrow{4}{*}{Tarea 10}
%%                      & RRHH alumno    &   156 & hs.  & \$  40  & \$ 6240  \\
%%                      & RRHH director  &    40 & hs.  & \$ 200  & \$ 8000  \\
%%                      & Impresión      &   400 & pág. & \$ 0,2  & \$ 80    \\
%%                      & Encuadernado   &     3 & unid.& \$ 30   & \$ 120   \\
%%     \mcol{5}{l}{\quad\SL{Subtotal C.D. tarea 10}}              & \SL{\$14440} \\\midrule
%%     \mrow{5}{*}{Costos Indirectos}
%%                      & Serv. Internet &     7 & mes  & \$  200 & \$ 1400  \\
%%                      & PC escritorio  &     1 & unid.& \$ 6000 & \$ 6000  \\
%%                      & Transporte     &    80 & pas. & \$ 2,9  & \$ 232   \\
%%                 & Elementos de oficina&\mcol{2}{c}{N/A}&       & \$ 150   \\
%%                      & Software     &\mcol{2}{c}{N/A}& \$ 0    & \$ 0     \\
%%     \mcol{5}{l}{\quad\SL{Subtotal C.I.}}                       & \SL{\$7782} \\\midrule
%%     \mcol{5}{l}{\GR{Costo total del Proyecto}}                 & \GR{\$65672}\\\bottomrule
%%   \end{tabular}
%% \end{center}
%
%
%\printglossaries
\renewcommand{\bibfont}{\normalfont\footnotesize}
\printbibliography[heading=bibnumbered,title=Bibliografía]
%
\end{document}
